\exercisesheader{}

% 13 - scope_airpoll

\eoce{\qt{Air pollution and birth outcomes, scope of inference\label{scope_airpoll}} 
Exercise~\ref{study_components_airpoll} introduces a study where researchers 
collected data to examine the relationship between air pollutants and preterm 
births in Southern California. During the study air pollution levels were 
measured by air quality monitoring stations. Length of gestation data were 
collected on 143,196 births between the years 1989 and 1993, and air pollution 
exposure during gestation was calculated for each birth.
\begin{parts}
\item Identify the population of interest and the sample in this study.
\item Comment on whether or not the results of the study can be generalized to the 
population, and if the findings of the study can be used to establish causal relationships.
\end{parts}
}{}

% 14 - scope_cheaters

\eoce{\qt{Cheaters, scope of inference\label{scope_cheaters}} 
Exercise~\ref{study_components_cheaters} introduces a study where researchers 
studying the relationship between honesty, age, and self-control conducted an 
experiment on 160 children between the ages of 5 and 15. The researchers asked 
each child to toss a fair coin in private and to record the outcome (white or black) 
on a paper sheet, and said they would only reward children who report white. 
Half the students were explicitly told not to cheat and the others were not given 
any explicit instructions. Differences were observed in the cheating rates in the
instruction and no instruction groups, as well as some differences across 
children's characteristics within each group.
\begin{parts}
\item Identify the population of interest and the sample in this study.
\item Comment on whether or not the results of the study can be generalized to the 
population, and if the findings of the study can be used to establish causal 
relationships.
\end{parts}
}{}

% 15 - scope_buteyko

\eoce{\qt{Buteyko method, scope of inference\label{scope_buteyko}} 
Exercise~\ref{study_components_buteyko} introduces a study on using the Buteyko 
shallow breathing technique to reduce asthma symptoms and improve quality of life.
As part of this study 600 asthma patients aged 18-69 who relied on medication for 
asthma treatment were recruited and randomly assigned to two groups: one practiced 
the Buteyko method and the other did not. Those in the Buteyko group experienced,
on average, a significant reduction in asthma symptoms and an improvement in quality 
of life.
\begin{parts}
\item Identify the population of interest and the sample in this study.
\item Comment on whether or not the results of the study can be generalized to the 
population, and if the findings of the study can be used to establish causal 
relationships.
\end{parts}
}{}

% 16 - scope_stealers

\eoce{\qt{Stealers, scope of inference\label{scope_stealers}} 
Exercise~\ref{study_components_stealers} introduces a study on the relationship 
between socio-economic class and unethical behavior. As part of this study 129 
University of California Berkeley undergraduates were asked to identify themselves 
as having low or high social-class by comparing themselves to others with the most 
(least) money, most (least) education, and most (least) respected jobs. They were 
also presented  with a jar of individually wrapped candies and informed that the
candies were for children in a nearby laboratory, but that they could take some if 
they wanted. After completing some unrelated tasks, participants reported the 
number of candies they had taken. It was found that those who were identified as 
upper-class took more candy than others.
\begin{parts}
\item Identify the population of interest and the sample in this study.
\item Comment on whether or not the results of the study can be generalized to the 
population, and if the findings of the study can be used to establish causal 
relationships.
\end{parts}
}{}

% 17 - relax_after_work_definitions

\eoce{\qt{Relaxing after work\label{relax_after_work_definitions}} The General 
Social Survey asked the question, ``After an average work day, about how many 
hours do you have to relax or pursue activities that you enjoy?" to a random 
sample of 1,155 Americans. The average relaxing time was found to be 1.65 
hours. Determine which of the following is an observation, a variable, a 
sample statistic (value calculated based on the observed sample), or a 
population parameter.
\begin{parts}
\item An American in the sample.
\item Number of hours spent relaxing after an average work day.
\item 1.65.
\item Average number of hours all Americans spend relaxing after an average 
work day.
\end{parts}
}{}

\D{\newpage}
% 18 - cats_on_youtube_definitions

\eoce{\qt{Cats on YouTube\label{cats_on_youtube_definitions}} Suppose you want to 
estimate the percentage of videos on YouTube that are cat videos. It is 
impossible for you to watch all videos on YouTube so you use a random video 
picker to select 1000 videos for you. You find that 2\% of these videos are 
cat videos.Determine which of the following is an observation, a variable, 
a sample statistic (value calculated based on the observed sample), 
or a population parameter.
\begin{parts}
\item Percentage of all videos on YouTube that are cat videos.
\item 2\%.
\item A video in your sample.
\item Whether or not a video is a cat video.
\end{parts}
}{}
