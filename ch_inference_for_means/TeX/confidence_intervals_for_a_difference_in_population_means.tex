\exercisesheader{}

% 37 - air_quality_shortened

\eoce{\qt{Air quality\label{air_quality_shortened}}
Air quality measurements were collected in 
a random sample of 25 country capitals in 2013, and then again in the same 
cities in 2014. We would like to use these data to compare
average air quality between the two years.
Should we use a paired or non-paired test? Explain your reasoning.
}{}

% 38 - fuel_eff_hway_ahss

\eoce{\qt{Fuel efficiency of manual and automatic cars, Part II\label{fuel_eff_hway_ahss}} 
Each year the US Environmental Protection Agency (EPA)
releases fuel economy data on cars manufactured in that year.
Below are summary statistics on fuel efficiency (in miles/gallon)
from random samples of cars with manual and automatic transmissions.
Use these statistics to calculate a 98\% confidence interval
for the difference between average highway mileage of manual
and automatic cars, and interpret this interval in the context
of the data.
 \footfullcite{data:epaMPG}
 
\noindent\begin{minipage}[c]{0.38\textwidth}
\begin{center}
\begin{tabular}{l c c }
\hline
        & \multicolumn{2}{c}{Hwy MPG} \\
\hline
            & Automatic     & Manual         \\
Mean    & 22.92         & 27.88          \\
SD      & 5.29          & 5.01           \\
n       & 26            & 26 \\
\hline
& \\
& \\
\end{tabular}
\end{center}
\end{minipage}
\begin{minipage}[c]{0.6\textwidth}
\begin{center}
\FigureFullPath[A side-by-side box plot is shown for "Highway MPG" for "automatic" and "manual" cars. The "automatic" box plot has its box spanning approximately 20 to 26, has a median of about 23, and its whiskers extending down to about 14 and up to about 34. The "manual" box plot has its box spanning approximately 26 to 32, has a median of about 29, and its whiskers extending down to about 17 and up to about 38.]{0.7}{ch_inference_for_means/figures/eoce/fuel_eff_hway_ahss/fuel_eff_hway_box}
\end{center}
\end{minipage}
}{}

% 39 - chick_wts_casein_soybean_ahss

\eoce{\qt{Chicken diet and weight, Part I\label{chick_wts_casein_soybean_ahss}} 
Chicken farming is a multi-billion dollar industry,
and any methods that increase the growth rate of young
chicks can reduce consumer costs while increasing
company profits, possibly by millions of dollars.
An experiment was conducted to measure and compare
the effectiveness of various feed supplements on the
growth rate of chickens.
Newly hatched chicks were randomly allocated into six groups, 
and each group was given a different feed supplement.
Below are some summary statistics from this data set along
with box plots showing the distribution of weights by
feed type.\footfullcite{data:chickwts}

\noindent\begin{minipage}[c]{0.65\textwidth}
\begin{center}
\FigureFullPath[A side-by-side box plot is shown for "Weight, in grams" for several feed types. The width of the data range for each feed type spans about 150 grams. However, they are centered at different locations: about 325 for "casein", about 150 for "horsebean", about 225 for "linseed", about 275 for "meatmeal", about 250 for "soybean", and about 325 for "sunflower".]{}{ch_inference_for_means/figures/eoce/chick_wts_casein_soybean_ahss/chick_wts_box}
\end{center}
\end{minipage}
\begin{minipage}[c]{0.35\textwidth}
{\footnotesize\begin{tabular}{l c c c}
\hline
            & Mean      & SD        & n \\
\hline
casein          & 323.58        & 64.43 & 12 \\
horsebean   & 160.20        & 38.63 & 10 \\
linseed         & 218.75        & 52.24 & 12 \\
meatmeal    & 276.91        & 64.90 & 11 \\
soybean         & 246.43        & 54.13 & 14 \\
sunflower       & 328.92        & 48.84 & 12 \\
\hline
\end{tabular}}
\end{minipage} 

Casein is a common weight gain supplement for humans. How does it compare to soybean as a feed supplement?  Construct a 95\% confidence interval to estimate the difference between average weight of chickens that would be fed casein and average weight of chickens that would be fed soybean.  Use the Identify, Check, Calculate, Conclude framework.  Is there evidence that average weight of chickens that would be fed casein is higher than average weight of chickens that would be fed soybean?  Justify your answer based on the confidence interval.
}{}

% 40 - work_hours_education_ahss

\eoce{\qt{Work hours and education\label{work_hours_education_ahss}} The General Social Survey 
collects data on demographics, education, and work, among many other characteristics 
of US adults. \footfullcite{data:gss}  Below are the summary statistics for hours worked for those with less than a high school degree and for those with a Bachelor's degree.  
\begin{center}

\begin{tabular}{l  r  r }
                & Less than HS  &  Bachelor's \\
\hline
Mean            & 38.67         & 42.55   \\
SD              & 15.81     & 13.62    \\
n               & 121                & 253   \\
\hline
\end{tabular}
\end{center}
\begin{parts}
\item
We would like to estimate the difference in hours worked between US adults with less than a high school degree and US adults with a Bachelor's degree. Discuss whether conditions for a two-sample $t$-interval for a difference in means are met.  
\item
Assuming conditions are met, construct a 95\% confidence interval to estimate the difference in hours worked between US adults with less than a high school degree and US adults with a Bachelor's degree.  Based on the interval, is there evidence that the average hours worked for these two groups differ?
\end{parts}	
}{}
