\exercisesheader{}

% 25 - find_T_pval_1_2_sided

\eoce{\qt{Find the p-value, Part I\label{find_T_pval_1_2_sided}}
A random sample 
is selected from an approximately normal population
with an unknown standard 
deviation.
Find the p-value for the given sample size and test statistic.
Also determine if the null hypothesis would be rejected at 
$\alpha = 0.05$.
\begin{parts}
\item $n = 11$, $T = 1.91$
\item $n = 17$, $T = -3.45$
\item $n = 7$, $T = 0.83$
\item $n = 28$, $T = 2.13$
\end{parts}
}{}

% 26 - find_T_pval_2_2_sided

\eoce{\qt{Find the p-value, Part II\label{find_T_pval_2_2_sided}}
A random sample 
is selected from an approximately normal population
with an unknown standard 
deviation.
Find the p-value for the given sample size and test statistic.
Also determine if the null hypothesis would be rejected at 
$\alpha = 0.01$.
\begin{parts}
\item $n = 26$, $T = 2.485$
\item $n = 18$, $T = 0.5$
\end{parts}
}{}

% 27 - online_communication

\eoce{\qt{Online communication\label{online_communication}} A study suggests that the 
average college student spends 10 hours per week communicating with others 
online. You believe that this is an underestimate and decide to collect your 
own sample for a hypothesis test. You randomly sample 60 students from your 
dorm and find that on average they spent 13.5 hours a week communicating with 
others online. A friend of yours, who offers to help you with the hypothesis 
test, comes up with the following set of hypotheses. Indicate any errors you see.
\begin{align*}
H_0&: \bar{x} < 10~hours \\
H_A&: \bar{x} > 13.5~hours
\end{align*}
}{}

% 28 - mean_travel_hyp_errors

\eoce{\qt{Mean travel time to work\label{mean_travel_hyp_errors}} Suppose it is known that the mean travel time to work for adults in the US is 25 minutes.  A social scientist thinks this value is different for the adults in her county.  She takes a random sample of 100 adults in her county and finds that their average travel time to work is 26.4 minutes.  
Below is how she set up her hypotheses to test if her county average is different than the US average.
Indicate any errors you see.
\begin{align*}
H_0&: \bar{x} \neq 26.4~minutes \\
H_A&: \bar{x} = 26.4~minutes
\end{align*}
}{}

% 29 - ny_sleep_habits_2_sided

\eoce{\qt{Sleep habits of New Yorkers\label{ny_sleep_habits_2_sided}}
New York is known as ``the city that never sleeps".
A random sample of 25 New Yorkers were asked how  much sleep they get per night.
Statistical summaries of these data are shown  below.
The point estimate suggests New Yorkers sleep less than 8~hours a night on average. 
Evaluate the claim that New York is the city that never sleeps keeping in mind that, despite this claim, the true average number of hours New Yorkers sleep could be less than 8 hours or more than 8 hours.
\begin{center}
\begin{tabular}{rrrrrr}
 \hline
n   & $\bar{x}$ & s     & min   & max \\ 
 \hline
25  & 7.73      & 0.77  & 6.17  & 9.78 \\ 
  \hline
\end{tabular}
\end{center}

\begin{parts}
\item Write the hypotheses in symbols and in words.
\item Check conditions, then calculate the test statistic, $T$, and the associated degrees of freedom.
\item Find and interpret the p-value in this context. Drawing a picture may be helpful.
\item What is the conclusion of the hypothesis test?
\item If you were to construct a 90\% confidence interval that corresponded to this hypothesis test, would you expect 8 hours to be in the interval?
\end{parts}
}{}

% 30 - adult_heights_ahss

\eoce{\qt{Heights of adults\label{adult_heights_ahss}}
Researchers studying anthropometry 
collected body girth measurements and skeletal diameter measurements, as well as 
age, weight, height and gender, for 507 physically active individuals. The mean height is 171.1 centimeters with a standard deviation of 9.4 centimeters.  The minimum height is 147.2 centimeters and the maximum height is 198.1 centimeters.
\begin{parts}
\item If we wanted to conduct a hypothesis test to determine whether there is evidence that the average height of physically active individuals is greater than 160 cm, what conditions would need to be met?  Do these conditions seem to be met here?
\item If we wanted to construct a confidence interval interval to estimate the mean height of physically active individuals, would the conditions be any different?  If so, which ones?
\end{parts}
}{}

% 31 - play_piano_ahss

\eoce{\qt{Play the piano\label{play_piano_ahss}} \videosolution{ahss_eoce_sol-play_piano} Georgianna claims that in a small city 
renowned for its music school, the average child takes less than 5 years of 
piano lessons. We have a random sample of 20 children from the city, with a 
mean of 4.6 years of piano lessons and a standard deviation of 2.2 years.
\begin{parts}
\item Evaluate Georgianna's claim using a hypothesis test.  Remember to Identify, Check, Calculate, and Conclude.
\item Use an appropriate procedure to estimate the average number of years students in 
this city take piano lessons with 95\% confidence.  Identify, Check, Calculate, and Conclude.
\item Do your results from the hypothesis test and the confidence interval 
agree? Explain your reasoning.
\end{parts}
}{}

% 32 - auto_exhaust_lead_exposure_2_sided

\eoce{\qt{Auto exhaust and
    lead exposure\label{auto_exhaust_lead_exposure_2_sided}} 
Researchers interested in lead exposure due to car exhaust
sampled the blood of 52 police officers subjected to constant
inhalation of automobile exhaust fumes while working traffic
enforcement in a primarily urban environment.
The blood samples of these officers had an average lead
concentration of 124.32 $\mu$g/l and a SD of 37.74 $\mu$g/l;
a previous study of individuals from a nearby suburb,
with no history of exposure, found an average blood level
concentration 
of 35 $\mu$g/l.\footfullcite{Mortada:2000}
\begin{parts}
\item
    Write down the hypotheses that would be appropriate for
    testing if the police officers appear to have been exposed
    to a different concentration of lead.
\item\label{auto_exhaust_lead_exposure_2_sided_cond}
    Explicitly state and check all conditions necessary for
    inference on these data.
\item
    Regardless of your answers in
    part~(\ref{auto_exhaust_lead_exposure_2_sided_cond}),
    test the hypothesis that the downtown police officers have
    a higher lead exposure than the group in the previous study.
    Interpret your results in context.
\end{parts}
}{}

% 33 - global_warming_ahss_HT

\eoce{\qt{Global warming, Part II\label{global_warming_ahss_HT}}
Consider a limited set of climate data,
examining temperature differences in 1948 vs~2018.
We sampled 197 locations from the
National Oceanic and Atmospheric Administration's
(NOAA) historical data,
where the data was available for both years of interest.
We want to know: were there more days with temperatures
exceeding 90\textdegree{}F in 2018 or
in~1948?\footfullcite{webpage:noaa_1948_2018}
The difference in number of days exceeding 90\textdegree{}F
(number of days in 2018 - number of days in 1948) was calculated
for each of the 197 locations.
The average of these differences was 2.9 days with 
a standard deviation of 17.2 days.
We are interested in determining whether these data provide
strong evidence that there were more days in 2018 that
exceeded 90\textdegree{}F from NOAA's weather
stations.\vspace{3mm}

\noindent%
\begin{minipage}[c]{0.65\textwidth}
\begin{parts}
\item
    Is there a relationship between the observations collected
    in 1948 and 2018?
    Or are the observations in the two groups independent?
    Explain.
\item
    Write hypotheses for this research in symbols and in words.
\item
    Check the conditions required to complete this test.
    A histogram of the differences is given to the right.
\item
    Calculate the test statistic, degrees of freedom and p-value.
\item
    Use $\alpha = 0.05$ to evaluate the test,
    and interpret your conclusion in context.
\item
    What type of error might we have made?
    Explain in context what the error means.
\item
    Based on the results of this hypothesis test,
    would you expect a confidence interval for the
    average difference between the number of days
    exceeding 90\textdegree{}F from 1948 and 2018
    to include 0?
    Explain your reasoning.
\end{parts}
\end{minipage}
\begin{minipage}[c]{0.02\textwidth}
\ 
\end{minipage}
\begin{minipage}[c]{0.32\textwidth}
\FigureFullPath[A histogram is shown for "Differences in Number of Days", which has bins between -70 and 60, where the bin width is 10. There is a prominent peak around zero, where much of the data lies between -40 and positive 40. The non-zero bins beyond this range are -70 to -60 has a bin height of 1, the 40 to 50 bin has a bin height of 2, and the 50 to 60 bin has a bin height of 1.]{}{ch_inference_for_means/figures/eoce/global_warming_ahss_HT/global_warming_v2_1_diffs}
\end{minipage}
% library(openintro); d <- climate70$dx90_2018 - climate70$dx90_1948; mean(d); sd(d); length(d); t.test(d)
}{}

% 34 - hs_beyond_ahss_HT

\eoce{\qt{High School and Beyond, Part II\label{hs_beyond_ahss_HT}} The National Center of 
Education Statistics conducted a survey of high school seniors, collecting test 
data on reading, writing, and several other subjects. Here we examine a simple 
random sample of 200 students from this survey. Side-by-side box plots of 
reading and writing scores as well as a histogram of the differences in scores 
are shown below.
\begin{center}
\FigureFullPath[A side-by-side box plot with dot plots also overlaid for each box plot. There are two categories shown, "read" and "write", for values ranging from about 27 to 77. The box portion of each distribution is nearly identical, ranging from about 45 to 60. The median of "read" is about 49 while the median of "write" is about 53. The whiskers for "read" extend down to about 27 and up to 77, while the whiskers for "write" extend down to about 32 and up to about 67. No points are shown beyond the whiskers for either box plot.]{0.44}{ch_inference_for_means/figures/eoce/hs_beyond_ahss_HT/hs_beyond_read_write_box}
\FigureFullPath[A histogram is shown for "Difference in scores (read minus write)", which is centered at approximately zero and is roughly bell-shaped with values ranging from -25 to positive 25.]{0.54}{ch_inference_for_means/figures/eoce/hs_beyond_ahss_HT/hs_beyond_diff_hist}
\end{center}
\begin{parts}
\item Is there a clear difference in the average reading and writing scores?
\item Are the reading and writing scores of each student independent of each 
other?
\item Create hypotheses appropriate for the following research question: is 
there an evident difference in the average scores of students in the reading 
and writing exam?
% is there evidence that students on average perform differently on the reading and writing exam?
\item Check the conditions required to complete this test.
\item The average observed difference in scores is 
$\bar{x}_{read-write} = -0.545$, and the standard deviation of the differences 
is 8.887 points. Do these data provide convincing evidence of a difference 
between the average scores on the two exams?
\item What type of error might we have made? Explain what the error means in 
the context of the application.
\item Based on the results of this hypothesis test, would you expect a 
confidence interval for the average difference between the reading and writing 
scores to include 0? Explain your reasoning.
\end{parts}
}{}
