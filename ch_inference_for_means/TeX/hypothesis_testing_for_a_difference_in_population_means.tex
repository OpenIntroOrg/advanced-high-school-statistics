\exercisesheader{}

% 41 - chick_wts_linseed_horsebean_ahss

\eoce{\qt{Chicken diet and weight,
    Part II\label{chick_wts_linseed_horsebean_ahss}} \videosolution{ahss_eoce_sol-chick_wts_linseed_horsebean} In Exercise~\ref{chick_wts_casein_soybean_ahss}, we learned about an experiment that was conducted to measure and compare
the effectiveness of various feed supplements on the
growth rate of chickens.
Newly hatched chicks were randomly allocated into six groups, 
and each group was given a different feed supplement.
Below are some summary statistics from this data set along
with box plots showing the distribution of weights by
feed type.\footfullcite{data:chickwts}

\noindent\begin{minipage}[c]{0.65\textwidth}
\begin{center}
\FigureFullPath[A side-by-side box plot is shown for "Weight, in grams" for several feed types. The width of the data range for each feed type spans about 150 grams. However, they are centered at different locations: about 325 for "casein", about 150 for "horsebean", about 225 for "linseed", about 275 for "meatmeal", about 250 for "soybean", and about 325 for "sunflower".]{}{ch_inference_for_means/figures/eoce/chick_wts_linseed_horsebean_ahss/chick_wts_box}
\end{center}
\end{minipage}
\begin{minipage}[c]{0.35\textwidth}
{\footnotesize\begin{tabular}{l c c c}
\hline
            & Mean      & SD        & n \\
\hline
casein          & 323.58        & 64.43 & 12 \\
horsebean   & 160.20        & 38.63 & 10 \\
linseed         & 218.75        & 52.24 & 12 \\
meatmeal    & 276.91        & 64.90 & 11 \\
soybean         & 246.43        & 54.13 & 14 \\
sunflower       & 328.92        & 48.84 & 12 \\
\hline
\end{tabular}}
\end{minipage} 

\begin{parts}
\item Describe the distributions of weights of chickens that were fed linseed 
and horsebean.
\item Do these data provide strong evidence that the average weights of 
chickens that would be fed linseed and horsebean are different? Use a 5\% 
significance level.
\item What type of error might we have committed? Explain.
\item Would your conclusion change if we used $\alpha = 0.01$?
\end{parts}
}{}

% 42 - fuel_eff_city_ahss

\eoce{\qt{Fuel efficiency of manual and automatic cars, City\label{fuel_eff_city_ahss}} 
The table provides summary statistics on highway fuel economy
of the same 52 cars from Exercise~\ref{fuel_eff_hway_ahss}.
Do these data provide strong evidence of a difference between the
average fuel efficiency of cars with manual and automatic
transmissions in terms of their average city mileage?  
\footfullcite{data:epaMPG}\\
\noindent\begin{minipage}[c]{0.38\textwidth}
\begin{center}
\begin{tabular}{l c c }
\hline
        & \multicolumn{2}{c}{City MPG} \\
\hline
        & Automatic     & Manual         \\
Mean    & 16.12         & 19.85      \\
SD      & 3.58          & 4.51       \\
n       & 26            & 26 \\
\hline
& \\
& \\
\end{tabular}
\end{center}
\end{minipage}
\begin{minipage}[c]{0.6\textwidth}
\begin{center}
\FigureFullPath[A side-by-side box plot is shown for "City MPG" for "automatic" and "manual" cars. The "automatic" box plot has its box spanning approximately 14 to 19, has a median of about 16, and its whiskers extending down to about 7 and up to about 24. The "manual" box plot has its box spanning approximately 18 to 24, has a median of about 21, and its whiskers extending down to about 8 and up to about 31.]{0.7}{ch_inference_for_means/figures/eoce/fuel_eff_city_ahss/fuel_eff_city_box}
\end{center}
\end{minipage}
}{}

% 43 - gaming_distracted_eating_intake

\eoce{\qt{Gaming and distracted eating, Part I\label{gaming_distracted_eating_intake}}
A group of researchers are interested in the possible effects of distracting 
stimuli during eating, such as an increase or decrease in the amount of food 
consumption. To test this hypothesis, they monitored food intake for a group of 
44 patients who were randomized into two equal groups. The treatment group ate 
lunch while playing solitaire, and the control group ate lunch without any 
added distractions. Patients in the treatment group ate 52.1 grams of biscuits, 
with a standard deviation of 45.1 grams, and patients in the control group ate 
27.1 grams of biscuits, with a standard deviation of 26.4 grams. Do these data 
provide convincing evidence that the average food intake (measured in amount of 
biscuits consumed) is different for the patients in the treatment group? Assume 
that conditions for inference are satisfied. \footfullcite{Oldham:2011}
}{}

\D{\newpage}

% 44 - gaming_distracted_eating_recall

\eoce{\qt{Gaming and distracted eating, Part II\label{gaming_distracted_eating_recall}} 
The researchers from Exercise~\ref{gaming_distracted_eating_intake} also 
investigated the effects of being distracted by a game on how much people eat. 
The 22 patients in the treatment group who ate their lunch while playing 
solitaire were asked to do a serial-order recall of the food lunch items they 
ate. The average number of items recalled by the patients in this group was 4.
9, with a standard deviation of 1.8. The average number of items recalled by 
the patients in the control group (no distraction) was 6.1, with a standard 
deviation of 1.8. Do these data provide strong evidence that the average number 
of food items recalled by the patients in the treatment and control groups are 
different?
}{}
