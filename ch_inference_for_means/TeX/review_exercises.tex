\reviewexercisesheader{}

% 45 - hen_eggs_ahss

\eoce{\qt{Hen eggs\label{hen_eggs_ahss}} The distribution of the number of eggs laid 
by a certain species of hen during their breeding period has a mean of 35 eggs 
with a standard deviation of 18.2. Suppose a group of researchers 
randomly samples 45 hens of this species, counts the number of eggs laid 
during their breeding period, and records the sample mean. They repeat 
this 1,000 times, and build a distribution of sample 
means. 
\begin{parts}
\item What is this distribution called? 
\item Would you expect the shape of this distribution to be symmetric, right 
skewed, or left skewed? Explain your reasoning.
\item Calculate the variability of this distribution and state the appropriate 
term used to refer to this value.
\item Suppose the researchers' budget is reduced and they are only able to 
collect random samples of 10 hens. The sample mean of the number of eggs is 
recorded, and we repeat this 1,000 times, and build a new distribution of sample 
means. How will the variability of this new distribution compare to the 
variability of the original distribution?
\end{parts}
}{}

% 46 - college_credits_ahss

\eoce{\qt{College credits\label{college_credits_ahss}}
A college counselor is interested in 
estimating how many credits a student typically enrolls
in each semester.
The counselor decides to randomly sample 100 students
by using the registrar's 
database of students.
The histogram below shows the distribution of the number 
of credits taken by these students.
Sample statistics for this distribution are 
also provided.\\
\begin{minipage}[c]{0.1\textwidth}
\ 
\end{minipage}
\begin{minipage}[c]{0.5\textwidth}
\begin{center}
\FigureFullPath[A histogram is shown for "Number of credits". The distribution is centered at about 13 and is very roughly bell-shaped with data ranging from 8 to 18 with no apparent outliers.]{}{ch_inference_for_means/figures/eoce/college_credits_ahss/college_credits_hist}
\end{center}
\end{minipage}
\begin{minipage}[c]{0.32\textwidth}
\begin{center}
\begin{tabular}{l|r l}
Min     & 8 \\
Q1      & 13 \\
Median  & 14 \\
Mean    & 13.65 \\
SD      & 1.91 \\
Q3      & 15 \\
Max     & 18 \\
\end{tabular}
\end{center}
\end{minipage}
\begin{parts}
\item What is the point estimate for the average
number of credits taken per semester by students at this college?
What about the median?
\item What is the point estimate for the standard deviation
of the number of credits taken per semester by students at
this college?
What about the IQR?
\item Is a load of 16 credits unusually high for this college?
What about 18 credits?
Explain your reasoning.
\item The college counselor takes another
random sample of 100 students and this 
time finds a sample mean of 14.02 units.
Should she be surprised that this sample
statistic is slightly different than the
one from the original sample? 
Explain your reasoning.
\item
The sample means given above are point estimates
for the mean number of 
credits taken by all students at that college.
What measure do we use to 
quantify the variability of this estimate?
Compute this quantity using the data 
from the original sample.
\end{parts}
}{}

% 47 - air_quality_shortened

\eoce{\qt{Air quality\label{air_quality_shortened}}
Air quality measurements were collected in 
a random sample of 25 country capitals in 2013, and then again in the same 
cities in 2014. We would like to use these data to compare
average air quality between the two years.
Should we use a paired or non-paired test? Explain your reasoning.
}{}

% 48 - tf_paired

\eoce{\qt{True / False: paired\label{tf_paired}} Determine if the following 
statements are true or false. If false, explain.
\begin{parts}
\item In a paired analysis we first take the difference of each pair of observations, 
and then we do inference on these differences.
\item Two data sets of different sizes cannot be analyzed as paired data.
\item Consider two sets of data that are paired with each other.
Each observation in one data set has a natural correspondence with 
exactly one observation from the other data set.
\item Consider two sets of data that are paired with each other.
Each observation in one data set is subtracted from the average of the 
other data set's observations.
\end{parts}
}{}

\D{\newpage}

% 49 - find_mean

\eoce{\qt{Find the mean\label{find_mean}} You are given the following hypotheses:
\begin{align*}
H_0&: \mu = 60 \\
H_A&: \mu < 60
\end{align*}
We know that the sample standard deviation is 8 and the sample size is 20. For 
what sample mean would the p-value be equal to 0.05? Assume that all conditions 
necessary for inference are satisfied.
}{}

% 50 - diamonds_1_ahss

\eoce{\qt{Diamonds, Part I\label{diamonds_1_ahss}} Prices of diamonds are determined by 
what is known as the 4 Cs: cut, clarity, color, and carat weight. The prices of 
diamonds go up as the carat weight increases, but the increase is not smooth. 
For example, the difference between the size of a 0.99 carat diamond and a 1 
carat diamond is undetectable to the naked human eye, but the price of a 1 
carat diamond tends to be much higher than the price of a 0.99 diamond. In this 
question we use two random samples of diamonds, 0.99 carats and 1 carat, each 
sample of size 23, and compare the average prices of the diamonds. In order to 
be able to compare equivalent units, we first divide the price for each diamond 
by 100 times its weight in carats. That is, for a 0.99 carat diamond, we divide 
the price by 99. For a 1 carat diamond, we divide the price by 100. The 
distributions and some sample statistics are shown below.\footfullcite{ggplot2} \\[1mm]
\begin{minipage}[c]{0.6\textwidth}
Conduct a hypothesis test to evaluate if there is a difference between the 
average standardized prices of 0.99 and 1 carat diamonds.  Remember to Identify, Check, Calculate, and Conclude. \\[2mm]
\begin{tabular}{l c c }
\hline
        & 0.99 carats       & 1 carat\\
\hline  
Mean    & \$44.51          & \$56.81           \\
SD      & \$13.32          &\$16.13            \\
n       &23             & 23 \\
\hline
\end{tabular}
\end{minipage}%
\begin{minipage}[c]{0.4\textwidth}
\begin{center}
\includegraphics[width=0.85\textwidth]{ch_inference_for_means/figures/eoce/diamonds_1_ahss/diamonds_box.pdf}
\end{center}
\end{minipage}
}{}

% 51 - friday_13th_traffic

\eoce{\qt{Friday the 13$^{\text{th}}$, Part I\label{friday_13th_traffic}} In the 
early 1990's, researchers in the UK collected data on traffic flow, number of 
shoppers, and traffic accident related emergency room admissions on Friday the 
13$^{\text{th}}$ and the previous Friday, Friday the 6$^{\text{th}}$. The 
histograms below show the distribution of number of cars passing by a specific 
intersection on Friday the 6$^{\text{th}}$ and Friday the 13$^{\text{th}}$ for 
many such date pairs. Also given are some sample statistics, where the 
difference is the number of cars on the 6th minus the number of cars on the 13th.\footfullcite{Scanlon:1993}
\begin{center}
\FigureFullPath[Three histograms are shown. The first histogram is for "Friday the 6th", which has values ranging from 110,000 to 140,000. The second histogram is for "Friday the 13th", which also has values ranging from 110,000 to 140,000. The third histogram is for "Difference", with values ranging from 0 to 5,000. While the first two distributions are relatively uniform across the range, the last distribution has most of its distribution ranging between 0 and 3,000, with one observation in the 4,000 to 5,000 bin, which represents one value.]{}{ch_inference_for_means/figures/eoce/friday_13th_traffic/friday_13th_traffic_hist} \\
$\:$ \\
{\small
\begin{tabular}{l c c c}
\hline
        & 6$^{\text{th}}$   & 13$^{\text{th}}$  & Diff.\\
\hline  
$\bar{x}$   &128,385            & 126,550       & 1,835 \\
$s$     &7,259          & 7,664         & 1,176 \\
$n$     &10             & 10                & 10 \\
\hline
\end{tabular}
}
\end{center}
\begin{parts}
\item Are there any underlying structures in these data that should be 
considered in an analysis? Explain.
\item What are the hypotheses for evaluating whether the number of people out 
on Friday the 6$^{\text{th}}$ is different than the number out on Friday the 
13$^{\text{th}}$?
\item Check conditions to carry out the hypothesis test from part~(b).
\item Calculate the test statistic and the p-value.
\item What is the conclusion of the hypothesis test?
\item Interpret the p-value in this context.
\item What type of error might have been made in the conclusion of your test? 
Explain.
\end{parts}
}{}

\D{\newpage}

% 52 - age_at_first_marriage_intro

\eoce{\qt{Age at first marriage, Part I\label{age_at_first_marriage_intro}} 
The National Survey of Family Growth conducted by the Centers for Disease 
Control gathers information on family life, marriage and divorce, pregnancy, 
infertility, use of contraception, and men's and women's health. One of the 
variables collected on this survey is the age at first marriage. The histogram 
below shows the distribution of ages at first marriage of 5,534 randomly sampled 
women between 2006 and 2010. The average age at first marriage among these women 
is 23.44 with a standard deviation of 4.72.\footfullcite{data:nsfg:2010}
\begin{center}
\FigureFullPath[A histogram is shown for "Age at first marriage". The distribution is right-skewed, centered at about 23, has a standard deviation of about 5. The data smoothly tapers off in each direction but do not extend below about 12 or above 45.]{0.6}{ch_inference_for_means/figures/eoce/age_at_first_marriage_intro/age_at_first_marriage_intro_hist}
\end{center}
Estimate the average age at first marriage of women using a 95\% confidence 
interval, and interpret this interval in context. Discuss any relevant 
assumptions.
}{}

% 53 - friday_13th_accident

\eoce{\qt{Friday the 13$^{\text{th}}$, Part II\label{friday_13th_accident}} \videosolution{friday_13_accident_solution} 
The Friday the $13^{th}$ study reported in
Exercise~\ref{friday_13th_traffic} also provides data on traffic
accident related emergency room admissions.
The distributions of these counts from Friday the 6$^{\text{th}}$ and
Friday the 13$^{\text{th}}$ are shown below for six such paired dates
along with summary statistics.
You may assume that conditions for inference are met.
\begin{center}
\FigureFullPath[Three histograms are shown. The first histogram is for "Friday the 6th", which has values ranging across 3 to 12. The second histogram is for "Friday the 13th", which has values ranging from 4 to 14. The third histogram is for "Difference", with values ranging from -8 to positive 2.]{0.72}{ch_inference_for_means/figures/eoce/friday_13th_accident/friday_13th_accident_hist} \\
$\:$ \\
\begin{minipage}[c]{0.3\textwidth}
\includegraphics[width=\textwidth]{ch_inference_for_means/figures/eoce/friday_13th_accident/friday_13th_accident_hist_diff}
\end{minipage}
\begin{minipage}[c]{0.36\textwidth}
\ \ \begin{tabular}{l c c c}
\hline
        & 6$^{\text{th}}$   & 13$^{\text{th}}$  & diff\\
\hline  
Mean    &7.5                & 10.83             & -3.33 \\
SD      &3.33           & 3.6               & 3.01 \\
n       &6              & 6             & 6 \\
\hline
\end{tabular}
\end{minipage}
\end{center}

\begin{parts}
\item Conduct a hypothesis test to evaluate if there is a difference between 
the average numbers of traffic accident related emergency room admissions 
between Friday the 6$^{\text{th}}$ and Friday the~13$^{\text{th}}$.
\item Calculate a 95\% confidence interval for the difference between the 
average numbers of traffic accident related emergency room admissions between 
Friday the 6$^{\text{th}}$ and Friday the 13$^{\text{th}}$.
\item The conclusion of the original study states, ``Friday 13th is unlucky for 
some. The risk of hospital admission as a result of a transport accident may be 
increased by as much as 52\%. Staying at home is recommended.'' Do you agree 
with this statement? Explain your reasoning.
\end{parts}
}{}

\D{\newpage}

% 54 - diamonds_2_ahss

\eoce{\qt{Diamonds, Part II\label{diamonds_2_ahss}} In Exercise~\ref{diamonds_1_ahss}, we 
discussed diamond prices (standardized by weight) for diamonds with weights 0.
99 carats and 1 carat. See the table for summary statistics, and then use 
a 95\% confidence interval procedure to estimate the difference in means between the standardized 
prices of 0.99 and 1 carat diamonds. Remember to Identify, Check, Calculate, and Conclude.
\begin{center}
\begin{tabular}{l c c }
\hline
        & 0.99 carats       & 1 carat\\
\hline  
Mean    & \$44.51          & \$56.81           \\
SD      & \$13.32          &\$16.13            \\
n       &23             & 23 \\
\hline
\end{tabular}
\end{center}
}{}
