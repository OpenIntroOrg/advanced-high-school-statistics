\exercisesheader{}

% 13 - identify_critical_t

\eoce{\qt{Identify the critical $\pmb{t}$\label{identify_critical_t}} \videosolution{ahss_eoce_sol-identify_critical_t} A random 
sample is selected from an approximately normal population with unknown 
standard deviation. Find the degrees of freedom and the critical $t$-value 
(t$^\star$) for the given sample size and confidence level.
%\begin{multicols}{4}
\begin{parts}
\item $n = 6$, CL = 90\%
\item $n = 21$, CL = 98\%
\item $n = 29$, CL = 95\%
\item $n = 12$, CL = 99\%
\end{parts}
%\end{multicols}
}{}

% 14 - t_distribution

\eoce{\qt{$\pmb{t}$-distribution\label{t_distribution}}
The figure on the right shows three 
unimodal and symmetric curves:
the standard normal (z) distribution,
the $t$-distribution with 5 degrees of freedom,
and the $t$-distribution with 1 degree of freedom.
Determine which is which, and explain your reasoning.
\begin{center}
\FigureFullPath[Three distributions are shown, all symmetric, bell-shaped, and centered at zero. The first is shown as a solid line and has the broadest peak of the three distributions, and the tails of this distribution also visually approach zero at about -3 and positive 3. The second curve that is shown as a dashed line has a less broad, slightly sharper peak than the distribution based on solid line. The tails of the distribution with the dashed line has tails that visually approach zero at values of about -4 and positive 4. The third curve is shown as a dotted line and has the sharpest peak of the three distributions. The tails of the dotted line distribution has tails that visually approach zero further out, beyond the limits shown in this plot of -4 and positive 4.]{0.4}{ch_inference_for_means/figures/eoce/t_distribution/t_distribution}
\end{center}
}{}

% 15 - work_backwards_1

\eoce{\qt{Working backwards, Part I\label{work_backwards_1}} A 95\% confidence 
interval for a population mean, $\mu$, is given as (18.985, 21.015). This 
confidence interval is based on a simple random sample of 36 observations. 
Calculate the sample mean and standard deviation. Assume that all conditions 
necessary for inference are satisfied. Use the $t$-distribution in any 
calculations.
}{}

% 16 - work_backwards_2

\eoce{\qt{Working backwards, Part II\label{work_backwards_2}} A 90\% confidence 
interval for a population mean is (65, 77). The population distribution is 
approximately normal and the population standard deviation is unknown. This 
confidence interval is based on a simple random sample of 25 observations. 
Calculate the sample mean, the margin of error, and the sample standard 
deviation.
}{}

% 17 - exclusive_relationships_ahss

\eoce{\qt{Exclusive relationships\label{exclusive_relationships_ahss}} A survey conducted 
on a reasonably random sample of 203 undergraduates asked, among many other 
questions, about the number of exclusive relationships these students have been 
in. The histogram below shows the distribution of the data from this sample. 
The sample average is 3.2 with a standard deviation of 1.97.
\begin{center}
\FigureFullPath[A histogram is shown for "Number of exclusive relationships". The bin at 1 is about 25\% of the data, the bin at 2 is about 25\%, bin 3 is about 1\% of the data, bin 4 about 25\% of the data, bin 5 about 12\%, bin 6 about 5\%, bin 7 about 3\%, and further tapering off up through bin 10.]{0.6}{ch_inference_for_means/figures/eoce/exclusive_relationships/exclusive_relationships_rel_hist}
\end{center}
Estimate the average number of exclusive relationships Duke students have been 
in using a 90\% confidence interval and interpret this interval in context. 
Check any conditions required for inference, and note any assumptions you must 
make as you proceed with your calculations and conclusions.
}{}

% 18 - forest_mgmt_tree_growth

\eoce{\qt{Forest management\label{forest_mgmt_tree_growth}}
Forest rangers wanted to better understand the rate
of growth for younger trees in the park.
They took measurements of a random sample of 50 young trees
in 2009 and again measured those same trees in 2019.
The data below summarize their measurements,
where the heights are in feet:
\begin{center}
\begin{tabular}{l c c c}
\hline
          & 2009   & 2019  & Differences\\
\hline  
$\bar{x}$ & 12.0  & 24.5  & 12.5 \\
$s$       & 3.5   & 9.5   & 7.2 \\
$n$       & 50    & 50    & 50 \\
\hline
\end{tabular}
\end{center}
Construct a 99\% confidence interval for the
average growth of (what had been) younger trees
in the park over 2009-2019.
}{}

% 19 - paired_or_not_1

\eoce{\qt{Paired or not? Part I\label{paired_or_not_1}} In each of the following 
scenarios, determine if the data are paired.
\begin{parts}
\item Compare pre- (beginning of semester) and post-test (end of semester) 
scores of students.
\item Assess gender-related salary gap by comparing salaries of randomly 
sampled men and women.
\item Compare artery thicknesses at the beginning of a study and after 2 years 
of taking Vitamin E for the same group of patients.
\item Assess effectiveness of a diet regimen by comparing the before and after 
weights of subjects.
\end{parts}
}{}

% 20 - paired_or_not_2

\eoce{\qt{Paired or not? Part II\label{paired_or_not_2}} In each of the following 
scenarios, determine if the data are paired.
\begin{parts}
\item We would like to know if Intel's stock and Southwest Airlines' stock have 
similar rates of return. To find out, we take a random sample of 50 days, and 
record Intel's and Southwest's stock on those same days.
\item We randomly sample 50 items from Target stores and note the price for 
each. Then we visit Walmart and collect the price for each of those same 50 
items.
\item A school board would like to determine whether there is a difference in 
average SAT scores for students at one high school versus another high school 
in the district. To check, they take a simple random sample of 100 students 
from each high school.
\end{parts}
}{}

% 21 - sample_size_pairing

\eoce{\qt{Sample size and pairing\label{sample_size_pairing}} Determine if the 
following statement is true or false, and if false, explain your reasoning: If 
comparing means of two groups with equal sample sizes, always use a paired test.
}{}

% 22 - critical_t_vs_z

\eoce{\qt{$\pmb{t^\star}$ vs. $\pmb{z^\star}$\label{critical_t_vs_z}} For a given confidence 
level, $t^{\star}_{df}$ is larger than $z^{\star}$. Explain how $t^{*}_{df}$ 
being slightly larger than $z^{*}$ affects the width of the confidence interval.
}{}

% 23 - global_warming_ahss_CI

\eoce{\qt{Global warming, Part I\label{global_warming_ahss_CI}}
Consider a limited set of climate data,
examining temperature differences in 1948 vs~2018.
We sampled 197 locations from the
National Oceanic and Atmospheric Administration's
(NOAA) historical data,
where the data was available for both years of interest.
We want to know: were there more days with temperatures
exceeding 90\textdegree{}F in 2018 or
in~1948?\footfullcite{webpage:noaa_1948_2018}
The difference in number of days exceeding 90\textdegree{}F
(number of days in 2018 - number of days in 1948) was calculated
for each of the 197 locations.
The average of these differences was 2.9 days with 
a standard deviation of 17.2 days.
We are interested in determining whether these data provide
strong evidence that there were more days in 2018 that
exceeded 90\textdegree{}F from NOAA's weather
stations.\vspace{3mm}

\noindent%
\begin{minipage}[c]{0.65\textwidth}
\begin{parts}
\item Discuss whether the conditions are met for a one-sample $t$-interval for a mean.
\item
    Calculate a 90\% confidence interval for the average
    difference between number of days exceeding 90\textdegree{}F
    between 1948 and 2018.
    We've already checked the conditions for you.
\item
    Interpret the interval in context.
\item
    Does the confidence interval provide convincing evidence
    that there were more days exceeding 90\textdegree{}F
    in 2018 than in 1948 at NOAA stations?
    Explain.
\end{parts}
\end{minipage}
\begin{minipage}[c]{0.02\textwidth}
\ 
\end{minipage}
\begin{minipage}[c]{0.32\textwidth}
\FigureFullPath[A histogram is shown for "Differences in Number of Days", which has bins between -70 and 60, where the bin width is 10. There is a prominent peak around zero, where much of the data lies between -40 and positive 40. The non-zero bins beyond this range are -70 to -60 has a bin height of 1, the 40 to 50 bin has a bin height of 2, and the 50 to 60 bin has a bin height of 1.]{}{ch_inference_for_means/figures/eoce/global_warming_ahss_CI/global_warming_v2_1_diffs}
\end{minipage}
}{}

% 24 - hs_beyond_ahss_CI

\eoce{\qt{High School and Beyond, Part I\label{hs_beyond_ahss_CI}} The National Center of 
Education Statistics conducted a survey of high school seniors, collecting test 
data on reading, writing, and several other subjects. Here we examine a simple 
random sample of 200 students from this survey. Side-by-side box plots of 
reading and writing scores as well as a histogram of the differences in scores for those students are shown below.  The 
mean and standard deviation of the differences are 
$\bar{x}_{read-write} = -0.545$ and $s_{read-write} = 8.887$ points.
\begin{center}
\FigureFullPath[A side-by-side box plot with dot plots also overlaid for each box plot. There are two categories shown, "read" and "write", for values ranging from about 27 to 77. The box portion of each distribution is nearly identical, ranging from about 45 to 60. The median of "read" is about 49 while the median of "write" is about 53. The whiskers for "read" extend down to about 27 and up to 77, while the whiskers for "write" extend down to about 32 and up to about 67. No points are shown beyond the whiskers for either box plot.]{0.44}{ch_inference_for_means/figures/eoce/hs_beyond_ahss_CI/hs_beyond_read_write_box}
\FigureFullPath[A histogram is shown for "Difference in scores (read minus write)", which is centered at approximately zero and is roughly bell-shaped with values ranging from -25 to positive 25.]{0.54}{ch_inference_for_means/figures/eoce/hs_beyond_ahss_CI/hs_beyond_diff_hist}
\end{center}
\begin{parts}
\item Discuss whether conditions are met for a one-sample $t$-interval for a mean.
\item Calculate a 95\% confidence interval for the average difference between 
the reading and writing scores of all students.
\item Interpret this interval in context.
\item Does the confidence interval provide convincing evidence that there is a 
real difference in the average scores? Explain.
\end{parts}
}{}
