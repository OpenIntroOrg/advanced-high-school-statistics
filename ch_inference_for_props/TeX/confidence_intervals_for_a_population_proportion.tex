\exercisesheader{}

% 15 - chronic_illness_intro

\eoce{\qt{Chronic illness, Part I\label{chronic_illness_intro}} 
In 2013, the Pew Research Foundation reported that ``45\% of U.S. adults report 
that they live with one or more chronic conditions''.
\footfullcite{data:pewdiagnosis:2013} However, this value was based on a sample, 
so it may not be a perfect estimate for the population parameter of interest on 
its own. The study reported a standard error of about 1.2\%, and a normal model 
may reasonably be used in this setting. Create a 95\% confidence interval for 
the proportion of U.S. adults who live with one or more chronic conditions. Also 
interpret the confidence interval in the context of the study.
}{}

% 16 - twitter_users_intro

\eoce{\qt{Twitter users and news, Part I\label{twitter_users_intro}} 
A poll conducted in 2013 found that 52\% of U.S. adult Twitter users 
get at least some news on Twitter.\footfullcite{data:pewtwitternews:2013}. 
The standard error for this estimate was 2.4\%, and a normal distribution 
may be used to model the sample proportion. Construct a 99\% confidence 
interval for the fraction of U.S. adult Twitter users who get some 
news on Twitter, and interpret the confidence interval in context.
}{}

% 17 - er_wait_intro_prop_ok

\eoce{\qt{Waiting at an ER, Part I\label{er_wait_intro_prop_ok}} A hospital administrator 
hoping to improve wait times decides to estimate the average emergency 
room waiting time at her hospital. She collects a simple random sample 
of 64 patients and determines the time (in minutes) between when they 
checked in to the ER until they were first seen by a doctor. A 95\% 
confidence interval based on this sample is (128 minutes, 147 minutes), 
which is based on the normal model for the mean. Determine whether the 
following statements are true or false, and explain your reasoning.
\begin{parts}
\item We are 95\% confident that the average waiting time of these 64 emergency 
room patients is between 128 and 147 minutes.
\item We are 95\% confident that the average waiting time of all patients at 
this hospital's emergency room is between 128 and 147 minutes.
\item 95\% of random samples have a sample mean between 128 and 147 minutes.
\item A 99\% confidence interval would be narrower than the 95\% confidence 
interval since we need to be more sure of our estimate.
\item The margin of error is 9.5 and the sample mean is 137.5.
\item In order to decrease the margin of error of a 95\% confidence interval to 
half of what it is now, we would need to double the sample size.
\end{parts}
}{}

% 18 - mental_health

\eoce{\qt{Mental health\label{mental_health}}
The General Social Survey asked the question:
``For how many days during the past 30 days was your 
mental health, which includes stress, depression,
and problems with emotions, not good?"
Based on responses from 1,151 US residents,
the survey reported a 95\% confidence interval of
3.40 to 4.24 days in 2010.
\begin{parts}
\item
    Interpret this interval in context of the data.
\item
    What does ``95\% confident" mean? Explain in the
    context of the application.
\item
    Suppose the researchers think a 99\% confidence level
    would be more appropriate for this interval.
    Will this new interval be smaller or wider than the
    95\% confidence interval?
\item
    If a new survey were to be done with 500 Americans,
    do you think the standard error of the estimate be
    larger, smaller, or about the same.
\end{parts}
}{}

% 19 - cyberbullying_prop_ci_ht

\eoce{\qt{Cyberbullying rates\label{cyberbullying_prop_ci_ht}}
Teens were surveyed about cyberbullying, and
54\% to 64\% reported experiencing cyberbullying
(95\% confidence interval).\footfullcite{pew_cyber_bully_2018}
Answer the following questions based on this interval.
\begin{parts}
\item 
    A newspaper claims that a majority of teens
    have experienced cyberbullying.
    Is this claim supported by the confidence interval?
    Explain your reasoning.
\item\label{cyberbullying_prop_ci_ht_researcher}
    A researcher conjectured that 70\% of teens have
    experienced cyberbullying.
    Is this claim supported by the confidence interval?
    Explain your reasoning.
\item
    Without actually calculating the interval, determine
    if the claim of the researcher from
    part~(\ref{cyberbullying_prop_ci_ht_researcher})
    would be supported based on a 90\% confidence interval?
\end{parts}
}{}

% 20 - er_wait_ci_ht_prop_ok

\eoce{\qt{Waiting at an ER, Part II\label{er_wait_ci_ht_prop_ok}}
Exercise~\ref{er_wait_intro_prop_ok} 
provides a 95\% confidence interval for the mean waiting
time at an emergency room (ER) of (128 minutes, 147 minutes).
Answer the following questions based on this interval.
\begin{parts}
\item
    A local newspaper claims that the average waiting time
    at this ER exceeds 3 hours.
    Is this claim supported by the confidence interval?
    Explain your reasoning.
\item\label{er_wait_ci_ht_prop_ok_dean}
    The Dean of Medicine at this hospital claims the
    average wait time is 2.2 hours.
    Is this claim supported by the confidence interval?
    Explain your reasoning.
\item
    Without actually calculating the interval,
    determine if the claim of the Dean from
    part~(\ref{er_wait_ci_ht_prop_ok_dean})
    would be supported based on a 99\% confidence interval?
\end{parts}
}{}

% 21 - orange_tabbies_CLT

\eoce{\qt{Orange tabbies\label{orange_tabbies_CLT}} Suppose that 90\% of orange
tabby cats are male. Determine if the following statements are true or false, 
and explain your reasoning.
\begin{parts}
\item The distribution of sample proportions of random samples of size 30 is 
left skewed.
\item Using a sample size that is 4 times as large will reduce the standard 
error of the sample proportion by one-half.
\item The distribution of sample proportions of random samples of size 140 is 
approximately normal.
\item The distribution of sample proportions of random samples of size 280 is 
approximately normal.
\end{parts}
}{}

% 22 - young_americans_CLT_2

\eoce{\qt{Young Americans, Part II\label{young_americans_CLT_2}} About 25\% of 
young Americans have delayed starting a family due to the continued economic 
slump. Determine if the following statements are true or false, and explain 
your reasoning.\footfullcite{news:youngAmericans2}
\begin{parts}
\item The distribution of sample proportions of young Americans who have 
delayed starting a family due to the continued economic slump in random 
samples of size 12 is right skewed.
\item In order for the distribution of sample proportions of young Americans 
who have delayed starting a family due to the continued economic slump to be 
approximately normal, we need random samples where the sample size is at 
least 40.
\item A random sample of 50 young Americans where 20\% have delayed starting 
a family due to the continued economic slump would be considered unusual.
\item A random sample of 150 young Americans where 20\% have delayed 
starting a family due to the continued economic slump would be considered 
unusual.
\item Tripling the sample size will reduce the standard error of the sample 
proportion by one-third.
\end{parts}
}{}

% 23 - gender_equality

\eoce{\qt{Gender equality\label{gender_equality}}
The General Social Survey asked a random sample of
1,390 Americans the following question:
``On the whole, do you think it should or should not be
the government's responsibility to  promote equality
between men and women?''
82\% of the respondents said it ``should be''.
At a 95\% confidence level, this sample has 2\% margin of error.
Based on this information, determine if the following statements
are true or false, and explain your reasoning.\footfullcite{data:gss}
\begin{parts}
\item We are 95\% confident that between 80\% and 84\% of Americans in this 
sample think it's the government's responsibility to promote equality between 
men and women.
\item We are 95\% confident that between 80\% and 84\% of all Americans 
think it's the government's responsibility to promote equality between 
men and women.
\item If we considered many random samples of 1,390 Americans, and we calculated 
95\% confidence intervals for each, 95\% of these intervals would include the 
true population proportion of Americans who think it's the government's 
responsibility to promote equality between men and women.
\item In order to decrease the margin of error to 1\%, we would need to 
quadruple (multiply by 4) the sample size.
\item Based on this confidence interval, there is sufficient evidence to 
conclude that a majority of Americans think it's the government's responsibility 
to promote equality between men and women.
\end{parts}

% n = 1390
% should be: 1142
% p = 1142/1390 = 0.82
% me = sqrt(.82*.08/1390)*1.96 = 0.02
}{}

% 24 - elderly_drivers_CI_concept

\eoce{\qt{Elderly drivers\label{elderly_drivers_CI_concept}} 
The Marist Poll published a report stating that 66\% of adults nationally 
think licensed drivers should be required to retake their road test once 
they reach 65 years of age. It was also reported that interviews were 
conducted on a random sample of 1,018 American adults, and that the margin of error was 3\% 
using a 95\% confidence level. \footfullcite{data:elderlyDriving}
\begin{parts}
\item Verify the margin of error reported by The Marist Poll. 
\item Based on a 95\% confidence interval, does the poll provide convincing 
evidence that \textit{more than} two thirds of the population think that licensed 
drivers should be required to retake their road test once they turn 65?
\end{parts}
}{}

% 25 - fireworks_CI_concept

\eoce{\qt{Fireworks on July 4$^{\text{th}}$\label{fireworks_CI_concept}} A local 
news outlet reported that 56\% of 600 randomly sampled Kansas residents planned 
to set off fireworks on July~$4^{th}$. Determine the margin of error for the 
56\% point estimate using a 95\% confidence level.\footfullcite{data:july4}
}{}

% 26 - greece_life_rating_CI

\eoce{\qt{Life rating in Greece\label{greece_life_rating_CI}} Greece has faced a 
severe economic crisis since the end of 2009. A Gallup poll surveyed 1,000 
randomly sampled Greeks in 2011 and found that 25\% of them said they would 
rate their lives poorly enough to be considered ``suffering''.\footfullcite{data:suffering}
\begin{parts}
\item Describe the population parameter of interest. What is the value of the 
point estimate of this parameter?
\item Check if the conditions required for constructing a confidence interval 
based on these data are met.
\item Construct a 95\% confidence interval for the proportion of Greeks who 
are ``suffering".
\item Without doing any calculations, describe what would happen to the 
confidence interval if we decided to use a higher confidence level.
\item Without doing any calculations, describe what would happen to the 
confidence interval if we used a larger sample.
\end{parts}
}{}

% 27 - study_abroad_CI_decision

\eoce{\qt{Study abroad\label{study_abroad_CI_decision}}
A survey on 1,509 high school seniors who took the SAT
and who completed an optional web survey shows that
55\% of high school seniors are fairly certain that
they will participate in a study abroad program in 
college.\footfullcite{data:studyAbroad}
\begin{parts}
\item
    Is this sample a representative sample from the population
    of all high school seniors in the US?
    Explain your reasoning.
\item
    Let's suppose the conditions for inference are met.
    Even if your answer to part (a) indicated that this approach
    would not be reliable, this analysis may still be interesting
    to carry out (though not report).
    Construct a 90\% confidence interval for the proportion of high
    school seniors (of those who took the SAT) who are fairly certain
    they will participate in a study abroad program in college,
    and interpret this interval in context.
\item
    What does ``90\% confidence" mean?
\item
    Based on this interval, would it be appropriate to claim that
    the majority of high school seniors are fairly certain that they
    will participate in a study abroad program in college?
\end{parts}
}{}

% 28 - legalize_marijuana_CI_decision

\eoce{\qt{Legalization of marijuana, Part I\label{legalize_marijuana_CI_decision}} 
The General Social Survey asked a random sample of 1,578 US residents:
``Do you think the use of marijuana should be made legal, or not?''
61\% of the respondents said 
it should be made legal.\footfullcite{data:gss}
\begin{parts}
\item Is 61\% a sample statistic or a population parameter? Explain.
\item Construct a 95\% confidence interval for the proportion of US 
residents who think marijuana should be made legal, and interpret it in the 
context of the data.
\item A critic points out that this 95\% confidence interval is only 
accurate if the statistic follows a normal distribution, or if the normal 
model is a good approximation. Is this true for these data? Explain.
\item A news piece on this survey's findings states, ``Majority of Americans 
think marijuana should be legalized.'' Based on your confidence 
interval, is this news piece's statement justified? 
\end{parts}

% 2348 surveyed
% 770 not asked question
% 2348 - 770 = 1578 asked question
% 968 said legalize
% 968 / 1578 = 0.61
}{}

% 29 - national_health_plan_CI_sample_size_replaced

\eoce{\qt{National Health Plan,
    Part II\label{national_health_plan_CI_sample_size_replaced}} 
Exercise~\ref{national_health_plan_HT} presents the results
of a poll evaluating support for a generic
``National Health Plan'' in the US in 2019,
reporting that 55\% of Independents are supportive.
If we want to estimate the percent of Independents who are supportive this year to within 1\% with
90\% confidence, what would be an appropriate sample size?
}{}

% 30 - legalize_marijuana_CI_sample_size

\eoce{\qt{Legalize Marijuana, Part II\label{legalize_marijuana_CI_sample_size}} As 
discussed in Exercise~\ref{legalize_marijuana_CI_decision},
the General Social Survey reported a sample where about
61\% of US residents thought marijuana should be made legal.
If we wanted to limit the margin of error of 
a 95\% confidence interval to 2\%, about how many
Americans would we need to survey?
}{}
