\exercisesheader{}

% 17

\eoce{\qt{Social experiment, Part I\label{social_experiment_conditions}} A ``social 
experiment" conducted by a TV program questioned what people do when they see 
a very obviously bruised woman getting picked on by her boyfriend. On two 
different occasions at the same restaurant, the same couple was depicted. In 
one scenario the woman was dressed ``provocatively'' and in the other 
scenario the woman was dressed ``conservatively''. The table below shows how 
many restaurant diners were present under each scenario, and whether or not 
they intervened.
\begin{center}
\begin{tabular}{ll cc c} 
            &               & \multicolumn{2}{c}{\textit{Scenario}} \\
\cline{3-4}
                            &       & Provocative   & Conservative  & Total \\
\cline{2-5}
\multirow{2}{*}{\textit{Intervene}} &Yes & 5   & 15  & 20    \\
                            &No     & 15      & 10  & 25 \\
\cline{2-5}
                            &Total  & 20      & 25  & 45 \\
\end{tabular}
\end{center}
Explain why the sampling distribution of the difference between the 
proportions of interventions under provocative and conservative scenarios 
does not follow an approximately normal distribution.
}{}

% 18

\eoce{\qt{Heart transplant success\label{heart_transplant_conditions}} The Stanford 
University Heart Transplant Study was conducted to determine whether an 
experimental heart transplant program increased lifespan. Each patient 
entering the program was officially designated a heart transplant candidate, 
meaning that he was gravely ill and might benefit from a new heart. Patients 
were randomly assigned into treatment and control groups. Patients in the 
treatment group received a transplant, and those in the control group did 
not. The table below displays how many patients survived and died in each 
group. \footfullcite{Turnbull+Brown+Hu:1974}\vspace{-2mm}
\begin{center}
\begin{tabular}{rcc}
\hline
            & control   & treatment \\ 
\hline
alive       & 4         & 24 \\ 
dead        & 30        & 45 \\ 
\hline
\end{tabular}
\end{center}
Suppose we are interested in estimating the difference in survival rate between 
the control and treatment groups using a confidence interval.
Explain why we cannot construct such an interval using the normal 
approximation. What might go wrong if we constructed the confidence interval 
despite this problem?
}{}

% 19

\eoce{\qt{Gender and color preference\label{gender_color_preference_CI_concept}} 
A study asked 1,924 male and 3,666 female undergraduate college students 
their favorite color.
A 95\% confidence interval for the difference between 
the proportions of males and females whose favorite color is black 
$(p_{male} - p_{female})$ was calculated to be (0.02, 0.06).
Based on this 
information, determine if the following statements are true or false, and 
explain your reasoning for each statement you identify as false.
\footfullcite{Ellis:2001}
\begin{parts}
\item We are 95\% confident that the true proportion of males whose favorite 
color is black is 2\% lower to 6\% higher than the true proportion of females 
whose favorite color is black.
\item We are 95\% confident that the true proportion of males whose favorite 
color is black is 2\% to 6\% higher than the true proportion of females whose 
favorite color is black.
\item 95\% of random samples will produce 95\% confidence intervals that 
include the true difference between the population proportions of males and 
females whose favorite color is black.
\item We can conclude that there is a significant difference between the 
proportions of males and females whose favorite color is black and that the 
difference between the two sample proportions is too large to plausibly be 
due to chance.
\item The 95\% confidence interval for $(p_{female} - p_{male})$ cannot be 
calculated with only the information given in this exercise.
\end{parts}
}{}

\D{\newpage}

% 20

\eoce{\qt{The Daily Show\label{daily_show_edu_CI_concept}}
A Pew Research foundation poll indicates that among
1,099 college graduates, 33\% watch The Daily Show.
Meanwhile, 22\% of the 1,110 people with a high school degree but 
no college degree in the poll watch The Daily Show. A 95\% confidence 
interval for $(p_\text{college grad} - p_\text{HS or less})$, where $p$ is 
the proportion of those who watch The Daily Show, is (0.07, 0.15). Based on 
this information, determine if the following statements are true or false, 
and explain your reasoning if you identify the statement as false.
\footfullcite{data:dailyShow}
\begin{parts}
\item At the 5\% significance level, the data provide convincing evidence of 
a difference between the proportions of college graduates and those with a 
high school degree or less who watch The Daily Show.
\item We are 95\% confident that 7\% less to 15\% more college graduates 
watch The Daily Show than those with a high school degree or less.
\item 95\% of random samples of 1,099 college graduates and 1,110 people with 
a high school degree or less will yield differences in sample proportions 
between 7\% and 15\%.
\item A 90\% confidence interval for 
$(p_\text{college grad} - p_\text{HS or less})$ would be wider.
\item A 95\% confidence interval for 
$(p_\text{HS or less} - p_\text{college grad})$ is (-0.15,-0.07).
\end{parts}
}{}

% 21

\eoce{\qt{National Health Plan,
    Part III\label{national_health_plan_CI_replaced}}
Exercise~\ref{national_health_plan_HT}
presents the results of a poll evaluating support for
a generically branded ``National Health Plan''
in the United States.
79\% of 347 Democrats and 55\% of 617 Independents
support a National Health Plan.
\begin{parts}
\item
    Calculate a 95\% confidence interval for the
    difference between the proportion of Democrats
    and Independents who support a National
    Health Plan $(p_{D} - p_{I})$, and interpret
    it in this context.
    We have already checked conditions for you.
\item
    True or false:
    If we had picked a random Democrat and a random
    Independent at the time of this poll, it is more
    likely that the Democrat would support the National
    Health Plan than the Independent.
\end{parts}
}{}

% 22

\eoce{\qt{Sleep deprivation, CA vs. OR, Part I\label{sleep_OR_CA_CI}} According to 
a report on sleep deprivation by the Centers for Disease Control and Prevention, 
the proportion of California residents who reported insufficient rest or sleep 
during each of the preceding 30 days is 8.0\%, while this proportion is 8.8\% 
for Oregon residents. These data are based on simple random samples of 11,545 
California and 4,691 Oregon residents. Calculate a 95\% confidence interval 
for the difference between the proportions of Californians and Oregonians who 
are sleep deprived and interpret it in context of the data.\footfullcite{data:sleepCAandOR}
}{}

% 23

\eoce{\qt{Offshore drilling, Part I\label{offshore_drill_edu_dontknow_HT}}
A survey asked 827 randomly sampled registered voters in California
``Do you support? Or do you oppose? Drilling for oil and natural gas
off the Coast of California? Or do you not know enough to say?''
Below is the distribution of 
responses, separated based on whether or not the respondent graduated from 
college. \footfullcite{data:prop19_and_offshoreDrill} \\[1.3mm]

\noindent\begin{minipage}[c]{0.6\textwidth}
\begin{parts}
\item What percent of college graduates and what percent of the non-college 
graduates in this sample do not know enough to have an opinion on drilling 
for oil and natural gas off the Coast of California?
\item Conduct a hypothesis test to determine if the data provide strong 
evidence that the proportion of college graduates who do not have an opinion 
on this issue is different than that of non-college graduates.
\end{parts}
\end{minipage}
\begin{minipage}[c]{0.4\textwidth}
\begin{center}
\begin{tabular}{l c c}
			& \multicolumn{2}{c}{\textit{College Grad}} \\
\cline{2-3}
			& Yes		& No	\\
\cline{1-3}
Support		& 154		& 132	\\
Oppose		& 180		& 126	\\
Do not know	& 104		& 131	\\
\cline{1-3}
 Total		& 438		& 389		
\end{tabular}
\end{center}
\end{minipage}
}{}

% 24

\eoce{\qt{Sleep deprivation, CA vs. OR, Part II\label{sleep_OR_CA_HT}} 
Exercise~\ref{sleep_OR_CA_CI} provides data on sleep deprivation rates of 
Californians and Oregonians. The proportion of California residents who 
reported insufficient rest or sleep during each of the preceding 30 days is 
8.0\%, while this proportion is 8.8\% for Oregon residents. These data are 
based on simple random samples of 11,545 California and 4,691 Oregon 
residents. 
\begin{parts}
\item Conduct a hypothesis test to determine if these data provide strong 
evidence the rate of sleep deprivation is different for the two states. 
(Reminder: Check conditions)
\item It is possible the conclusion of the test in part (a) is incorrect. If 
this is the case, what type of error was made?
\end{parts}
}{}

\D{\newpage}

% 25

\eoce{\qt{Offshore drilling, Part II\label{offshore_drill_edu_support_HT}}
Results of a poll evaluating support for drilling for oil
and natural gas off the coast of California were introduced
in Exercise~\ref{offshore_drill_edu_dontknow_HT}.
\begin{center}
\begin{tabular}{l c c}
				& \multicolumn{2}{c}{\textit{College Grad}} \\
\cline{2-3}
						& Yes		& No				\\
\cline{1-3}
Support		& 154		& 132			\\
Oppose		& 180		& 126			\\
Do not know	& 104		& 131			\\
\cline{1-3}
 Total		& 438		& 389		
\end{tabular}
\end{center}
\begin{parts}
\item What percent of college graduates and what percent of the non-college 
graduates in this sample support drilling for oil and natural gas off the Coast 
of California?
\item Conduct a hypothesis test to determine if the data provide strong evidence 
that the proportion of college graduates who support off-shore drilling in California 
is different than that of non-college graduates.
\end{parts}
}{}

% 26

\eoce{\qt{Full body scan, Part I\label{full_body_scan_HT_Error}} A news article 
reports that ``Americans have differing views on two potentially inconvenient 
and invasive practices that airports could implement to uncover potential 
terrorist attacks." This news piece was based on a survey conducted among a 
random sample of 1,137 adults nationwide, where one of the questions on the 
survey was ``Some airports are 
now using `full-body' digital x-ray machines to electronically screen 
passengers in airport security lines. Do you think these new x-ray machines 
should or should not be used at airports?" Below is a summary of responses 
based on party affiliation. \footfullcite{news:fullBodyScan}
\begin{center}
\begin{tabular}{ll  cc c} 
            &   & \multicolumn{3}{c}{\textit{Party Affiliation}} \\
\cline{3-5}
                                &           & Republican & Democrat & Independent   \\
\cline{2-5}
\multirow{3}{*}{\textit{Answer}}& Should    & 264        & 299      & 351 \\
                                & Should not& 38         & 55       & 77 \\
                                & Don't know/No answer & 16 & 15    & 22 \\
\cline{2-5}
                                & Total      & 318       & 369      & 450
\end{tabular}
\end{center}
\begin{parts}
\item Conduct an appropriate hypothesis test evaluating whether there is a 
difference in the proportion of Republicans and Democrats who think the full-
body scans should be applied in airports. Assume that all relevant conditions 
are met.
\item The conclusion of the test in part (a) may be incorrect, meaning a 
testing error was made. If an error was made, was it a Type~1 or a Type~2 
Error? Explain.
\end{parts}
}{}

% 27

\eoce{\qt{Sleep deprived transportation workers\label{sleep_deprived_driver_HT}} 
The National Sleep Foundation conducted a survey on the sleep habits of 
randomly sampled transportation workers and a control sample of non-transportation 
workers. The results of the survey are shown below.
\footfullcite{data:sleepTransport}\vspace{-1.8mm}
\begin{center}
\begin{tabular}{l c c c c c }
						& 			& \multicolumn{4}{c}{\textit{Transportation Professionals}} \\
\cline{3-6}
			& 			& 		& Truck	& Train		& Bus/Taxi/Limo		\\
			& \textit{Control}& Pilots	& Drivers	& Operators	& Drivers	\\
\cline{1-6}
Less than 6 hours of sleep	& 35		& 19		& 35	& 29	& 21	\\
6 to 8 hours of sleep		& 193		& 132	    & 117	& 119	& 131	\\
More than 8 hours			& 64		& 51		& 51	& 32	& 58	\\
\cline{1-6}
Total						& 292		& 202	    & 203	& 180	& 210		
\end{tabular}
\end{center}\vspace{-1.2mm}
Conduct a hypothesis test to evaluate if these data provide evidence of a 
difference between the proportions of truck drivers and non-transportation 
workers (the control group) who get less than 6 hours of sleep per day, i.e. 
are considered sleep deprived.
}{}

\D{\newpage}

% 28

\eoce{\qt{Prenatal vitamins and Autism\label{prenatal_vitamin_autism_HT}} 
Researchers studying the link between prenatal vitamin use and autism 
surveyed the mothers of a random sample of children aged 24 - 60 months with 
autism and conducted another separate random sample for children with typical 
development. The table below shows the number of mothers in each group who 
did and did not use prenatal vitamins during the three months before 
pregnancy (periconceptional period).\footfullcite{Schmidt:2011}\vspace{-1.8mm}
\begin{center}
\begin{tabular}{l l c c c}
		&			& \multicolumn{2}{c}{\textit{Autism}}	&		\\
\cline{3-4}
		&			& Autism		& Typical development		& Total	\\
\cline{2-5}
\textit{Periconceptional}	& No vitamin	& 111	& 70		& 181	\\
\textit{prenatal vitamin}	& Vitamin	& 143		& 159		& 302	\\
\cline{2-5}
							& Total		& 254		& 229		& 483
\end{tabular}
\end{center}\vspace{-4.2mm}
\begin{parts}
\item State appropriate hypotheses to test for independence of use of 
prenatal vitamins during the three months before pregnancy and autism.
\item Complete the hypothesis test and state an appropriate conclusion. 
(Reminder: Verify any necessary conditions for the test.)
\item A New York Times article reporting on this study was titled ``Prenatal 
Vitamins May Ward Off Autism". Do you find the title of this article to be 
appropriate? Explain your answer. Additionally, propose an alternative title.
\footfullcite{news:prenatalVitAutism}
\end{parts}
}{}

% 29

\eoce{\qt{HIV in sub-Saharan Africa\label{hiv_africa_HT}}
In July 2008 the US National Institutes of Health announced
that it was stopping a clinical study early because of unexpected
results.
The study population consisted of HIV-infected women in sub-Saharan
Africa who had been given single dose Nevaripine (a treatment for HIV)
while giving birth, to prevent transmission of HIV to the infant.
The study was a randomized comparison of continued
treatment of a woman (after successful childbirth)
with Nevaripine vs Lopinavir, a second drug used to treat HIV.
240 women participated in the study;
120 were randomized to each of the two treatments.
Twenty-four weeks after starting the study treatment,
each woman was tested to determine if the HIV infection
was becoming worse (an outcome called \textit{virologic failure}).
Twenty-six of the 120 women treated with Nevaripine experienced
virologic failure, while 10 of the 120 women treated with the
other drug experienced virologic failure.\footfullcite{Lockman:2007}
\begin{parts}
\item Create a two-way table presenting the results of this study.
\item State appropriate hypotheses to test for difference in virologic failure
rates between treatment groups.
\item Complete the hypothesis test and state an appropriate conclusion. 
(Reminder: Verify any necessary conditions for the test.)
\end{parts}
}{}

% 30

\eoce{\qt{An apple a day keeps the doctor
    away\label{apple_doctor_HT_concept}}
A physical education teacher at a high school wanting
to increase awareness on issues of nutrition and health
asked her students at the beginning of the semester
whether they believed the expression
``an apple a day keeps the doctor away'',
and 40\% of the students responded yes.
Throughout the semester she started each class with
a brief discussion of a study highlighting positive
effects of eating more fruits and vegetables.
She conducted the same apple-a-day survey at the end
of the semester, and this time 60\% of the students
responded yes.
Can she used a two-proportion method from this section
for this analysis?
Explain your reasoning.
}{}
