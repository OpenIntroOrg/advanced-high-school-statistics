\reviewexercisesheader{}

% 39

\eoce{\qt{Active learning\label{active_learning_HT_concept}} A teacher wanting to 
increase the active learning component of her course is concerned about 
student reactions to changes she is planning to make. She conducts a survey 
in her class, asking students whether they believe more active learning in 
the classroom (hands on exercises) instead of traditional lecture will helps 
improve their learning. She does this at the beginning and end of the 
semester and wants to evaluate whether students' opinions have changed over 
the semester. Can she used the methods we learned in this chapter for this 
analysis? Explain your reasoning.
}{}

% 40

\eoce{\qt{Website experiment\label{web_ctr_exp_chisq}}
The OpenIntro website occasionally experiments with design
and link placement.
We conducted one experiment testing three different placements
of a download link for this textbook on the book's main page
to see which location, if any, led to the most downloads.
The number of site visitors included in the experiment
was~701 and is captured in one of the response combinations
in the following table:
\begin{center}
\begin{tabular}{r cc}
  \hline
            & Download & No Download \\ 
  \hline
Position 1  & 13.8\%   & 18.3\% \\ 
Position 2  & 14.6\%   & 18.5\% \\ 
Position 3  & 12.1\%   & 22.7\% \\ 
  \hline
\end{tabular}
\end{center}
% x <- matrix(c(97, 102, 85, 128, 130, 159), 3, 2)
\begin{parts}
\item
    Calculate the actual number of site visitors in each
    of the six response categories.
\item
    Each individual in the experiment had an equal chance of
    being in any of the three experiment groups.
    However, we see that there are slightly different
    totals for the groups.
    Is there any evidence that the groups were actually imbalanced?
    Make sure to clearly state hypotheses, check conditions,
    calculate the appropriate test statistic and the p-value,
    and make your conclusion in context of the data.
\item
   Complete an appropriate hypothesis test to check whether
   there is evidence that there is a higher rate of site visitors
   clicking on the textbook link in any of the three groups.
\end{parts}
}{}

% 41

\eoce{\qt{Shipping holiday gifts\label{ship_gifts_chisq_indep_conditions}} A 
local news survey asked 500 randomly sampled Los Angeles residents 
which shipping carrier they prefer to use for shipping holiday gifts. 
The table below shows the distribution of responses by age group as 
well as the expected counts for each cell (shown in parentheses).
\begin{center}
\begin{tabular}{l l | c c | c c | c c | c }
		&		& \multicolumn{6}{c|}{\textit{Age}}	&		\\
\cline{3-8}
		&		& \multicolumn{2}{c|}{18-34}		& \multicolumn{2}{c|}{35-54}	& \multicolumn{2}{c|}{55+}	& Total	\\
\cline{2-9}
\multirow{5}{*}{\textit{Shipping Method}}	& USPS		& 72	& \ec{81}	& 97	& \ec{102}	& 76 	& \ec{62}		& 245 \\
		& UPS		& 52	& \ec{53}	& 76	& \ec{68}	& 34	& \ec{41}		& 162 \\
		& FedEx		& 31	& \ec{21}	& 24	& \ec{27}	& 9	& \ec{16}		& 64 \\
		& Something else	& 7 & \ec{5}	& 6	& \ec{7}	& 3	& \ec{4}		& 16 \\
		& Not sure	& 3	& \ec{5}	& 6	& \ec{5}	& 4	& \ec{3}		& 13 \\
\cline{2-9}
		& Total		& \multicolumn{2}{c|}{165}		& \multicolumn{2}{c|}{209}		& \multicolumn{2}{c|}{126}		& 500
\end{tabular}
\end{center} 
\begin{parts}
\item State the null and alternative hypotheses for testing for independence 
of age and preferred shipping method for holiday gifts among Los Angeles residents.
\item Are the conditions for inference using a chi-square test satisfied?
\end{parts}
}{}

% 42

\eoce{\qt{The Civil War\label{civil_war_HT_CI_2_sided}}
A national survey conducted 
among a simple random sample of 1,507 adults shows that 56\% of
Americans think the Civil War is still relevant to American politics and 
political life.%
\footfullcite{data:civilWar} 
\begin{parts}
\item Conduct a hypothesis test to determine if these data provide strong 
evidence that the majority of the Americans think the Civil War is still 
relevant.
\item Interpret the p-value in this context.
\item Calculate a 90\% confidence interval for the proportion of Americans 
who think the Civil War is still relevant. Interpret the interval in this 
context, and comment on whether or not the confidence interval agrees with 
the conclusion of the hypothesis test.
\end{parts}
}{}

\D{\newpage}

% 43

\eoce{\qt{College smokers\label{college_smokers_CI_sample_size}} \videohref{ahss_eoce_sol-college_smokers_CI_sample_size}\ \ We are interested 
in estimating the proportion of students at a university who smoke. Out of a 
random sample of 200 students from this university, 40 students smoke.
\begin{parts}
\item Calculate a 95\% confidence interval for the proportion of students at 
this university who smoke, and interpret this interval in context. 
(Reminder: Check conditions.)
\item If we wanted the margin of error to be no larger than 2\% at a 95\% 
confidence level for the proportion of students who smoke, how big of a 
sample would we need? 
\end{parts}
}{}

%44

\eoce{\qt{Acetaminophen and liver damage\label{acetaminophen_CI_sample_size}} It 
is believed that large doses of acetaminophen (the active ingredient in over 
the counter pain relievers like Tylenol) may cause damage to the liver. A 
researcher wants to conduct a study to estimate the proportion of 
acetaminophen users who have liver damage. For participating in this study, 
he will pay each subject \$20 and provide a free medical consultation if the 
patient has liver damage.
\begin{parts}
\item If he wants to limit the margin of error of his 98\% confidence 
interval to 2\%, what is the minimum amount of money he needs to set aside 
to pay his subjects?
\item The amount you calculated in part (a) is substantially over his budget 
so he decides to use fewer subjects. How will this affect the width of his 
confidence interval?
\end{parts}
}{}

% 45

\eoce{\qt{Life after college\label{life_after_college_CI}} We are interested in 
estimating the proportion of graduates at a mid-sized university who found 
a job within one year of completing their undergraduate degree. Suppose we 
conduct a survey and find out that 348 of the 400 randomly sampled graduates 
found jobs. The graduating class under consideration included over 4500 students.
\begin{parts}
\item Describe the population parameter of interest. What is the value of 
the point estimate of this parameter?
\item Check if the conditions for constructing a confidence interval based 
on these data are met.
\item Calculate a 95\% confidence interval for the proportion of graduates 
who found a job within one year of completing their undergraduate degree at 
this university, and interpret it in the context of the data.
\item What does ``95\% confidence" mean?
\item Now calculate a 99\% confidence interval for the same parameter and 
interpret it in the context of the data.
\item Compare the widths of the 95\% and 99\% confidence intervals. Which 
one is wider? Explain.
\end{parts}
}{}

% 46

\eoce{\qt{Diabetes and unemployment\label{diabetes_unemp_effect_size}} A 
Gallup poll surveyed Americans about their employment status and whether or 
not they have diabetes. The survey results indicate that 1.5\% of the 47,774 
employed (full or part time) and 2.5\% of the 5,855 unemployed 18-29 year 
olds have diabetes.\footfullcite{data:employmentDiabetes}
\begin{parts}
\item Create a two-way table presenting the results of this study.
\item State appropriate hypotheses to test for difference in proportions of 
diabetes between employed and unemployed Americans.
\item The sample difference is about 1\%. If we completed the hypothesis 
test, we would find that the p-value is very small (about 0), meaning the 
difference is statistically significant. Use this result to explain the 
difference between statistically significant and practically significant 
findings.
\end{parts}
}{}

% 47

\eoce{\qt{Rock-paper-scissors\label{rps_chisq_GOF}} Rock-paper-scissors is a hand 
game played by two or more people where players choose to sign either rock, 
paper, or scissors with their hands. For your statistics class project, 
you want to evaluate whether players choose between these three options 
randomly, or if certain options are favored above others. You ask two friends 
to play rock-paper-scissors and count the times each option is played. The 
following table summarizes the data:
\begin{center}
\begin{tabular}{c c c}
Rock	& Paper	& Scissors 	 \\
\hline
43		& 21	& 35	
\end{tabular}
\end{center}
Use these data to evaluate whether players choose between these three options 
randomly, or if certain options are favored above others. Make sure to clearly 
outline each step of your analysis, and interpret your results in context of 
the data and the research question.
}{}

\D{\newpage}

% 48

\eoce{\qt{2010 Healthcare Law\label{healthcare_CI_concept}} On June 28, 2012 the 
U.S. Supreme Court upheld the much debated 2010 healthcare law, declaring it 
constitutional. A Gallup poll released the day after this decision indicates 
that 46\% of 1,012 Americans agree with this decision. At a 95\% confidence 
level, this sample has a 3\% margin of error. Based on this information, 
determine if the following statements are true or false, and explain your 
reasoning.\footfullcite{data:healthcare2010}
\begin{parts}
\item We are 95\% confident that between  43\% and 49\% of Americans in this 
sample support the decision of the U.S. Supreme Court on the 2010 healthcare 
law.
\item We are 95\% confident that between 43\% and 49\% of Americans support 
the decision of the U.S. Supreme Court on the 2010 healthcare law.
\item If we considered many random samples of 1,012 Americans, and we 
calculated the sample proportions of those who support the decision of the 
U.S. Supreme Court, 95\% of those sample proportions will be between 43\% and 
49\%.
\item The margin of error at a 90\% confidence level would be higher than 3\%.
\end{parts}
}{}

% 49

\eoce{\qt{Browsing on the mobile device\label{mobile_browsing_HT_CI}} A 
survey of 2,254 American adults indicates that 17\% of cell phone owners
browse the internet exclusively on their phone rather than a computer
or other device.
\footfullcite{data:mobileBrowse}
\begin{parts}
\item  According to an online article, a report from a mobile research 
company indicates that 38 percent of Chinese mobile web users only access 
the internet through their cell phones.
\footfullcite{news:mobileBrowseChinese} Conduct a hypothesis test to 
determine if these data provide strong evidence that the proportion of 
Americans who only use their cell phones to access the internet is different 
than the Chinese proportion of 38\%.
\item Interpret the p-value in this context.
\item Calculate a 95\% confidence interval for the proportion of Americans 
who access the internet on their cell phones, and interpret the interval in 
this context.
\end{parts}
}{}

% 50

\eoce{\qt{Coffee and Depression\label{coffee_depression_chisq_indep}} 
Researchers conducted a study investigating the relationship between 
caffeinated coffee consumption and risk of depression in women. They 
collected data on 50,739 women free of depression symptoms at the start 
of the study in the year 1996, and these women were followed through 
2006. The researchers used questionnaires to collect data on 
caffeinated coffee consumption, asked each individual about physician-
diagnosed depression, and also asked about the use of antidepressants. 
The table below shows the distribution of incidences of depression by 
amount of caffeinated coffee consumption.\footfullcite{Lucas:2011}
\begin{adjustwidth}{-4em}{-4em}
{\small
\begin{center}
\begin{tabular}{l  l rrrrrr}
	&  \multicolumn{1}{c}{}		& \multicolumn{5}{c}{\textit{Caffeinated coffee consumption}} \\
\cline{3-7}
	&		& $\le$ 1	& 2-6	& 1	& 2-3	& $\ge$ 4	&   \\
	&		& cup/week	& cups/week	& cup/day	& cups/day	& cups/day	& Total  \\
\cline{2-8}
\textit{Clinical} & Yes	& 670 & \fbox{\textcolor{oiB}{373}}	& 905	& 564	& 95 	& 2,607 \\
\textit{depression}	& No& 11,545	& 6,244	& 16,329	& 11,726	& 2,288 	& 48,132 \\
\cline{2-8}
				& Total	& 12,215	& 6,617 & 17,234	& 12,290	& 2,383 	& 50,739 \\
\cline{2-8}
\end{tabular}
\end{center}
}
\end{adjustwidth}
\begin{parts}
\item What type of test is appropriate for evaluating if there is an 
association between coffee intake and depression?
\item Write the hypotheses for the test you identified in part (a).
\item Calculate the overall proportion of women who do and do not 
suffer from depression.
\item Identify the expected count for the highlighted cell, and 
calculate the contribution of this cell to the test statistic, i.e. 
$(Observed-Expected)^2/Expected$.
\item The test statistic is $\chi^2=20.93$. What is the p-value?
\item What is the conclusion of the hypothesis test?
\item One of the authors of this study was quoted on the NYTimes as 
saying it was ``too early to recommend that women load up on extra 
coffee" based on just this study.\footfullcite{news:coffeeDepression} 
Do you agree with this statement? Explain your reasoning.
\end{parts}
}{}
