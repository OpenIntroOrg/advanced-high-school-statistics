\exercisesheader{}

% 49 - sleep_deprived_driver_HT_ahss

\eoce{\qt{Sleep deprived transportation workers\label{sleep_deprived_driver_HT_ahss}} 
The National Sleep Foundation conducted a survey on the sleep habits of 
randomly sampled transportation workers and randomly sampled non-transportation 
workers that serve as a ``control" for comparison. The results of the survey are shown below.
\footfullcite{data:sleepTransport}\vspace{-1.8mm}
\begin{center}
\begin{tabular}{l c c c c c }
						& 			& \multicolumn{4}{c}{\textit{Transportation Professionals}} \\
\cline{3-6}
			& 			& 		& Truck	& Train		& Bus/Taxi/Limo		\\
			& \textit{Control}& Pilots	& Drivers	& Operators	& Drivers	\\
\cline{1-6}
Less than 6 hours of sleep	& 35		& 19		& 35	& 29	& 21	\\
6 to 8 hours of sleep		& 193		& 132	    & 117	& 119	& 131	\\
More than 8 hours			& 64		& 51		& 51	& 32	& 58	\\
\cline{1-6}
Total						& 292		& 202	    & 203	& 180	& 210		
\end{tabular}
\end{center}\vspace{-1.2mm}
Conduct a hypothesis test to evaluate if these data provide evidence of a 
difference between the proportion of truck drivers and non-transportation 
workers (the ``control" group) who get less than 6 hours of sleep per day (i.e. 
are considered sleep deprived).  Remember to Identify, Check, Calculate, and Conclude.
}{}

% 50 - sleep_OR_CA_HT_ahss

\eoce{\qt{Sleep deprivation, CA vs. OR, Part II\label{sleep_OR_CA_HT_ahss}} 
Exercise~\ref{sleep_OR_CA_CI_ahss} provides data on sleep deprivation rates of 
Californians and Oregonians. The proportion of California residents who 
reported insufficient rest or sleep during each of the preceding 30 days is 
8.0\%, while this proportion is 8.8\% for Oregon residents. These data are 
based on simple random samples of 11,545 California and 4,691 Oregon 
residents. 
\begin{parts}
\item Conduct a full hypothesis test to determine if these data provide strong 
evidence that the rate of sleep deprivation is different for the two states.  Remember to Identify, Check, Calculate and Conclude.
\item It is possible the conclusion of the test in part (a) is incorrect. If 
this is the case, what type of error was made?
\end{parts}
}{}

% 51 - malaria_vaccine_ahss

\eoce{\qt{Malaria vaccine\label{malaria_vaccine_ahss}} 
With no currently licensed vaccines to inhibit malaria, good news was welcomed with a recent study reporting long-awaited vaccine success for children in Burkina Faso. With 450 children randomized to either one of two different doses of the malaria vaccine or a control vaccine, 89 of 292 malaria vaccine and 106 out of 147 control vaccine children contracted malaria within 12 months after the treatment.
\begin{parts}
\item Conduct a full hypothesis test to determine if these data provide evidence that a lower proportion of children like those in the study would contract malaria if given the  malaria vaccine than if given the control vaccine. Remember to Identify, Check, Calculate and Conclude.
\item Interpret the p-value that you calculated in the context of the problem.
\end{parts}
}{}

% 52 - prenatal_vitamin_autism_HT_props_ahss

\eoce{\qt{Prenatal vitamins and Autism\label{prenatal_vitamin_autism_HT_props_ahss}} 
Researchers studying the link between prenatal vitamin use and autism 
surveyed the mothers of a random sample of children aged 24 - 60 months with 
autism and conducted another separate random sample for children with typical 
development. The table below shows the number of mothers in each group who 
did and did not use prenatal vitamins during the three months before 
pregnancy (periconceptional period).\footfullcite{Schmidt:2011}\vspace{-1.8mm}
\begin{center}
\begin{tabular}{l l c c c}
		&			& \multicolumn{2}{c}{\textit{Autism}}	&		\\
\cline{3-4}
		&			& Autism		& Typical development		& Total	\\
\cline{2-5}
\textit{Periconceptional}	& No vitamin	& 111	& 70		& 181	\\
\textit{prenatal vitamin}	& Vitamin	& 143		& 159		& 302	\\
\cline{2-5}
							& Total		& 254		& 229		& 483
\end{tabular}
\end{center}\vspace{-4.2mm}
\begin{parts}
\item State appropriate hypotheses to test for a difference in autism rates between those whose mothers used prenatal vitamins during the three months before pregnancy and those whose mothers did not.
\item Complete the hypothesis test and state an appropriate conclusion. 
(Reminder: Verify any necessary conditions for the test.)
\item A New York Times article reporting on this study was titled ``Prenatal 
Vitamins May Ward Off Autism". Do you find the title of this article to be 
appropriate? Explain your answer. Additionally, propose an alternative title.
\footfullcite{news:prenatalVitAutism}
\end{parts}
}{}
