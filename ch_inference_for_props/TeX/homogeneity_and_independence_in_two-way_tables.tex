\exercisesheader{}

% 35 - quitters_chisq_independence

\eoce{\qt{Quitters\label{quitters_chisq_independence}} Does being part of a 
support group affect the ability of people to quit smoking? A county 
health department enrolled 300 smokers in a randomized experiment. 150 
participants were randomly assigned to a group that used a nicotine patch and 
met weekly with a support group; the other 150 received the patch and 
did not meet with a support group. At the end of the study, 40 of the 
participants in the patch plus support group had quit smoking while 
only 30 smokers had  quit in the other group.
\begin{parts}
\item Create a two-way table presenting the results of this study.
\item Answer each of the following questions under the null hypothesis 
that being part of a support group does not affect the ability of 
people to quit smoking, and indicate whether the expected values are 
higher or lower than the observed values.
\begin{subparts}
\item How many subjects in the ``patch + support" group would you 
expect to quit?
\item How many subjects in the ``patch only" group would you expect to 
not quit?
\end{subparts}
\end{parts}
}{}

% 36 - full_body_scan_chisq_indep

\eoce{\qt{Full body scan, Part II\label{full_body_scan_chisq_indep}} The 
table below summarizes a data set we first encountered in 
Exercise~\ref{full_body_scan_HT_Error} regarding views on full-body 
scans and political affiliation. The differences in each political 
group may be due to chance. Complete the following computations under 
the null hypothesis of independence between an individual's party 
affiliation and his support of full-body scans. It may be useful to 
first add on an extra column for row totals before proceeding with the 
computations.
\begin{center}
\begin{tabular}{ll  cc c} 
            &   & \multicolumn{3}{c}{\textit{Party Affiliation}} \\
\cline{3-5}
                                &           & Republican & Democrat & Independent   \\
\cline{2-5}
\multirow{3}{*}{\textit{Answer}}& Should    & 264        & 299      & 351 \\
                                & Should not& 38         & 55       & 77 \\
                                & Don't know/No answer & 16 & 15    & 22 \\
\cline{2-5}
                                & Total      & 318       & 369      & 450
\end{tabular}
\end{center}
\begin{parts}
\item How many Republicans would you expect to not support the use of 
full-body scans?
\item How many Democrats would you expect to support the use of full-
body scans?
\item How many Independents would you expect to not know or not answer?
\end{parts}
}{}

% 37 - offshore_drilling_chisq_indep_ahss

\eoce{\qt{Offshore drilling, Part II\label{offshore_drilling_chisq_indep}} \videosolution{ahss_eoce_sol-offshore_drilling_chisq_indep}
The table below summarizes a data set we first encountered in 
Exercise~\ref{offshore_drill_edu_dontknow_HT} that examines the 
responses of a random sample of college graduates and non-graduates on 
the topic of oil drilling. Complete a chi-square test for these data to 
check whether there is a statistically significant difference in 
responses from college graduates and non-graduates.
\begin{center}
\begin{tabular}{l c c}
			& \multicolumn{2}{c}{\textit{College Grad}} \\
\cline{2-3}
			& Yes		& No				\\
\cline{1-3}
Support		& 154		& 132			\\
Oppose		& 180		& 126			\\
Do not know	& 104		& 131			\\
\cline{1-3}
 Total		& 438		& 389		
\end{tabular}
\end{center}
}{}

% 38 - parasitic_worm_chisq

\eoce{\qt{Parasitic worm\label{parasitic_worm_chisq}}
Lymphatic filariasis is a disease caused by a parasitic worm.
Complications of the disease can lead to extreme swelling
and other complications.
Here we consider results from a randomized experiment
that compared three
different drug treatment options to clear people of the
this parasite, which people are working to eliminate entirely.
The results for the second year of the study are
given below:\footfullcite{King_Suamani_2018}
\begin{center}
\begin{tabular}{l cc}
  \hline
  & Clear at Year 2 & Not Clear at Year 2 \\ 
  \hline
  Three drugs & 52 & 2 \\ 
  Two drugs & 31 & 24 \\ 
  Two drugs annually & 42 & 14 \\ 
  \hline
\end{tabular}
\end{center}
\begin{parts}
\item\label{parasitic_worm_chisq_hyp}
    Set up hypotheses for evaluating
    whether there is any difference in the
    performance of the treatments,
    and also check conditions.
\item
    Statistical software was used to run
    a chi-square test, which output:
    \begin{align*}
    &X^2 = 23.7
    &&df = 2
    &&\text{p-value} = \text{7.2e-6}
    \end{align*}
    Use these results to evaluate the hypotheses
    from part~(\ref{parasitic_worm_chisq_hyp}),
    and provide a conclusion
    in the context of the problem.
\end{parts}
}{}
