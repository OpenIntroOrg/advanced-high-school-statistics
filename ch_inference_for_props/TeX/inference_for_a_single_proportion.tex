\exercisesheader{}

% 1 - veg_coll_students_CLT

\eoce{\qt{Vegetarian college students\label{veg_coll_students_CLT}} Suppose that 8\% 
of college students are vegetarians. Determine if the following statements are 
true or false, and explain your reasoning.
\begin{parts}
\item The distribution of the sample proportions of vegetarians in random 
samples of size 60 is approximately normal since $n \ge 30$. 
\item The distribution of the sample proportions of vegetarian college 
students in random samples of size 50 is right skewed.
\item A random sample of 125 college students where 12\% are vegetarians 
would be considered unusual. 
\item A random sample of 250 college students where 12\% are vegetarians 
would be considered unusual.
\item The standard error would be reduced by one-half if we increased the 
sample size from 125 to~250.
\end{parts}
 
}{}

% 2 - young_americans_CLT_1

\eoce{\qt{Young Americans, Part I\label{young_americans_CLT_1}} About 77\% of 
young adults think they can achieve the American dream. Determine if the 
following statements are true or false, and explain your reasoning.
\footfullcite{news:youngAmericans1}
\begin{parts}
\item The distribution of sample proportions of young Americans who think 
they can achieve the American dream in random samples of size 20 is left skewed.
\item The distribution of sample proportions of young Americans who think 
they can achieve the American dream in random samples of size 40 is 
approximately normal since $n \ge 30$. 
\item A random sample of 60 young Americans where 85\% think they can achieve 
the American dream would be considered unusual.
\item A random sample of 120 young Americans where 85\% think they can 
achieve the American dream would be considered unusual.
\end{parts}
}{}

% 3 - orange_tabbies_CLT

\eoce{\qt{Orange tabbies\label{orange_tabbies_CLT}} Suppose that 90\% of orange
tabby cats are male. Determine if the following statements are true or false, 
and explain your reasoning.
\begin{parts}
\item The distribution of sample proportions of random samples of size 30 is 
left skewed.
\item Using a sample size that is 4 times as large will reduce the standard 
error of the sample proportion by one-half.
\item The distribution of sample proportions of random samples of size 140 is 
approximately normal.
\item The distribution of sample proportions of random samples of size 280 is 
approximately normal.
\end{parts}
}{}

% 4 - young_americans_CLT_2

\eoce{\qt{Young Americans, Part II\label{young_americans_CLT_2}} About 25\% of 
young Americans have delayed starting a family due to the continued economic 
slump. Determine if the following statements are true or false, and explain 
your reasoning.\footfullcite{news:youngAmericans2}
\begin{parts}
\item The distribution of sample proportions of young Americans who have 
delayed starting a family due to the continued economic slump in random 
samples of size 12 is right skewed.
\item In order for the distribution of sample proportions of young Americans 
who have delayed starting a family due to the continued economic slump to be 
approximately normal, we need random samples where the sample size is at 
least 40.
\item A random sample of 50 young Americans where 20\% have delayed starting 
a family due to the continued economic slump would be considered unusual.
\item A random sample of 150 young Americans where 20\% have delayed 
starting a family due to the continued economic slump would be considered 
unusual.
\item Tripling the sample size will reduce the standard error of the sample 
proportion by one-third.
\end{parts}
}{}

\D{\newpage}
% 5 - gender_equality

\eoce{\qt{Gender equality\label{gender_equality}}
The General Social Survey asked a random sample of
1,390 Americans the following question:
``On the whole, do you think it should or should not be
the government's responsibility to  promote equality
between men and women?''
82\% of the respondents said it ``should be''.
At a 95\% confidence level, this sample has 2\% margin of error.
Based on this information, determine if the following statements
are true or false, and explain your reasoning.\footfullcite{data:gss}
\begin{parts}
\item We are 95\% confident that between 80\% and 84\% of Americans in this 
sample think it's the government's responsibility to promote equality between 
men and women.
\item We are 95\% confident that between 80\% and 84\% of all Americans 
think it's the government's responsibility to promote equality between 
men and women.
\item If we considered many random samples of 1,390 Americans, and we calculated 
95\% confidence intervals for each, 95\% of these intervals would include the 
true population proportion of Americans who think it's the government's 
responsibility to promote equality between men and women.
\item In order to decrease the margin of error to 1\%, we would need to 
quadruple (multiply by 4) the sample size.
\item Based on this confidence interval, there is sufficient evidence to 
conclude that a majority of Americans think it's the government's responsibility 
to promote equality between men and women.
\end{parts}

% n = 1390
% should be: 1142
% p = 1142/1390 = 0.82
% me = sqrt(.82*.08/1390)*1.96 = 0.02
}{}

% 6 - elderly_drivers_CI_concept

\eoce{\qt{Elderly drivers\label{elderly_drivers_CI_concept}} 
The Marist Poll published a report stating that 66\% of adults nationally 
think licensed drivers should be required to retake their road test once 
they reach 65 years of age. It was also reported that interviews were 
conducted on a random sample of 1,018 American adults, and that the margin of error was 3\% 
using a 95\% confidence level. \footfullcite{data:elderlyDriving}
\begin{parts}
\item Verify the margin of error reported by The Marist Poll. 
\item Based on a 95\% confidence interval, does the poll provide convincing 
evidence that \textit{more than} two thirds of the population think that licensed 
drivers should be required to retake their road test once they turn 65?
\end{parts}
}{}

% 7 - fireworks_CI_concept

\eoce{\qt{Fireworks on July 4$^{\text{th}}$\label{fireworks_CI_concept}} A local 
news outlet reported that 56\% of 600 randomly sampled Kansas residents planned 
to set off fireworks on July~$4^{th}$. Determine the margin of error for the 
56\% point estimate using a 95\% confidence level.\footfullcite{data:july4}
}{}

% 8 - greece_life_rating_CI

\eoce{\qt{Life rating in Greece\label{greece_life_rating_CI}} Greece has faced a 
severe economic crisis since the end of 2009. A Gallup poll surveyed 1,000 
randomly sampled Greeks in 2011 and found that 25\% of them said they would 
rate their lives poorly enough to be considered ``suffering''.\footfullcite{data:suffering}
\begin{parts}
\item Describe the population parameter of interest. What is the value of the 
point estimate of this parameter?
\item Check if the conditions required for constructing a confidence interval 
based on these data are met.
\item Construct a 95\% confidence interval for the proportion of Greeks who 
are ``suffering".
\item Without doing any calculations, describe what would happen to the 
confidence interval if we decided to use a higher confidence level.
\item Without doing any calculations, describe what would happen to the 
confidence interval if we used a larger sample.
\end{parts}
}{}

% 9 - study_abroad_CI_decision

\eoce{\qt{Study abroad\label{study_abroad_CI_decision}}
A survey on 1,509 high school seniors who took the SAT
and who completed an optional web survey shows that
55\% of high school seniors are fairly certain that
they will participate in a study abroad program in 
college.\footfullcite{data:studyAbroad}
\begin{parts}
\item
    Is this sample a representative sample from the population
    of all high school seniors in the US?
    Explain your reasoning.
\item
    Let's suppose the conditions for inference are met.
    Even if your answer to part (a) indicated that this approach
    would not be reliable, this analysis may still be interesting
    to carry out (though not report).
    Construct a 90\% confidence interval for the proportion of high
    school seniors (of those who took the SAT) who are fairly certain
    they will participate in a study abroad program in college,
    and interpret this interval in context.
\item
    What does ``90\% confidence" mean?
\item
    Based on this interval, would it be appropriate to claim that
    the majority of high school seniors are fairly certain that they
    will participate in a study abroad program in college?
\end{parts}
}{}

% 10 - legalize_marijuana_CI_decision

\eoce{\qt{Legalization of marijuana, Part I\label{legalize_marijuana_CI_decision}} 
The General Social Survey asked a random sample of 1,578 US residents:
``Do you think the use of marijuana should be made legal, or not?''
61\% of the respondents said 
it should be made legal.\footfullcite{data:gss}
\begin{parts}
\item Is 61\% a sample statistic or a population parameter? Explain.
\item Construct a 95\% confidence interval for the proportion of US 
residents who think marijuana should be made legal, and interpret it in the 
context of the data.
\item A critic points out that this 95\% confidence interval is only 
accurate if the statistic follows a normal distribution, or if the normal 
model is a good approximation. Is this true for these data? Explain.
\item A news piece on this survey's findings states, ``Majority of Americans 
think marijuana should be legalized.'' Based on your confidence 
interval, is this news piece's statement justified? 
\end{parts}

% 2348 surveyed
% 770 not asked question
% 2348 - 770 = 1578 asked question
% 968 said legalize
% 968 / 1578 = 0.61
}{}

% 11 - national_health_plan_HT

\eoce{\qt{National Health Plan, Part I\label{national_health_plan_HT}}
A \textit{Kaiser Family Foundation} poll for a random sample of US adults
in 2019 found that 79\% of Democrats, 55\% of Independents,
and 24\% of Republicans supported a generic ``National Health Plan''.
There were 347 Democrats, 298 Republicans, and 617 Independents
surveyed.\footfullcite{data:KFF2019_nat_health_plan}
\begin{parts}
\item
    A political pundit on TV claims that a majority of Independents
    support a National Health Plan.
    Do these data provide strong evidence to support this type
    of statement?
\item
    Would you expect a confidence interval for the proportion
    of Independents who oppose the public option plan to
    include 0.5?
    Explain.
\end{parts}
}{}

% 12 - college_worth_it_HT_CI

\eoce{\qt{Is college worth it? Part I\label{college_worth_it_HT_CI}} Among a simple 
random sample of 331 American adults who do not have a four-year college degree 
and are not currently enrolled in school, 48\% said they decided not to go to 
college because they could not afford school. \footfullcite{data:collegeWorthIt}
\begin{parts}
\item A newspaper article states that only a minority of the Americans who 
decide not to go to college do so because they cannot afford it and uses the 
point estimate from this survey as evidence. Conduct a hypothesis test to 
determine if these data provide strong evidence supporting this statement.
\item Would you expect a confidence interval for the proportion of American 
adults who decide not to go to college because they cannot afford it to 
include 0.5? Explain.
\end{parts}
}{}

% 13 - taste_test_HT

\eoce{\qt{Taste test\label{taste_test_HT}} \videosolution{ahss_eoce_sol-taste_test_HT} Some people claim that they can tell the 
difference between a diet soda and a regular soda in the first sip. A researcher 
wanting to test this claim randomly sampled 80 such people. He then filled 80 
plain white cups with soda, half diet and half regular through random assignment, 
and asked each person to take one sip from their cup and identify the soda as 
diet or regular. 53 participants correctly identified the soda.
\begin{parts}
\item Do these data provide strong evidence that these people are able to detect 
the difference between diet and regular soda, in other words, are the results 
significantly better than just random guessing?
\item Interpret the p-value in this context.
\end{parts}
}{}

% 14 - college_worth_it_CI_sample_size

\eoce{\qt{Is college worth it? Part II\label{college_worth_it_CI_sample_size}} 
Exercise~\ref{college_worth_it_HT_CI} presents a poll where 
48\% of 331 randomly selected Americans reported that they decided not to go to college because they 
cannot afford it.
\begin{parts}
\item Calculate a 90\% confidence interval for the proportion of Americans 
who decide to not go to college because they cannot afford it, and interpret 
the interval in context.
\item Suppose we wanted the margin of error for the 90\% confidence level to 
be about 1.5\%. How large of a survey would you recommend?
\end{parts}
}{}

% 15 - national_health_plan_CI_sample_size_replaced

\eoce{\qt{National Health Plan,
    Part II\label{national_health_plan_CI_sample_size_replaced}} 
Exercise~\ref{national_health_plan_HT} presents the results
of a poll evaluating support for a generic
``National Health Plan'' in the US in 2019,
reporting that 55\% of Independents are supportive.
If we want to estimate the percent of Independents who are supportive this year to within 1\% with
90\% confidence, what would be an appropriate sample size?
}{}

% 16 - legalize_marijuana_CI_sample_size

\eoce{\qt{Legalize Marijuana, Part II\label{legalize_marijuana_CI_sample_size}} As 
discussed in Exercise~\ref{legalize_marijuana_CI_decision},
the General Social Survey reported a sample where about
61\% of US residents thought marijuana should be made legal.
If we wanted to limit the margin of error of 
a 95\% confidence interval to 2\%, about how many
Americans would we need to survey?
}{}
