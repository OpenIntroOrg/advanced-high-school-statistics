\exercisesheader{}

% 31 - scope_airpoll

\eoce{\qt{Air pollution and birth outcomes, scope of inference\label{scope_airpoll}} 
Exercise~\ref{study_components_airpoll} introduces a study where researchers 
collected data to examine the relationship between air pollutants and preterm 
births in Southern California. During the study air pollution levels were 
measured by air quality monitoring stations. Length of gestation data were 
collected on 143,196 births between the years 1989 and 1993, and air pollution 
exposure during gestation was calculated for each birth.
\begin{parts}
\item Identify the population of interest and the sample in this study.
\item Comment on whether or not the results of the study can be generalized to the 
population, and if the findings of the study can be used to establish causal relationships.
\end{parts}
}{}

% 32 - scope_cheaters

\eoce{\qt{Cheaters, scope of inference\label{scope_cheaters}} 
Exercise~\ref{study_components_cheaters} introduces a study where researchers 
studying the relationship between honesty, age, and self-control conducted an 
experiment on 160 children between the ages of 5 and 15. The researchers asked 
each child to toss a fair coin in private and to record the outcome (white or black) 
on a paper sheet, and said they would only reward children who report white. 
Half the students were explicitly told not to cheat and the others were not given 
any explicit instructions. Differences were observed in the cheating rates in the
instruction and no instruction groups, as well as some differences across 
children's characteristics within each group.
\begin{parts}
\item Identify the population of interest and the sample in this study.
\item Comment on whether or not the results of the study can be generalized to the 
population, and if the findings of the study can be used to establish causal 
relationships.
\end{parts}
}{}

% 33 - scope_buteyko

\eoce{\qt{Buteyko method, scope of inference\label{scope_buteyko}} 
Exercise~\ref{study_components_buteyko} introduces a study on using the Buteyko 
shallow breathing technique to reduce asthma symptoms and improve quality of life.
As part of this study 600 asthma patients aged 18-69 who relied on medication for 
asthma treatment were recruited and randomly assigned to two groups: one practiced 
the Buteyko method and the other did not. Those in the Buteyko group experienced,
on average, a significant reduction in asthma symptoms and an improvement in quality 
of life.
\begin{parts}
\item Identify the population of interest and the sample in this study.
\item Comment on whether or not the results of the study can be generalized to the 
population, and if the findings of the study can be used to establish causal 
relationships.
\end{parts}
}{}

% 34 - stressed_out_observational

\eoce{\qt{Stressed out, Part I\label{stressed_out_observational}} A study that 
surveyed a random sample of otherwise healthy high school students found that 
they are more likely to get muscle cramps when they are stressed. The study 
also noted that students drink more coffee and sleep less when they are 
stressed.
\begin{parts}
\item What type of study is this?
\item Can this study be used to conclude a causal relationship between 
increased stress and muscle cramps?
\item State possible confounding variables that might explain the observed 
relationship between increased stress and muscle cramps. 
\end{parts}
}{}
