\reviewexercisesheader{}

% 49 - makeup_exam

\eoce{\qt{Make-up exam\label{makeup_exam}} \videosolution{ahss_eoce_sol-makeup_exam} In a class of 25 students, 24 of them took an exam 
in class and 1 student took a make-up exam the following day. The professor graded the 
first batch of 24 exams and found an average score of 74 points with a standard 
deviation of 8.9 points. The student who took the make-up the following day scored 64 
points on the exam.
\begin{parts}
\item Does the new student's score increase or decrease the average score?
\item What is the new average?
\item Does the new student's score increase or decrease the standard deviation of the 
scores?
\end{parts}
}{}

% 50 - infant_mortality_rel_freq

\eoce{\qt{Infant mortality\label{infant_mortality}} The infant mortality rate is defined as 
the number of infant deaths per 1,000 live births. This rate is often used as an 
indicator of the level of health in a country. The relative frequency histogram below 
shows the distribution of estimated infant death rates for 224 countries for which such 
data were available in 2014. 
\footfullcite{data:ciaFactbook}

\noindent\begin{minipage}[c]{0.43\textwidth}
\begin{parts}
\item Estimate Q1, the median, and Q3 from the histogram.
\item Would you expect the mean of this data set to be smaller or larger than the 
median? Explain your reasoning.
\end{parts} \vfill \
\end{minipage}
\begin{minipage}[c]{0.52\textwidth}
\hfill\FigureFullPath[A histogram is shown for the variable "Infant Mortality (per 1000 live births)" with axis range of 0 to 120. The histogram vertical axis is for "Fraction of Countries" and runs from 0 to 0.4. The bins are as follows: the 0 to 10 bin has a height of 0.38, 10 to 20 has a height of 0.22, 20 to 30 a height of 0.11, 30 to 40 a height of 0.06, 40 to 50 a height of 0.07, 50 to 60 a height of 0.08, 60 to 70 a height of 0.04, 70 to 80 a height of 0.03, 80 to 90 a height of 0.01, 90 to 100 a height of 0.02, and 100 to 110 a height of 0.01.]{0.85}{ch_one_variable_data_collecting_data/figures/eoce/infant_mortality_rel_freq/infant_mortality_rel_freq_hist}
\end{minipage}
}{}

% 51 - dist_shape_tv_watchers

\eoce{\qt{TV watchers\label{dist_shape_TV_watchers}} Students in an AP Statistics class 
were asked how many hours of television they watch per week (including online 
streaming). This sample yielded an average of 4.71 hours, with a standard 
deviation of 4.18 hours. Is the distribution of number of hours students watch 
television weekly symmetric? If not, what shape would you expect this distribution 
to have? Explain your reasoning.
}{}

% 52 - new_stat

\eoce{\qt{A new statistic\label{new_stat}} The statistic $\frac{\bar{x}}{median}$ can 
be used as a measure of skewness. Suppose we have a distribution where all 
observations are greater than 0, $x_i > 0$. What is the expected shape of 
the distribution under the following conditions? Explain your reasoning.
\begin{parts}
\item $\frac{\bar{x}}{median} = 1$
\item $\frac{\bar{x}}{median} < 1$
\item $\frac{\bar{x}}{median} > 1$
\end{parts}
}{}

% 53 - oscar_winners

\eoce{\qt{Oscar winners\label{oscar_winners}} The first Oscar awards for best actor 
and best actress were given out in 1929. The histograms below show the age 
distribution for all of the best actor and best actress winners from 1929 to 
2018. Summary statistics for these distributions are also provided. Compare the 
distributions of ages of best actor and actress winners.\footfullcite{data:oscars} \\
\begin{minipage}[c]{0.72\textwidth}
\begin{center}
\tabspecial{tableau-oscar-winner}{
\FigureFullPath[Two histograms are shown, one for "Best Actress" and a second for "Best Actor", where values for the histogram range from 15 to 85. The heights of the bins or the Best Actress histogram are as follows: the bin of 15 to 25 has a height of 9, the 25 to 35 bin has a height of 50, 35 to 45 a height of 19, 45 to 55 a height of 6, 55 to 65 a height of 8, 65 to 75 a height of 1, and 75 to 85 a height of 1. The heights of the bins or the Best Actress histogram are as follows: the bin of 15 to 25 has a height of 0, the 25 to 35 bin has a height of 14, 35 to 45 a height of 45, 45 to 55 a height of 23, 55 to 65 a height of 11, 65 to 75 a height of 0, and 75 to 85 a height of 1.]{0.95}{ch_one_variable_data_collecting_data/figures/eoce/oscar_winners/oscars_winners_hist}}
\end{center}
\end{minipage}
\begin{minipage}[c]{0.27\textwidth}
{\small
\begin{tabular}{l c}
\hline
        & Best Actress  \\
\hline
Mean    & 36.2      \\
SD      & 11.9      \\
n       & 92        \\  
        & \\
        & \\
        & \\
        & \\
        & \\
\hline
        & Best Actor \\
\hline
Mean    & 43.8 \\
SD      & 8.83 \\
n       & 92
\end{tabular}
}
\end{minipage}
}{}

% 54 - dist_shape_exam_scores

\eoce{\qt{Exam scores\label{dist_shape_exam_scores}} The average on a history exam 
(scored out of 100 points) was 85, with a standard deviation of 15. Is the 
distribution of the scores on this exam symmetric? If not, what shape would 
you expect this distribution to have? Explain your reasoning.
}{}

% 55 - stats_scores_box

\eoce{\qt{Stats scores\label{stats_scores_box}} Below are the final exam scores of twenty 
introductory statistics students.
\begin{center}
57, 66, 69, 71, 72, 73, 74, 77, 78, 78, 79, 79, 81, 81, 82, 83, 83, 88, 89, 94
\end{center}
Create a box plot of the distribution of these scores. The five number summary provided below may be useful.
\begin{center}
\renewcommand\arraystretch{1.5}
\begin{tabular}{ccccc}
Min & Q1    & Q2 (Median)   & Q3    & Max \\
\hline
57  & 72.5  & 78.5          & 82.5  & 94 \\
\end{tabular}
\end{center}
}{}

% 56 - marathon_winners

\eoce{\qt{Marathon winners\label{marathon_winners}} The histogram and box plots below show the distribution of finishing times for male and female winners of the New York Marathon between 1970 and 1999.
\begin{center}
\FigureFullPath[Two plots are shown, one that is a histogram and one that is a box plot, where the range of data for each is from 2.0 to 3.2. The bins for the histogram are as follows: the 2.0 to 2.2 bin has a height of 21, bin 2.2 to 2.4 a height of 6, 2.4 to 2.6 a height of 25, 2.6 to 2.8 a height of 3, 2.8 to 3.0 a height of 2, and 3.0 to 3.2 a height of 2. The box plot shows the box spanning 2.2 to 2.5, with the median line centered at 2.4. The whiskers extend from about 2.15 to 2.75. There are four points marked beyond the upper whisker at 2.9, 3.0, 3.10, and 3.15.]{0.56}{ch_one_variable_data_collecting_data/figures/eoce/marathon_winners/marathon_winners_hist_box}
\end{center}
\begin{parts}
\item What features of the distribution are apparent in the histogram and not the box plot? What features are apparent in the box plot but not in the histogram?
\item What may be the reason for the bimodal distribution? Explain.
\item Compare the distribution of marathon times for men and women based on the box plot shown below.
\begin{center}
\FigureFullPath[A side-by-side box plot is shown for marathon run times, one box plot for men and one for women. The axis for the run times spans from 2.0 to 3.2. All values described as follows are estimates. For the men box plot, the box spans 2.16 to 2.22 with the median line at 2.19. The whiskers span to 2.12 up to 2.27. There are 6 points above the upper whisker at 2.32, 2.36, 2.38, 2.44, 2.46, and 2.50. For the women box plot, the box spans from 2.44 to 2.52, with a median value of 2.46. The whiskers span from 2.41 to 2.57. There are 6 points above the upper whisker: 2.72, 2.78, 2.9, 2.92, 3.12, and 3.15.]{0.56}{ch_one_variable_data_collecting_data/figures/eoce/marathon_winners/marathon_winners_gender_box}
\end{center}
\item The time series plot shown below is another way to look at these data. Describe what is visible in this plot but not in the others.
\end{parts}
\begin{center}
\FigureFullPath[A time series plot is shown, which in this case gives the appearance of a scatterplot. The horizontal variable is for year, which runs from 1970 to 2000, and the vertical variable is "Marathon times", which runs from 2.0 to 3.2 hours. There are two colors of points, one for men and one for women, and there is one point for men and one for women for each year. The points start at about 2.5 for men in 1970 and 2.9 for women in 1971. The points bounce around for a few years and then decline in 1975 or 1976 to 2.2 for men and 2.7 for women. The values for women decreases for a few more years to about 2.5. For the remainder of the years, the values fluctuate up or down 0.1 hours from year to year but are stable until 1999, which is the last data points provided.]{0.6}{ch_one_variable_data_collecting_data/figures/eoce/marathon_winners/marathon_winners_time_series} \\
\end{center}
}{}

% 57 - chia_weight_loss

\eoce{\qt{Chia seeds and weight loss\label{chia_weight_lostt}} Chia Pets -- those terra-cotta 
figurines that sprout fuzzy green hair -- made the chia plant a household name. But chia 
has gained an entirely new reputation as a diet supplement.  In one 2009 study, a team 
of researchers recruited 38 men and divided them randomly into two groups: treatment or 
control. They also recruited 38 women, and they randomly placed half of these 
participants into the treatment group and the other half into the control group. One 
group was given 25 grams of chia seeds twice a day, and the other was given a placebo. 
The subjects volunteered to be a part of the study. After 12 weeks, the scientists found 
no significant difference between the groups in appetite or weight loss. 
\footfullcite{Nieman:2009}
\begin{parts}
\item What type of study is this? 
\item What are the experimental and control treatments in this study?
\item Has blocking been used in this study? If so, what is the blocking variable?
\item Has blinding been used in this study?
\item Comment on whether or not we can make a causal statement, and indicate whether or 
not we can generalize the conclusion to the population at large.
\end{parts}
}{}

% 58 - city_council_survey

\eoce{\qt{City council survey\label{city_council_survey}}
A city council has requested a household survey be conducted
in a suburban area of their city.
The area is broken into many distinct and unique neighborhoods,
some including large homes, some with only apartments, and others
a diverse mixture of housing structures.
For each part below,
identify the sampling methods described,
and describe the statistical pros and cons of the method
in the city's context.
\begin{parts}
\item
    Randomly sample 200 households from the city.
\item
    Divide the city into 20 neighborhoods,
    and sample 10 households from each neighborhood.
\item
    Divide the city into 20 neighborhoods,
    randomly sample 3 neighborhoods,
    and then sample all households from those 3 neighborhoods.
\item
    Divide the city into 20 neighborhoods,
    randomly sample 8 neighborhoods,
    and then randomly sample 50 households
    from those neighborhoods.
\item
    Sample the 200 households closest to the city council offices.
\end{parts}
}{}

% 59 - flawed_reasoning

\eoce{\qt{Flawed reasoning\label{flawed_reasoning}} Identify the flaw(s) in reasoning 
in the following scenarios. Explain what the individuals in the study should 
have done differently if they wanted to make such strong conclusions.
\begin{parts}
\item Students at an elementary school are given a questionnaire that they 
are asked to return after their parents have completed it. One of the questions 
asked is, ``Do you find that your work schedule makes it difficult for you to 
spend time with your kids after school?" Of the parents who replied, 85\% said 
``no". Based on these results, the school officials conclude that a great 
majority of the parents have no difficulty spending time with their kids 
after school.
\item A survey is conducted on a simple random sample of 1,000 women who 
recently gave birth, asking them about whether or not they smoked during 
pregnancy. A follow-up survey asking if the children have respiratory problems 
is conducted 3 years later. However, only 567 of these women are reached at the 
same address. The researcher reports that these 567 women are representative 
of all mothers.
\item An orthopedist administers a questionnaire to 30 of his patients who do 
not have any joint problems and finds that 20 of them regularly go running. 
He concludes that running decreases the risk of joint problems.
\end{parts}
}{}

% 60 - stressed_out_experiment

\eoce{\qt{Stressed out, Part II\label{stressed_out_experiment}} In a study evaluating the 
relationship between stress and muscle cramps, half the subjects are randomly assigned to be exposed to increased stress by being placed into an elevator that falls rapidly and stops abruptly and the other half are left at no or baseline stress.
\begin{parts}
\item What type of study is this?
\item Can this study be used to conclude a causal relationship between increased stress 
and muscle cramps?
\end{parts}
}{}

% 61 - eat_better_feel_better

\eoce{\qt[?]{Eat better, feel better\label{eat_better_feel_better}}
In a public health 
study on the effects of consumption of fruits and vegetables on psychological 
well-being in young adults, participants were randomly assigned to three 
groups: (1) diet-as-usual, (2) an ecological momentary intervention involving 
text message reminders to increase their fruits and vegetable consumption plus 
a voucher to purchase them, or (3) a fruit and vegetable intervention in 
which participants were given two additional daily servings of fresh fruits and 
vegetables to consume on top of their normal diet. Participants were asked to 
take a nightly survey on their smartphones.
Participants were student volunteers at the University of 
Otago, New Zealand.
At the end of the 14-day study, only participants in the third
group showed improvements to their psychological well-being across
the 14-days relative to the other groups.\footfullcite{conner2017let}
\begin{parts}
\item
    What type of study is this?
\item
    Identify the explanatory and response variables.
\item
    Comment on whether the results of the study can be generalized to
    the population.
\item
    Comment on whether the results of the study can be used to establish
    causal relationships.
\item
    A newspaper article reporting on the study states,
    ``The results of this study provide proof that giving young adults
    fresh fruits and vegetables to eat can have psychological benefits,
    even over a brief period of time.''
    How would you suggest revising this statement so that it can be
    supported by the study?
\end{parts}
}{}

\D{\newpage}

% 62 - screen_time_well_being

\eoce{\qt{Screens, teens, and psychological well-being\label{screen_time_well_being}}
In a study of three nationally representative large-scale data sets from Ireland, 
the United States, and the United Kingdom (n = 17,247), teenagers between the 
ages of 12 to 15 were asked to keep a diary of their screen time and answer questions about how they felt or acted.
The answers to these questions 
were then used to compute a psychological well-being score.
Additional data were collected and included in the analysis,
such as each child's sex and age, and on the mother's education,
ethnicity, psychological distress, and employment.
The study concluded that there is little clear-cut evidence
that screen time decreases adolescent
well-being.\footfullcite{orben2018screens}
\begin{parts}
\item
    What type of study is this?
\item
    Identify the explanatory variables.
\item
    Identify the response variable.
\item
    Comment on whether the results of the study can be generalized
    to the population, and why.
\item
    Comment on whether the results of the study can be used
    to establish causal relationships.
\end{parts}
}{}

% 63 - stanford_open_policing

\eoce{\qt{Stanford Open Policing\label{stanford_open_policing}}
The Stanford Open Policing project gathers, analyzes, and
releases records from traffic stops by law enforcement 
agencies across the United States.
Their goal is to help researchers, journalists, and policymakers
investigate and improve interactions between police and the
public.\footfullcite{pierson2017large}
The following is an excerpt from a summary table created based off of the data 
collected as part of this project.
\begin{center}
\begin{tabular}{lllrrr}
\hline
               &           & Driver's  & No. of stops & \multicolumn{2}{c}{\% of stopped}  \\
County         & State     & race      & per year     & cars searched & drivers arrested \\ 
\hline
Apaice County  & Arizona   & Black     & 266          & 0.08          & 0.02 \\ 
Apaice County  & Arizona   & Hispanic  & 1008         & 0.05          & 0.02 \\ 
Apaice County  & Arizona   & White     & 6322         & 0.02          & 0.01 \\ 
Cochise County & Arizona   & Black     & 1169         & 0.05          & 0.01 \\ 
Cochise County & Arizona   & Hispanic  & 9453         & 0.04          & 0.01 \\ 
Cochise County & Arizona   & White     & 10826        & 0.02          & 0.01 \\ 
$\cdots$       & $\cdots$  & $\cdots$  & $\cdots$     & $\cdots$      & $\cdots$ \\
Wood County    & Wisconsin & Black     & 16           & 0.24          & 0.10 \\ 
Wood County    & Wisconsin & Hispanic  & 27           & 0.04          & 0.03 \\ 
Wood County    & Wisconsin & White     & 1157         & 0.03          & 0.03 \\ 
\hline 
\end{tabular}
\end{center}
\begin{parts}
\item
    What variables were collected on each individual traffic stop
    in order to create to the summary table above?
\item
    State whether each variable is numerical or categorical.
    If numerical, state whether it is continuous or discrete.
    If categorical, state whether it is ordinal or not.
\item
    Suppose we wanted to evaluate whether vehicle search rates
    are different for drivers of different races.
    In this analysis, which variable would be the response
    variable and which variable would be the explanatory variable?
\end{parts}
}{}

% 64 - space_launches

\eoce{\qt{Space launches\label{space_launches}}
The following summary table shows the number of space
launches in the US by the type of launching agency and
the outcome of the launch (success or
failure).\footfullcite{data:spacelaunches}
\begin{center}
\begin{tabular}{l | rr | rr}
\hline
        & \multicolumn{2}{| c}{1957 - 1999} & \multicolumn{2}{| c}{2000 - 2018} \\
        & Failure & Success & Failure & Success \\ 
\hline
Private &      13 &     295 &      10 &     562 \\ 
State   &     281 &    3751 &      33 &     711 \\ 
Startup &       - &       - &       5 &      65 \\
\hline
\end{tabular}
\end{center}
\begin{parts}
\item
    What variables were collected on each launch in order
    to create to the summary table above?
\item
    State whether each variable is numerical or categorical.
    If numerical, state whether it is continuous or discrete.
    If categorical, state whether it is ordinal or not.
\item
    Suppose we wanted to study how the success rate of
    launches vary between launching agencies and over time.
    In this analysis, which variable would be the response
    variable and which variable would be the explanatory
    variable?
\end{parts}
}{}
