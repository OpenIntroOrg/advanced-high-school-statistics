\exercisesheader{}

% 13 - smoking

\eoce{\qt{Smoking habits of UK residents, Part I\label{UKSmoking_amounts}} A survey was conducted to study the smoking habits of UK residents. The histograms below display the distributions of the number of cigarettes smoked on weekdays and weekends, and they exclude data from people who identified themselves as non-smokers. Describe the two distributions and compare them. \footfullcite{data:smoking}
\begin{center}
\includegraphics[width = 0.7\textwidth]{ch_one_variable_data_collecting_data/figures/eoce/smoking/smoking_amountHist}
\end{center}
}{}

% 14 - stats_final_scores

\eoce{\qt{Stats scores, Part I\label{stats_final_scores}} Below are the final exam scores of twenty introductory statistics students.
\begin{center}
79, 83, 57, 82, 94, 83, 72, 74, 73, 71, 66, 89, 78, 81, 78, 81, 88, 69, 77, 79
\end{center}
Draw a histogram of these data and describe the distribution.
}{}

% 15 - UKSmoking_amounts_data

\eoce{\qt{Smoking habits of UK residents, Part II} \label{UKSmoking_amounts_data} A random sample of 5 smokers from the data set discussed in Exercise~\ref{UKSmoking_amounts} is provided below.
{\footnotesize
\begin{center}
\begin{tabular}{ccccccc}
  \hline
gender & age & maritalStatus & grossIncome & smoke & amtWeekends & amtWeekdays \\ 
  \hline
Female &  51 & Married & $\pounds$2,600 to $\pounds$5,200 & Yes &  20 cig/day &  20 cig/day\\ 
  Male &  24 & Single & $\pounds$10,400 to $\pounds$15,600 & Yes &  20 cig/day&  15 cig/day\\ 
  Female &  33 & Married & $\pounds$10,400 to $\pounds$15,600 & Yes &  20 cig/day&  10 cig/day\\ 
  Female &  17 & Single & $\pounds$5,200 to $\pounds$10,400 & Yes &  20 cig/day&  15 cig/day\\ 
  Female &  76 & Widowed & $\pounds$5,200 to $\pounds$10,400 & Yes &  20 cig/day&  20 cig/day\\ 
   \hline
\end{tabular}
\end{center}
}
\begin{parts}
\item Find the mean amount of cigarettes smoked on weekdays and weekends by these 5 respondents.
\item Find the standard deviation of the amount of cigarettes smoked on weekdays and on weekends by these 5 respondents. Is the variability higher on weekends or on weekdays?
\end{parts}
}{}

% 16 - factory_defective_rate

\eoce{\qt{Factory defective rate} A factory quality control manager decides to investigate the percentage of defective items produced each day. Within a given work week (Monday through Friday) the percentage of defective items produced was 2\%, 1.4\%, 4\%, 3\%, 2.2\%.
\begin{parts}
\item Calculate the mean for these data.
\item Calculate the standard deviation for these data, showing each step in detail.
\end{parts}
}{}

% 17 - days_off_mining

\eoce{\qt{Days off at a mining plant\label{days_off_mining}} Workers at a particular mining 
site receive an average of 35 days paid vacation, which is lower than the national 
average. The manager of this plant is under pressure from a local union to increase the 
amount of paid time off. However, he does not want to give more days off to the workers 
because that would be costly. Instead he decides he should fire 10 employees in such a 
way as to raise the average number of days off that are reported by his employees. In 
order to achieve this goal, should he fire employees who have the most number of days 
off, least number of days off, or those who have about the average number of days off?
}{}

% 18 - medians_iqrs

\eoce{\qt{Medians and IQRs} For each part, compare distributions (1) and (2) based on their medians and IQRs. You do not need to calculate these statistics; simply state how the medians and IQRs compare. Make sure to explain your reasoning. 
\begin{multicols}{2}
\begin{parts}
\item (1) 3, 5, 6, 7, 9 \\
(2) 3, 5, 6, 7, 20
\item (1) 3, 5, 6, 7, 9 \\
(2) 3, 5, 7, 8, 9
\item (1) 1, 2, 3, 4, 5 \\
(2) 6, 7, 8, 9, 10
\item (1) 0, 10, 50, 60, 100 \\
(2) 0, 100, 500, 600, 1000
\end{parts}
\end{multicols}
}{}

% 19 - means_sds

\eoce{\qt{Means and SDs} For each part, compare distributions (1) and (2) based on their means and standard deviations. You do not need to calculate these statistics; simply state how the means and the standard deviations compare. Make sure to explain your reasoning. \textit{Hint:} It may be useful to sketch dot plots of the distributions.
\begin{multicols}{2}
\begin{parts}
\item (1) 3, 5, 5, 5, 8, 11, 11, 11, 13 \\
(2) 3, 5, 5, 5, 8, 11, 11, 11, 20 \\
\item (1) -20, 0, 0, 0, 15, 25, 30, 30 \\
(2) -40, 0, 0, 0, 15, 25, 30, 30
\item (1) 0, 2, 4, 6, 8, 10 \\
(2) 20, 22, 24, 26, 28, 30
\item (1) 100, 200, 300, 400, 500 \\
(2) 0, 50, 300, 550, 600
\end{parts}
\end{multicols}
}{}

% 20 - hist_box_match

\eoce{\qt{Mix-and-match} Describe the distribution in the histograms below and match them to the box plots. \\
\begin{center}
\FigureFullPath[Six plots are shown, three histograms labeled a, b, and c, and 3 box plots labeled 1, 2, and 3. Plot (a) shows a histogram with horizontal range for the data of 50 to 70. The data are bell-shaped and centered in the plot, with only a little data reaching close to the lower end of 50 and the upper end of 70. Plot (b) shows another histogram, where the horizontal axis extends from 0 to 100, and the histogram bins are relatively steady in their height in the first bin near zero across the plot to the last bin near 100. Plot (c) is a histogram with a horizontal axis running from 0 to about 7. The first few bins rise quickly to a peak at the horizontal location of 1 and then fall until reaching 2 and then decline much more gradually until about 4, where the bins are near zero and stay near zero for larger values. Plot (1) is a box plot. The vertical axis for the box plot spans from 0 to about 7. The lower whisker is at 0, the box spans about 1 to 2, with the center line for the box plot at about 1.4. The upper whisker extends up to about 3.5, and then there are several points marked individually extending further upwards to about 7. Plot (2) is a box plot with a vertical axis spanning about 50 to 70. The box for the plot is centered at 60 and runs from about 58 to 62. The whiskers span about 52 to 68. There are 2 individually points shown below 52 and about 4 points shown above 68. Plot (3) is a box plot spanning from 0 to 100. The box is centered at about 50, and the box spans about 25 to 75. The whiskers extend down to 0 and up to 100.]{}{ch_one_variable_data_collecting_data/figures/eoce/hist_box_match/hist_box_match}
\end{center}
}{}

% 21 - air_quality_durham

\eoce{\qt{Air quality\label{air_quality_durham}} Daily air quality is measured by the air 
quality index (AQI) reported by the Environmental Protection Agency. This index reports 
the pollution level and what associated health effects might be a concern. The index is 
calculated for five major air pollutants regulated by the Clean Air Act and takes values 
from 0 to 300, where a higher value indicates lower air quality. AQI was reported for a 
sample of 91 days in 2011 in Durham, NC. The relative frequency histogram below shows 
the distribution of the AQI values on these days. \footfullcite{data:durhamAQI:2011} \\
\begin{minipage}[c]{0.55\textwidth}
\begin{parts}
\item Estimate the median AQI value of this sample.
\item Would you expect the mean AQI value of this sample to be higher or lower than the 
median? Explain your reasoning.
\item Estimate Q1, Q3, and IQR for the distribution.
\item Would any of the days in this sample be considered to have an unusually low or 
high AQI? Explain your reasoning.
\end{parts}
\end{minipage}
\begin{minipage}[c]{0.45\textwidth}
\begin{center}
\FigureFullPath[A histogram of "Daily AQI", where the horizontal axis for the data runs from about 5 to 65. The bin width is 5, there are 12 bins from 5 to 60, and the vertical axis shows proportions. The heights of the 12 bins, in order from left to right, are about 0.02 (for the bin 5 to 10), 0.06, 0.20, 0.06, 0.20, 0.15, 0.07, 0.04, 0.07, 0.08, 0.03, and 0.02 for the last bin for 60 to 65.]{}{ch_one_variable_data_collecting_data/figures/eoce/air_quality_durham/air_quality_durham_rel_freq_hist} 
\end{center}
\end{minipage}
}{}

% 22 - estimate_mean_median_simple

\eoce{\qt{Median vs. mean\label{estimate_mean_median_simple}} Estimate the median for the 
400 observations shown in the histogram, and note whether you expect the mean 
to be higher or lower than the median.
\begin{center}
\FigureFullPath[A histogram is shown, with the horizontal axis for the data runs from 40 to 100, with a bin size width of 5. The frequencies for the bins are as follows, where counts are approximate: 2 (for bin 40 to 45), 4, 2, 10, 20, 25, 50, 75, 70, 85, 45, and 10 for the last bin from 95 to 100.]{0.6}{ch_one_variable_data_collecting_data/figures/eoce/estimate_mean_median_simple/estimate_mean_median_simple} 
\end{center}
}{}

% 23 - hist_vs_box

\eoce{\qt{Histograms vs. box plots\label{hist_vs_box}} Compare the two plots below. What 
characteristics of the distribution are apparent in the histogram and not in the box 
plot? What characteristics are apparent in the box plot but not in the histogram?
\begin{center}
\FigureFullPath[Two plots are shown, first a histogram and second a box plot. The data for each plot runs from about 0 to 30. The histogram has bins of width 2. The bins, starting at the lower values, shows an initial peak at about the horizontal location of 5, then declines to near the horizontal axis at 10, before rising again between 10 and 14, and then lower values again for bins between 15 to 30. The box plot has its box centered at 10 and runs from about 5 to 12. The whiskers reach out to about 2 and up to about 22. There are a few points above the upper whisker.]{0.6}{ch_one_variable_data_collecting_data/figures/eoce/hist_vs_box/hist_vs_box}
\end{center}
}{}

% 24 - dist_shape_fb_friends

\eoce{\qt{Facebook friends\label{dist_shape_fb_friends}} Facebook data indicate that 
50\% of Facebook users have 100 or more friends, and that the average friend 
count of users is 190. What do these findings suggest about the shape of the 
distribution of number of friends of Facebook users? \footfullcite{Backstrom:2011}
}{}

% 25 - dist_shape_pets_dist_height

\eoce{\qt{Distributions and appropriate statistics, Part I\label{dist_shape_pets_dist_height}} 
For each of the following, state whether you expect the distribution to be 
symmetric, right skewed, or left skewed. Also specify whether the mean or 
median would best represent a typical observation in the data, and whether 
the variability of observations would be best represented using the 
standard deviation or IQR. Explain your reasoning.
\begin{parts}
\item Number of pets per household. 
\item Distance to work, i.e. number of miles between work and home.
\item Heights of adult males.
\end{parts}
}{}

% 26 - dist_shape_housing_alcohol_salary

\eoce{\qt{Distributions and appropriate statistics, Part II\label{dist_shape_housing_alcohol_salary}} 
For each of the following, state whether you expect the distribution to be symmetric, 
right skewed, or left skewed. Also specify whether the mean or median would best 
represent a typical observation in the data, and whether the variability of observations 
would be best represented using the standard deviation or IQR. Explain your reasoning.
\begin{parts}
\item Housing prices in a country where 25\% of the houses cost below \$350,000, 
50\% of the houses cost below \$450,000, 75\% of the houses cost below \$1,000,000 
and there are a meaningful number of houses that cost more than \$6,000,000.
\item Housing prices in a country where 25\% of the houses cost below \$300,000, 
50\% of the houses cost below \$600,000, 75\% of the houses cost below \$900,000 
and very few houses that cost more than \$1,200,000.
\item Number of alcoholic drinks consumed by college students in a given week. 
Assume that most of these students don't drink since they are under 21 years old, 
and only a few drink excessively.
\item Annual salaries of the employees at a Fortune 500 company where only a few 
high level executives earn much higher salaries than all the other employees.
\end{parts}
}{}

% 27 - income_coffee_shop

\eoce{\qt{Income at the coffee shop\label{income_coffee_shop}} The first histogram 
below shows the distribution of the yearly incomes of 40 patrons at a college 
coffee shop. Suppose two new people walk into the coffee shop: one making 
\$225,000 and the other \$250,000. The second histogram shows the new income 
distribution. Summary statistics are also provided. \\
\begin{minipage}[c]{0.57\textwidth}
\FigureFullPath[Two histograms are shown and are labeled 1 and 2. Plot 1 has a horizontal axis from \$60,000 to \$70,000. The bins, from left to right, generally rise steadily from frequencies of 2 to 3 at \$60,000 to \$62,000 and up to a peak of about 7 to 8 between \$64,000 to \$66,000. From here, the bin counts steadily decline down to about 2 for the last bin, \$69,000 to \$70,000. Plot (2) shows a histogram, with the horizontal axis running from about \$60,000 to \$260,000. The width of the bins are \$1,000, like in the first plot, and the first 10 bins reflect those described in Plot (1). Two additional bins are shown at about \$225,000 and \$250,000, each with a bin height of 1.]{}{ch_one_variable_data_collecting_data/figures/eoce/income_coffee_shop/income_coffee_shop}
\end{minipage}
\begin{minipage}[c]{0.4\textwidth}
\begin{center}
\begin{tabular}{rrr}
\hline
        & (1)       & (2) \\ 
\hline
n       & 40        & 42 \\ 
Min.    & 60,680    & 60,680 \\ 
1st Qu. & 63,620    & 63,710 \\ 
Median  & 65,240    & 65,350 \\ 
Mean    & 65,090    & 73,300 \\ 
3rd Qu. & 66,160    & 66,540 \\ 
Max.    & 69,890    & 250,000 \\ 
SD      & 2,122     & 37,321 \\ 
\hline
\end{tabular}
\end{center}
\end{minipage}
\begin{parts}
\item Would the mean or the median best represent what we might think of as a 
typical income for the 42 patrons at this coffee shop? What does this say about 
the robustness of the two measures?
\item Would the standard deviation or the IQR best represent the amount of 
variability in the incomes of the 42 patrons at this coffee shop? What does 
this say about the robustness of the two measures?
\end{parts}
}{}

% 28 - midrange

\eoce{\qt{Midrange\label{midrange}} The \textit{midrange} of a distribution is defined as 
the average of the maximum and the minimum of that distribution. Is this statistic 
robust to outliers and extreme skew? Explain your reasoning
}{}

% 29 - county_commute_times_ap

\eoce{\qt{Commute times\label{county_commute_times_ap}} The US census collects data on 
time it takes Americans to commute to work, among many other variables. The 
histogram below shows the distribution of average commute times in 3,144 US 
counties in 2023.  Describe the numerical distribution for commute times. 
\begin{center}
\includegraphics[width=0.48\textwidth]{ch_one_variable_data_collecting_data/figures/eoce/county_commute_times_ap/county_commute_times_hist.pdf}
\end{center}
}{}

% 30 - county_hispanic_latine_pop_skew

\eoce{\qt{Hispanic/Latine population\label{county_hispanic_latine_pop}} The US census collects 
data on race and ethnicity of Americans, among many other variables. The 
histogram below shows the distribution of the percentage of the population 
that is Hispanic/Latine in 3,144 counties in the US in 2023.  
\begin{parts}
\item Describe the shape of the distribution.
\item Is the mean of this distribution greater than or less than the median?  
\item Provide a range for the median percentage of the population 
that is Hispanic/Latine for counties in the US.
\end{parts}
\begin{center}
\includegraphics[width=0.45\textwidth]{ch_one_variable_data_collecting_data/figures/eoce/county_hispanic_latine_pop_skew/county_hispanic_pop_hist.pdf}
\end{center}


}{}
