\exercisesheader{}

% 9 - antibiotic_use_children

\eoce{\qt{Antibiotic use in children\label{antibiotic_use_children}} The bar plot 
and the pie chart below show the distribution of pre-existing medical 
conditions of children involved in a study on the optimal duration of 
antibiotic use in treatment of tracheitis, which is an upper respiratory 
infection.
\begin{center}
\FigureFullPath[A bar plot is shown, where values on the axis range of relative frequency from 0 to just over 0.35. The values, in decreasing order and their approximate values, are Prematurity at 0.36, Cardiovascular at 0.17, Respiratory at 0.14, Trauma at 0.11, and Neuromuscular at 0.11, Genetic/metabolic at 0.07, Immunocompromised at 0.02, and Gastrointestinal at 0.02.]{0.45}{ch_one_variable_data_collecting_data/figures/eoce/antibiotic_use_children/antibiotic_use_children_bar}
\FigureFullPath[A pie chart is shown of the same data from a previous chart, which was a bar chart. The Prematurity category appears to represent about a third of the pie chart (though this and other proportions are difficult to estimate accurately). The Cardiovascular group is roughly one-sixth of the total pie. About a quarter of the pie consists of an even split between Respiratory and Trauma. The remaining categories represent just under a quarter of the pie: Neuromascular (about an eighth of the pie), Genetic/metabolic (about one-fifteenth of the pie), and the remainder evenly distributed between Immunocompromised and Gastrointestinal.]{0.45}{ch_one_variable_data_collecting_data/figures/eoce/antibiotic_use_children/antibiotic_use_children_pie}
\end{center}
\begin{parts}
\item What features are apparent in the bar plot but not in the pie chart?
\item What features are apparent in the pie chart but not in the bar plot?
\item Which graph would you prefer to use for displaying these categorical data?
\end{parts}
}{}

% 10 - health_coverage_freqs_freqs_ahss

\eoce{\qt{Health coverage, frequencies\label{health_coverage_freqs_ahss}} The 
Behavioral Risk Factor Surveillance System (BRFSS) is an annual telephone survey 
designed to identify risk factors in the adult population and report emerging 
health trends. The following table summarizes the respondents reported health status: \footfullcite{data:BRFSS2010}
\begin{center}
\begin{tabular}{lr}
\textit{Health Status}&\\
\cline{1-2}
Excellent & 4,657 \\
Very good & 6,972 \\
Good  & 5,675\\
Fair  &  2,019 \\
Poor  & 677 \\
\cline{1-2}
Total& 20,000
\end{tabular}                             
\end{center}
\begin{parts}
\item  Report the relative frequency of respondents with Excellent health status as a ratio, a proportion, and a percent.
\item Report the relative frequency of respondents with Fair or Poor health status as a ratio, a proportion, and a percent.
\item True or False: Most people reported Very good health status.
\end{parts}
}{}
