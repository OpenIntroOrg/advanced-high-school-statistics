\reviewexercisesheader{}

% 25 - seattle_pet_names

\eoce{% Replaces gpa_study_hours
\qt{Pet names\label{seattle_pet_names}}
The city of Seattle, WA has an open data portal that
includes pets registered in the city.
For each registered pet,
we have information on the pet's name and species.
The following visualization plots the proportion of dogs
with a given name versus the proportion of cats with the
same name.
The 20 most common cat and dog names are displayed.
The diagonal line on the plot is the $x = y$ line;
if a name appeared on this line, the name's popularity
would be exactly the same for dogs and cats.

\noindent\begin{minipage}[c]{0.4\textwidth}
\raggedright\begin{parts}
\item
    Are these data collected as part of an experiment
    or an observational study?
\item
    What is the most common dog name? What is the most
    common cat name?
\item
    What names are more common for cats than dogs?
\item
    Is the relationship between the two variables
    positive or negative? 
    What does this mean in context of the data?
\end{parts}\vspace{5mm}
\end{minipage}
\begin{minipage}[c]{0.05\textwidth}
\ 
\end{minipage}
\begin{minipage}[c]{0.53\textwidth}
\begin{center}
\FigureFullPath[A scatterplot is shown, where each point is labeled with a pet name. The horizontal axis represents "Proportion of cats" and runs from 0.002 to 0.010. The vertical axis represents "Proportion of dogs" and runs from 0.002 to 0.010. There is also a diagonal line (y = x), and only two points fall below this line: "Oliver" at about (0.0045, 0.004) and "Lily" at about (0.005, 0.004). There is a slightly positive trend in the data, the most extreme cases (highest proportions for dogs or cats) are "Lucy" at (0.006, 0.0095), "Charlie" at (0.005, 0.009), "Luna" at (0.0065, 0.007), and "Bella" at (0.005, 0.007).]{}{ch_regression_analysis/figures/eoce/seattle_pet_names/seattle_pet_names}
\end{center}
\end{minipage}
}{}

% 26 - trees_volume_height_diameter

\eoce{\qt{Trees\label{trees_volume_height_diameter}} The scatterplots below 
show the relationship between height, diameter, and volume of timber 
in 31 felled black cherry trees. The diameter of the tree is measured 
4.5 feet above the ground.\footfullcite{data:trees}
\begin{center}
\includegraphics[width=0.46\textwidth]{ch_regression_analysis/figures/eoce/trees_volume_height_diameter/trees_volume_height.pdf}
\hspace{0.07\textwidth}%
\includegraphics[width=0.46\textwidth]{ch_regression_analysis/figures/eoce/trees_volume_height_diameter/trees_volume_diameter.pdf}
\end{center}
\begin{parts}
\item Describe the relationship between volume and height of these trees.
\item Describe the relationship between volume and diameter of these trees.
\item Suppose you have height and diameter measurements for another black 
cherry tree. Which of these variables would be preferable to use to predict 
the volume of timber in this tree using a simple linear regression model? 
Explain your reasoning.
\end{parts}
}{}

% 27 - tf_correlation

\eoce{\qt{True / False\label{tf_correlation}}
Determine if the following statements are true or false.
If false, explain why.
\begin{parts}
\item A correlation coefficient of -0.90 indicates a stronger 
linear relationship than a correlation of 0.5.
\item Correlation is a measure of the association between any 
two variables.
\end{parts}
}{}

\D{\newpage}

% 28 - urban_homeowners_cond_ahss

\eoce{\qt{Urban homeowners\label{urban_homeowners_cond_ahss}}
 The scatterplot shows the percent of families who own their 
home vs. the percent of the population living in urban areas.
\footfullcite{data:urbanOwner} There are 51 observations, each 
corresponding to a state and Puerto Rico.  A residual plot is also shown.

\noindent\begin{minipage}[c]{0.45\textwidth}
{\raggedright\begin{parts}
\item For these data, $R^2=0.28$. What is the correlation? How can 
you tell if it is positive or negative?
\item Examine the residual plot. What do you observe? Is a simple 
least squares fit appropriate for these data?
\end{parts}\vspace{15mm}}
\end{minipage}
\begin{minipage}[c]{0.1\textwidth}
$\:$ \\
\end{minipage}
\begin{minipage}[c]{0.43\textwidth}
\begin{center}
\includegraphics[width=\textwidth]{ch_regression_analysis/figures/eoce/urban_homeowners_cond_ahss/urban_homeowners_cond.pdf}
\end{center}
\end{minipage}
}{}

% 29 - starbucks_cals_protein_ap

\eoce{\qt{Nutrition at Starbucks, Part II\label{starbucks_cals_protein}} 
Exercise~\ref{starbucks_cals_carbos_ap} introduced a data set on nutrition 
information on Starbucks food menu items. Based on the scatterplot 
and the residual plot provided, describe the relationship between the 
protein content and calories of these menu items, and determine if a 
simple linear model is appropriate to predict amount of protein from 
the number of calories.
\begin{center}
\includegraphics[width=0.35\textwidth]{ch_regression_analysis/figures/eoce/starbucks_cals_protein_ap/starbucks_cals_protein}
\end{center}
}{}

% 30 - helmet_lunch_ahss

\eoce{\qt{Helmets and lunches\label{helmet_lunch_ahss}}
The scatterplot shows the 
relationship between socioeconomic status measured as the percentage of 
children in a neighborhood receiving reduced-fee lunches at school 
({\tt lunch}) and the percentage of bike riders in the neighborhood 
wearing helmets ({\tt helmet}). The regression equation is given by: $\hat{y} = 55.34 - 0.537x$.

\noindent\begin{minipage}[c]{0.5\textwidth}
{\raggedright\begin{parts}
\item If the $R^2$ for the least-squares regression line for these 
data is $72\%$, what is the correlation between {\tt lunch} 
and {\tt helmet}?
\item Interpret the intercept of the least-squares regression line in 
the context of the application.
\item Interpret the slope of the least-squares regression line in the 
context of the application.
\item What would the value of the residual be for a neighborhood where 
40\% of the children receive reduced-fee lunches and 40\% of the bike 
riders wear helmets? Interpret the meaning of this residual in the context 
of the application.
\end{parts}}
\end{minipage}
\begin{minipage}[c]{0.05\textwidth}
$\:$ \\
\end{minipage}
\begin{minipage}[c]{0.42\textwidth}
\begin{center}
\includegraphics[width=\textwidth]{ch_regression_analysis/figures/eoce/helmet_lunch_ahss/helmet_lunch.pdf} \\
\end{center}
\end{minipage}
}{}

\D{\newpage}

% 31 - match_corr_3

\eoce{\qt{Match the correlation, Part III\label{match_corr_3}} 
Match each correlation to the corresponding scatterplot.

\noindent%
\begin{minipage}[c]{0.17\textwidth}
\begin{parts}
\item $r = -0.72$
\item $r = 0.07$ 
\item $r = 0.86$ 
\item $r = 0.99$
\end{parts}\vspace{3mm}
\end{minipage}%
\begin{minipage}[c]{0.83\textwidth}
\includegraphics[width=0.245\textwidth]{ch_regression_analysis/figures/eoce/match_corr_3/scatter_1}
\includegraphics[width=0.245\textwidth]{ch_regression_analysis/figures/eoce/match_corr_3/scatter_2}
\includegraphics[width=0.245\textwidth]{ch_regression_analysis/figures/eoce/match_corr_3/scatter_3}
\includegraphics[width=0.245\textwidth]{ch_regression_analysis/figures/eoce/match_corr_3/scatter_4}
\end{minipage}
}{}

% 32 - mcu_box_office_us_predict_ahss

\eoce{\qt{MCU, predict US theater sales\label{mcu_box_office_us_predict_ahss}} 
The Marvel Comic Universe movies were an international movie sensation,
containing 23 movies at the time of this writing.
Here we consider a model predicting an MCU film's gross theater sales
in the US based on the first weekend sales performance in the US.
The data are presented below in both a scatterplot and the model in
a~regression table.
Scientific notation is used below, e.g. 42.5e6 corresponds to
$42.5\times 10^6$.
\begin{center}
\begin{tabular}{rrrrr}
  \hline
  & Estimate & Std. Error & t value & Pr($>$$|$t$|$) \\ 
  \hline
  (Intercept) & 42.5e6 & 26.6e6 & 1.60 & 0.1251 \\ 
  opening\us{}weekend\us{}us & 2.4361 & 0.1739 & 14.01 & 0.0000 \\ 
  \hline
\end{tabular}
\end{center}
\begin{minipage}[c]{0.48\textwidth}
\begin{parts}
\item
  Describe the relationship between gross theater sales in the US and
  first weekend sales in the~US.
\item
  Write the equation of the regression line. Interpret the slope
  and intercept in~context.
\item
  The correlation coefficient for gross sales and first weekend sales
  is~0.950.
  Calculate $R^2$ and interpret it in~context.
\item
  Suppose we consider a set of all films ever released.
  Do you think the relationship between opening weekend sales
  and total sales would have as strong of a relationship as
  what we see with the MCU~films?
\end{parts}
\vspace{3mm}
\end{minipage}%
\begin{minipage}[c]{0.5\textwidth}
\hspace{0.1\textwidth}\includegraphics[width=0.9\textwidth]{ch_regression_analysis/figures/eoce/mcu_box_office_us_predict_ahss/mcu_box_office_us_predict} \\[3mm]
\end{minipage}
}{}
