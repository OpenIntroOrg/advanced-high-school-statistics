\exercisesheader{}

% 13 - visualize_residuals

\eoce{\qt{Visualize the residuals\label{visualize_residuals}} 
The scatterplots shown below each have a 
superimposed regression line. If we were to construct a residual plot 
(residuals versus $x$) for each, describe what those plots would look 
like.
\begin{center}
\includegraphics[width=0.42\textwidth]{ch_regression_analysis/figures/eoce/visualize_residuals/visualize_residuals_linear.pdf} 
\includegraphics[width=0.42\textwidth]{ch_regression_analysis/figures/eoce/visualize_residuals/visualize_residuals_fan_back.pdf}
\end{center}
}{}

% 14 - trends_in_residuals

\eoce{\qt{Trends in the residuals\label{trends_in_residuals}} 
Shown below are two plots of residuals 
remaining after fitting a linear model to two different sets of data. 
Describe important features and determine if a linear model would be 
appropriate for these data. Explain your reasoning.
\begin{center}
\includegraphics[width=0.42\textwidth]{ch_regression_analysis/figures/eoce/trends_in_residuals/trends_in_residuals_fan.pdf} 
\includegraphics[width=0.42\textwidth]{ch_regression_analysis/figures/eoce/trends_in_residuals/trends_in_residuals_log.pdf}
\end{center}
}{}

% 15 - regression_units

\eoce{\qt{Units of regression\label{regression_units}} Consider a regression 
predicting weight (kg) from height (cm) for a sample of adult males. 
What are the units of the correlation coefficient, the intercept, 
and the slope?
}{}

% 16 - which_higher_scatter_ahss

\eoce{\qtq{Which is higher\label{which_higher_scatter_ahss}} Determine if I or II 
is higher or if they are equal. Explain your reasoning.
\noindent For a regression line, the uncertainty associated with the 
slope estimate, $b$, is higher when
\begin{enumerate}
\item[I.] there is a lot of scatter around the regression line or
\item[II.] there is very little scatter around the regression line
\end{enumerate}
}{}

% 17 - residual_apple_weight

\eoce{\qt{Over-under, Part I\label{residual_apple_weight}} Suppose we fit a 
regression line to predict the shelf life of an apple based on its weight. 
For a particular apple, we predict the shelf life to be 4.6 days. The 
apple's residual is -0.6 days. Did we over or under estimate the 
shelf-life of the apple? Explain your reasoning.
}{}

% 18 - residual_sun_cancer

\eoce{\qt{Over-under, Part II\label{residual_sun_cancer}} Suppose we fit a 
regression line to predict the number of incidents of skin cancer per 
1,000 people from the number of sunny days in a year. For a particular 
year, we predict the incidence of skin cancer to be 1.5 per 1,000 people, 
and the residual for this year is 0.5. Did we over or under estimate 
the incidence of skin cancer? Explain your reasoning.
}{}

\D{\newpage}

% 19 - tourism_spending_reg_conds_ap

\eoce{\qt{Tourism spending\label{tourism_spending_reg_conds_ap}} The Association of 
Turkish Travel Agencies reports the number of foreign tourists 
visiting Turkey and tourist spending by year.
\footfullcite{data:turkeyTourism} Three plots are provided: 
scatterplot showing the relationship between these two variables 
along with the least squares fit, residuals plot, and histogram of 
residuals.
\begin{center}
\includegraphics[width=0.32\textwidth]{ch_regression_analysis/figures/eoce/tourism_spending_reg_conds_ap/tourism_spending_count.pdf}
\includegraphics[width=0.32\textwidth]{ch_regression_analysis/figures/eoce/tourism_spending_reg_conds_ap/tourism_spending_count_residuals.pdf}
\end{center}
\begin{parts}
\item Describe the relationship between number of tourists and spending.
\item What are the explanatory and response variables?
\item Why might we want to fit a regression line to these data?
\item Is a linear model appropriate in this context?
\end{parts}
}{}

% 20 - starbucks_cals_carbos_ap

\eoce{\qt{Nutrition at Starbucks, Part I\label{starbucks_cals_carbos_ap}} 
The scatterplot below shows the relationship between the number of 
calories and amount of carbohydrates (in grams) Starbucks food menu 
items contain.\footfullcite{data:starbucksCals} Since Starbucks only 
lists the number of calories on the display items, we are interested 
in predicting the amount of carbs a menu item has based on its 
calorie content.
\begin{center}
\includegraphics[width=0.32\textwidth]{ch_regression_analysis/figures/eoce/starbucks_cals_carbos_ap/starbucks_cals_carbos.pdf}
\includegraphics[width=0.32\textwidth]{ch_regression_analysis/figures/eoce/starbucks_cals_carbos_ap/starbucks_cals_carbos_residuals.pdf}
\includegraphics[width=0.32\textwidth]{ch_regression_analysis/figures/eoce/starbucks_cals_carbos_ap/starbucks_cals_carbos_residuals_hist.pdf}
\end{center}
\begin{parts}
\item Describe the relationship between number of calories and amount 
of carbohydrates (in grams) that Starbucks food menu items contain.
\item In this scenario, what are the explanatory and response 
variables?
\item Why might we want to fit a regression line to these data?
\item Is a linear model appropriate in this context?
\end{parts}
}{}
