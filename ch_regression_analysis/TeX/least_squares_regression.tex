\exercisesheader{}

% 21 - coast_starlight_reg_ahss

\eoce{\qt{The Coast Starlight, Part II\label{coast_starlight_reg_ahss}} \videosolution{ahss_eoce_sol-coast_starlight_reg} 
Exercise~\ref{coast_starlight_corr_units_ap} introduces data on the Coast Starlight 
Amtrak train that runs from Seattle to Los Angeles. The mean travel 
time from one stop to the next on the Coast Starlight is 129 mins, 
with a standard deviation of 113 minutes. The mean distance traveled 
from one stop to the next is 108 miles with a standard deviation of 
99 miles. The correlation between travel time and distance traveled is 0.636.  A least-squares regression line for predicting travel time from distance traveled is fit to the data and has a slope of 0.726 and a y-intercept of 51.  
\begin{parts}
\item Write the equation of the regression line for predicting travel 
time and define any variables used.
\item Interpret the slope and the intercept in this context.
\item Calculate $R^2$ of the regression line for predicting travel 
time from distance traveled for the Coast Starlight, and interpret 
$R^2$ in the context of the application.
\item The distance between Santa Barbara and Los Angeles is 103 
miles. Use the model to estimate the time it takes for the Starlight 
to travel between these two cities.
\item It actually takes the Coast Starlight about 168 mins to travel 
from Santa Barbara to Los Angeles. Calculate the residual and explain 
the meaning of this residual value.
\item Suppose Amtrak is considering adding a stop to the Coast 
Starlight 500 miles away from Los Angeles. Would it be appropriate to 
use this linear model to predict the travel time from Los Angeles to 
this point? 
\end{parts}
}{}

% 22 - body_measurements_shoulder_height_reg_ahss

\eoce{\qt{Body measurements\label{body_measurements_shoulder_height_reg_ahss}}
Researchers studying anthropometry collected body girth measurements
and skeletal diameter measurements, as well as age, weight, height and gender for 507 physically active
individuals.  The mean shoulder girth is 107.20 cm with a standard deviation of 
10.37 cm. The mean height is 171.14 cm with a standard deviation 
of 9.41 cm. The correlation between height and shoulder girth is 0.67.  A least-squares regression line for predicting height from shoulder~girth is fit to the data and has a slope of 0.604 and a y-intercept of 106.39.  
\begin{parts}
\item Write the equation of the regression line for predicting height and define any variables used.
\item Interpret the slope and the intercept in this context.
\item Calculate $R^2$ of the regression line for predicting height 
from shoulder girth, and interpret it in the context of the 
application.
\item A randomly selected student from your class has a shoulder 
girth of 100 cm. Predict the height of this student using the model.
\item The student from part~(d) is 160 cm tall. Calculate the 
residual, and explain what this residual means.
\item A one year old has a shoulder girth of 56 cm. Would it be 
appropriate to use this linear model to predict the height of this 
child?
\end{parts}
}{}

% 23 - murders_poverty_reg

\eoce{\qt{Murders and poverty, Part I\label{murders_poverty_reg}} \videosolution{ahss_eoce_sol-murders_poverty_reg} The following 
regression output is for predicting annual murders per million from 
percentage living in poverty in a random sample of 20 metropolitan 
areas.\\[2mm]
\begin{minipage}[c]{0.56\textwidth}
{\footnotesize
\begin{tabular}{rrrrr}
    \hline
            & Estimate  & Std. Error    & t value   & Pr($>$$|$t$|$) \\ 
    \hline
(Intercept) & -29.901   & 7.789         & -3.839    & 0.001 \\ 
poverty\%   & 2.559     & 0.390         & 6.562     & 0.000 \\ 
   \hline
\end{tabular}\\
$s = 5.512 \hfill R^2 = 70.52\% \hfill R^2_{adj} = 68.89\%$ 
}
\begin{parts}
\item Write out the linear model.
\item Interpret the intercept.
\item Interpret the slope.
\item Interpret $R^2$.
\item Calculate the correlation coefficient.
\end{parts}
\end{minipage}
\begin{minipage}[c]{0.02\textwidth}
$\:$\\
\end{minipage}
\begin{minipage}[c]{0.39\textwidth}
\includegraphics[width=\textwidth]{ch_regression_analysis/figures/eoce/murders_poverty_reg/murders_poverty.pdf}
\end{minipage}
}{}

\D{\newpage}

% 24 - cat_body_heart_reg

\eoce{\qt{Cats, Part I\label{cat_body_heart_reg}} The following regression output is 
for predicting the heart weight (in g) of cats from their body weight 
(in kg). The coefficients are estimated using a dataset of 144 
domestic cats.\\[2mm]
\begin{minipage}[c]{0.56\textwidth}
{\footnotesize
\begin{tabular}{rrrrr}
    \hline
            & Estimate  & Std. Error    & t value   & Pr($>$$|$t$|$) \\ 
    \hline
(Intercept) & -0.357    & 0.692         & -0.515    & 0.607 \\ 
body wt     & 4.034     & 0.250         & 16.119    & 0.000 \\ 
    \hline
\end{tabular}\\
$s = 1.452 \hfill R^2 = 64.66\% \hfill R^2_{adj} = 64.41\%$ 
}
\begin{parts}
\item Write out the linear model.
\item Interpret the intercept.
\item Interpret the slope.
\item Interpret $R^2$.
\item Calculate the correlation coefficient.
\end{parts}
\end{minipage}
\begin{minipage}[c]{0.02\textwidth}
$\:$\\
\end{minipage}
\begin{minipage}[c]{0.39\textwidth}
\includegraphics[width=\textwidth]{ch_regression_analysis/figures/eoce/cat_body_heart_reg/cat_body_heart.pdf}
\end{minipage}
}{}
