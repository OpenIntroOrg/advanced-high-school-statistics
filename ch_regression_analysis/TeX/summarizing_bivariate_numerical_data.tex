\exercisesheader{}

% 1 - acs_ahss

\eoce{\qt{ACS, Part I\label{acs}} Each year, the US Census Bureau surveys about 3.5 million households with The American Community Survey (ACS). Data collected from the ACS have been crucial in government and policy decisions, helping to determine the allocation of federal and state funds each year. Some of the questions asked on the survey are about their income, age (in years), and gender. The table below contains this information for a random sample of 20 respondents to the 2012 ACS. \footfullcite{data:acs:2012} \label{acs_age_income}
\begin{center}
\begin{minipage}[c]{0.4\textwidth}
{\small
\begin{tabular}{rrrl}
  \hline
 & Income & Age & Gender \\ 
  \hline
1 & 53,000 &  28 & male \\ 
  2 & 1600 &  18 & female \\ 
  3 & 70,000 &  54 & male \\ 
  4 & 12,800 &  22 & male \\ 
  5 & 1,200 &  18 & female \\ 
  6 & 30,000 &  34 & male \\ 
  7 & 4,500 &  21 & male \\ 
  8 & 20,000 &  28 & female \\ 
  9 & 25,000 &  29 & female \\ 
  10 & 42,000 &  33 & male \\ 
   \hline
\end{tabular}
}
\end{minipage}
\begin{minipage}[c]{0.4\textwidth}
{\small
\begin{tabular}{rrrl}
  \hline
 & Income & Age & Gender \\ 
  \hline
  11 & 670 &  34 & female \\ 
  12 & 29,000 &  55 & female \\ 
  13 & 44,000 &  33 & female \\ 
  14 & 48,000 &  41 & male \\ 
  15 & 30,000 &  47 & female \\ 
  16 & 60,000 &  30 & male \\ 
  17 & 108,000 &  61 & male \\ 
  18 & 5,800 &  50 & female \\ 
  19 & 50,000 &  24 & female \\ 
  20 & 11,000 &  19 & male \\ 
   \hline
\end{tabular}
}
\end{minipage}
\end{center}
\begin{parts}
\item Use technology to create a scatterplot of income vs. age, and describe the relationship between these two variables.
\item Now create two scatterplots: one for income vs. age for males and another for females.
\item How, if at all, do the relationships between income and age differ for males and females?
\end{parts}
}{}

% 2 - mlb_ahss

\eoce{\qt{MLB stats\label{mlb}} A baseball team's success in a season is usually measured by their number of wins. In order to win, the team has to have scored more points (runs) than their opponent in any given game. As such, number of runs is often a good proxy for the success of the team. The table below shows number of runs, home runs, and batting averages for a random sample of 10 teams in the 2014 Major League Baseball season. \footfullcite{data:MLB:2014}
\begin{center}
{\small
\begin{tabular}{rlrrr}
  \hline
 & Team & Runs & Home runs & Batting avg. \\ 
  \hline
1 & Baltimore &  705 &  211 & 0.256 \\ 
  2 & Boston &  634 &  123 & 0.244 \\ 
  3 & Cincinnati &  595 &  131 & 0.238 \\ 
  4 & Cleveland &  669 &  142 & 0.253 \\ 
  5 & Detroit &  757 &  155 & 0.277 \\ 
  6 & Houston &  629 &  163 & 0.242 \\ 
  7 & Minnesota &  715 &  128 & 0.254 \\ 
  8 & NY Yankees &  633 &  147 & 0.245 \\ 
  9 & Pittsburgh &  682 &  156 & 0.259 \\ 
  10 & San Francisco &  665 &  132 & 0.255 \\ 
   \hline
\end{tabular}
}
\end{center}
\begin{parts}
\item Use technology to create a scatterplot of runs vs. home runs.
\item Now create a scatterplot of runs vs. batting averages.
\item Are home runs or batting averages more strongly associated with number of runs? Explain your reasoning.
\end{parts}
}{}

% 3 - mammal_life_spans

\eoce{\qt{Mammal life spans\label{mammal_life_spans}} Data were collected on life spans (in 
years) and gestation lengths (in days) for 62 mammals. A scatterplot of life span versus 
length of gestation is shown below. \footfullcite{Allison+Cicchetti:1975}

\noindent\begin{minipage}[c]{0.44\textwidth}
\begin{parts}
\item What type of an association is apparent between life span and length of gestation?
\item What type of an association would you expect to see if the axes of the plot were reversed, i.e. if we plotted length of gestation versus life span?
\item Are life span and length of gestation independent? Explain your reasoning.
\end{parts}
\end{minipage}
\begin{minipage}[c]{0.55\textwidth}
\begin{center}
\FigureFullPath[A scatterplot of 62 points is shown. The variable "Gestation" is shown along the horizontal axis with a range of 0 days to about 650 days. The variable "Life Span" is shown along the vertical axis with a range of 0 years to 100 years. The a large cluster of points is shown between 0 to 250 gestational days and 0 to 30 years. Outside of this cluster, there is one point at approximately (10, 50). There is another cluster of points between 250 and 450 gestational days and 25 and 50 years. Beyond the points so far described are three points located at (250 days, 100 years), (640 days, 70 years), and (650 days, 45 years).]{0.86}{ch_regression_analysis/figures/eoce/mammal_life_spans/mammal_life_spans_scatterplot}
\end{center}
\end{minipage}
}{}

% 4 - association_plots

\eoce{\qt{Associations\label{association_plots}}
Indicate which of the plots show
(a)~a positive association,
(b)~a negative association, or
(c)~no~association.
Also determine if the positive and negative associations
are linear or nonlinear.
Each part may refer to more than one plot.
\begin{center}
\FigureFullPath[Four scatterplots are shown and are labeled 1, 2, 3, and 4. There are no label axes on these plots, as only the patterns of the points in the plots are important for this exercise. In plot 1, the points are moderately clustered in the lower left corner of the plot and remain clustered looking further right in the plot, where the points follow steadily upwards to the top-right corner. In plot 2, the points appear to be scattered almost randomly all around the rectangular plotting region. Plot 3 shows points clustered tightly in the lower left corner and the data points remain clustered even as moving right, with the data trending upwards gradually and then more steeply as it reaches the right side of the plot. Plot 4, when looking on the left portion, shows data moderately clustered in the upper-left corner, which then steadily trends downward to the lower-right corner of the plot.]{0.95}{ch_regression_analysis/figures/eoce/association_plots/association_plots}
\end{center}
}{}

% 5 - exams_grades_correlation

\eoce{\qt{Exams and grades\label{exams_grades_correlation}} 
The two scatterplots below show the 
relationship between final and mid-semester exam grades recorded 
during several years for a Statistics course at a university.
\begin{parts}
\item Based on these graphs, which of the two exams has the strongest 
correlation with the final exam grade? Explain.
\item Can you think of a reason why the correlation between the exam 
you chose in part (a) and the final exam is higher?
\end{parts}
\begin{center}
\includegraphics[width=0.485\textwidth]{ch_regression_analysis/figures/eoce/exams_grades_correlation/exam_grades_1.pdf}
\hspace{0.02\textwidth}%
\includegraphics[width=0.485\textwidth]{ch_regression_analysis/figures/eoce/exams_grades_correlation/exam_grades_2.pdf}
\end{center}
}{}

% 6 - spouses_1_ap

\eoce{\qt{Spouses, Part I\label{spouses_1_ap}}
The Great Britain Office of Population Census and Surveys once 
collected data on a random sample of 170 married women in 
Britain, recording the age (in years) and heights (converted 
here to inches) of the women and their spouses.\footfullcite{Hand:1994} 
The scatterplot on the left shows the spouse's age plotted against the woman's age, and the plot on the right shows spouse's height 
plotted against the woman's height.
\begin{center}
\includegraphics[width=0.36\textwidth]{ch_regression_analysis/figures/eoce/spouses_1_ap/husbands_wives_age} 
\hspace{5mm}
\includegraphics[width=0.36\textwidth]{ch_regression_analysis/figures/eoce/spouses_1_ap/husbands_wives_height}
\end{center}
\begin{parts}
\item Describe the relationship between the ages of women in the sample and their spouses' ages.
\item Describe the relationship between the heights of women in the sample and their spouses' heights.
\item Which plot shows a stronger correlation? Explain your reasoning.
\item Data on heights were originally collected in centimeters, and 
then converted to inches. Does this conversion affect the correlation 
between heights of women in the sample and their spouses' heights?
\end{parts}
}{}

% 7 - match_corr_1

\eoce{\qt{Match the correlation, Part I\label{match_corr_1}} 
Match each correlation to the corresponding scatterplot.

\noindent%
\begin{minipage}[c]{0.17\textwidth}
\begin{parts}
\item $r = -0.7$
\item $r = 0.45$ 
\item $r = 0.06$
\item $r = 0.92$
\end{parts}\vspace{3mm}
\end{minipage}%
\begin{minipage}[c]{0.83\textwidth}
\includegraphics[width=0.245\textwidth]{ch_regression_analysis/figures/eoce/match_corr_1/match_corr_1_u}
\includegraphics[width=0.245\textwidth]{ch_regression_analysis/figures/eoce/match_corr_1/match_corr_2_strong_pos}
\includegraphics[width=0.245\textwidth]{ch_regression_analysis/figures/eoce/match_corr_1/match_corr_3_weak_pos}
\includegraphics[width=0.245\textwidth]{ch_regression_analysis/figures/eoce/match_corr_1/match_corr_4_weak_neg}
\end{minipage}
}{}

% 8 - match_corr_2

\eoce{\qt{Match the correlation, Part II\label{match_corr_2}} 
Match each correlation to the corresponding scatterplot.

\noindent%
\begin{minipage}[c]{0.17\textwidth}
\begin{parts}
\item $r = 0.49$
\item $r = -0.48$ 
\item $r = -0.03$ 
\item $r = -0.85$
\end{parts}\vspace{3mm}
\end{minipage}%
\begin{minipage}[c]{0.83\textwidth}
\includegraphics[width=0.245\textwidth]{ch_regression_analysis/figures/eoce/match_corr_2/match_corr_1_strong_neg_curved}
\includegraphics[width=0.245\textwidth]{ch_regression_analysis/figures/eoce/match_corr_2/match_corr_2_weak_pos}
\includegraphics[width=0.245\textwidth]{ch_regression_analysis/figures/eoce/match_corr_2/match_corr_3_n}
\includegraphics[width=0.245\textwidth]{ch_regression_analysis/figures/eoce/match_corr_2/match_corr_4_weak_neg}
\end{minipage}
}{}

% 9 - speed_height_gender

\eoce{\qt{Speed and height\label{speed_height_gender}} 1,302 UCLA students 
were asked to fill out a survey where they were asked about their height, 
fastest speed they have ever driven, and gender. The scatterplot on the 
left displays the relationship between height and fastest speed, and 
the scatterplot on the right displays the breakdown by gender in 
this relationship.
\begin{center}
\includegraphics[width=0.485\textwidth]{ch_regression_analysis/figures/eoce/speed_height_gender/speed_height.pdf}
\hspace{0.02\textwidth}%
\includegraphics[width=0.485\textwidth]{ch_regression_analysis/figures/eoce/speed_height_gender/speed_height_gender.pdf}
\end{center}
\begin{parts}
\item Describe the relationship between height and fastest speed.
\item Why do you think these variables are positively associated?
\item What role does gender play in the relationship between height 
and fastest driving speed?
\end{parts}
}{}

% 10 - guess_correlation

\eoce{\qt{Guess the correlation\label{guess_correlation}} Eduardo and Rosie 
are both collecting data on number of rainy days in a year and the total 
rainfall for the year. Eduardo records rainfall in inches and Rosie in 
centimeters. How will their correlation coefficients compare?
}{}

% 11 - coast_starlight_corr_units_ap

\eoce{\qt{The Coast Starlight, Part I\label{coast_starlight_corr_units_ap}} 
The Coast Starlight Amtrak train runs from Seattle to Los Angeles. 
The scatterplot below displays the distance between each stop 
(in miles) and the amount of time it takes to travel from one stop 
to another (in minutes).\vspace{2mm}

\noindent\begin{minipage}[c]{0.4\textwidth}
{\raggedright\begin{parts}
\item Describe the relationship between distance and travel time.
\item How would the relationship change if travel time was instead measured 
in hours, and distance was instead measured in kilometers?
\item The correlation between travel time (in minutes) and distance (in miles) 
is $r = 0.636$.  What are the units for the correlation in this context?
\item Suppose we had instead measured travel time in hours
and measured distance in kilometers (km).
What would be the correlation in these different units?
\item How would the correlation change if the $x$ and $y$ variables were swapped?
\end{parts}\vspace{7mm}}
\end{minipage}
\begin{minipage}[c]{0.1\textwidth}
$\:$\\
\end{minipage}
\begin{minipage}[c]{0.485\textwidth}
\includegraphics[width=\textwidth]{ch_regression_analysis/figures/eoce/coast_starlight_corr_units_ap/coast_starlight.pdf}
\end{minipage}
}{}

% 12 - crawling_babies_corr_units_ap

\eoce{\qt{Crawling babies\label{crawling_babies_corr_units_ap}}  
A study conducted at the University of Denver investigated whether babies 
take longer to learn to crawl in cold months, when they are often bundled 
in clothes that restrict their movement, than in warmer months.
\footfullcite{Benson:1993} Infants born during the study year were split 
into twelve groups, one for each birth month. We consider the average 
crawling age of babies in each group against the average temperature when 
the babies are six months old (that's when babies often begin trying to 
crawl). Temperature is measured in degrees Fahrenheit (\degree F) and age 
is measured in weeks.\vspace{2mm}

\noindent\begin{minipage}[c]{0.4\textwidth}
{\raggedright\begin{parts}
\item Describe the relationship between temperature and crawling age.
\item How would the relationship change if temperature was measured in 
degrees Celsius (\degree C) and age was measured in months?
\item What are
\item The correlation between temperature in \degree F and age in weeks 
was $r=-0.70$. What are the units for the correlation in this context?
\item If we converted the temperature to \degree C and age to 
months, what would the correlation be?
\item How would the correlation change if the $x$ and $y$ variables were swapped?
\end{parts}\vspace{3mm}}
\end{minipage}
\begin{minipage}[c]{0.1\textwidth}
$\:$\\
\end{minipage}
\begin{minipage}[c]{0.485\textwidth}
\includegraphics[width=\textwidth]{ch_regression_analysis/figures/eoce/crawling_babies_corr_units_ap/crawling_babies.pdf}
\end{minipage}
}{}
