\exercisesheader{}

% 7 - smoking

\eoce{\qt{Smoking habits of UK residents, Part I\label{UKSmoking_amounts}} A survey was conducted to study the smoking habits of UK residents. The histograms below display the distributions of the number of cigarettes smoked on weekdays and weekends, and they exclude data from people who identified themselves as non-smokers. Describe the two distributions and compare them. \footfullcite{data:smoking}
\begin{center}
\includegraphics[width = 0.7\textwidth]{ch_summarizing_data/figures/eoce/smoking/smoking_amountHist}
\end{center}
}{}

% 8 - stats_final_scores

\eoce{\qt{Stats scores, Part I\label{stats_final_scores}} Below are the final exam scores of twenty introductory statistics students.
\begin{center}
79, 83, 57, 82, 94, 83, 72, 74, 73, 71, 66, 89, 78, 81, 78, 81, 88, 69, 77, 79
\end{center}
Draw a histogram of these data and describe the distribution.
}{}

% 9 - UKSmoking_amounts_data

\eoce{\qt{Smoking habits of UK residents, Part II} \label{UKSmoking_amounts_data} A random sample of 5 smokers from the data set discussed in Exercise~\ref{UKSmoking_amounts} is provided below.
{\footnotesize
\begin{center}
\begin{tabular}{ccccccc}
  \hline
gender & age & maritalStatus & grossIncome & smoke & amtWeekends & amtWeekdays \\ 
  \hline
Female &  51 & Married & $\pounds$2,600 to $\pounds$5,200 & Yes &  20 cig/day &  20 cig/day\\ 
  Male &  24 & Single & $\pounds$10,400 to $\pounds$15,600 & Yes &  20 cig/day&  15 cig/day\\ 
  Female &  33 & Married & $\pounds$10,400 to $\pounds$15,600 & Yes &  20 cig/day&  10 cig/day\\ 
  Female &  17 & Single & $\pounds$5,200 to $\pounds$10,400 & Yes &  20 cig/day&  15 cig/day\\ 
  Female &  76 & Widowed & $\pounds$5,200 to $\pounds$10,400 & Yes &  20 cig/day&  20 cig/day\\ 
   \hline
\end{tabular}
\end{center}
}
\begin{parts}
\item Find the mean amount of cigarettes smoked on weekdays and weekends by these 5 respondents.
\item Find the standard deviation of the amount of cigarettes smoked on weekdays and on weekends by these 5 respondents. Is the variability higher on weekends or on weekdays?
\end{parts}
}{}

% 10 - factory_defective_rate

\eoce{\qt{Factory defective rate} A factory quality control manager decides to investigate the percentage of defective items produced each day. Within a given work week (Monday through Friday) the percentage of defective items produced was 2\%, 1.4\%, 4\%, 3\%, 2.2\%.
\begin{parts}
\item Calculate the mean for these data.
\item Calculate the standard deviation for these data, showing each step in detail.
\end{parts}
}{}

% 11 - days_off_mining

\eoce{\qt{Days off at a mining plant\label{days_off_mining}} Workers at a particular mining 
site receive an average of 35 days paid vacation, which is lower than the national 
average. The manager of this plant is under pressure from a local union to increase the 
amount of paid time off. However, he does not want to give more days off to the workers 
because that would be costly. Instead he decides he should fire 10 employees in such a 
way as to raise the average number of days off that are reported by his employees. In 
order to achieve this goal, should he fire employees who have the most number of days 
off, least number of days off, or those who have about the average number of days off?
}{}

% 12 - medians_iqrs

\eoce{\qt{Medians and IQRs} For each part, compare distributions (1) and (2) based on their medians and IQRs. You do not need to calculate these statistics; simply state how the medians and IQRs compare. Make sure to explain your reasoning. 
\begin{multicols}{2}
\begin{parts}
\item (1) 3, 5, 6, 7, 9 \\
(2) 3, 5, 6, 7, 20
\item (1) 3, 5, 6, 7, 9 \\
(2) 3, 5, 7, 8, 9
\item (1) 1, 2, 3, 4, 5 \\
(2) 6, 7, 8, 9, 10
\item (1) 0, 10, 50, 60, 100 \\
(2) 0, 100, 500, 600, 1000
\end{parts}
\end{multicols}
}{}

% 13 - means_sds

\eoce{\qt{Means and SDs} For each part, compare distributions (1) and (2) based on their means and standard deviations. You do not need to calculate these statistics; simply state how the means and the standard deviations compare. Make sure to explain your reasoning. \textit{Hint:} It may be useful to sketch dot plots of the distributions.
\begin{multicols}{2}
\begin{parts}
\item (1) 3, 5, 5, 5, 8, 11, 11, 11, 13 \\
(2) 3, 5, 5, 5, 8, 11, 11, 11, 20 \\
\item (1) -20, 0, 0, 0, 15, 25, 30, 30 \\
(2) -40, 0, 0, 0, 15, 25, 30, 30
\item (1) 0, 2, 4, 6, 8, 10 \\
(2) 20, 22, 24, 26, 28, 30
\item (1) 100, 200, 300, 400, 500 \\
(2) 0, 50, 300, 550, 600
\end{parts}
\end{multicols}
}{}

% 14 - hist_box_match

\eoce{\qt{Mix-and-match} Describe the distribution in the histograms below and match them to the box plots. \\
\begin{center}
\includegraphics[width=\textwidth]{ch_summarizing_data/figures/eoce/hist_box_match/hist_box_match.pdf}
\end{center}
}{}

% 15 - air_quality_durham

\eoce{\qt{Air quality\label{air_quality_durham}} Daily air quality is measured by the air 
quality index (AQI) reported by the Environmental Protection Agency. This index reports 
the pollution level and what associated health effects might be a concern. The index is 
calculated for five major air pollutants regulated by the Clean Air Act and takes values 
from 0 to 300, where a higher value indicates lower air quality. AQI was reported for a 
sample of 91 days in 2011 in Durham, NC. The relative frequency histogram below shows 
the distribution of the AQI values on these days. \footfullcite{data:durhamAQI:2011} \\
\begin{minipage}[c]{0.55\textwidth}
\begin{parts}
\item Estimate the median AQI value of this sample.
\item Would you expect the mean AQI value of this sample to be higher or lower than the 
median? Explain your reasoning.
\item Estimate Q1, Q3, and IQR for the distribution.
\item Would any of the days in this sample be considered to have an unusually low or 
high AQI? Explain your reasoning.
\end{parts}
\end{minipage}
\begin{minipage}[c]{0.45\textwidth}
\begin{center}
\includegraphics[width = \textwidth]{ch_summarizing_data/figures/eoce/air_quality_durham/air_quality_durham_rel_freq_hist.pdf} 
\end{center}
\end{minipage}
}{}

% 16 - estimate_mean_median_simple

\eoce{\qt{Median vs. mean\label{estimate_mean_median_simple}} Estimate the median for the 
400 observations shown in the histogram, and note whether you expect the mean 
to be higher or lower than the median.
\begin{center}
\includegraphics[width = 0.6\textwidth]{ch_summarizing_data/figures/eoce/estimate_mean_median_simple/estimate_mean_median_simple.pdf} 
\end{center}
}{}

% 17 - hist_vs_box

\eoce{\qt{Histograms vs. box plots\label{hist_vs_box}} Compare the two plots below. What 
characteristics of the distribution are apparent in the histogram and not in the box 
plot? What characteristics are apparent in the box plot but not in the histogram?
\begin{center}
\includegraphics[width = 0.6\textwidth]{ch_summarizing_data/figures/eoce/hist_vs_box/hist_vs_box.pdf}
\end{center}
}{}

% 18 - dist_shape_fb_friends

\eoce{\qt{Facebook friends\label{dist_shape_fb_friends}} Facebook data indicate that 
50\% of Facebook users have 100 or more friends, and that the average friend 
count of users is 190. What do these findings suggest about the shape of the 
distribution of number of friends of Facebook users? \footfullcite{Backstrom:2011}
}{}

% 19 - dist_shape_pets_dist_height

\eoce{\qt{Distributions and appropriate statistics, Part I\label{dist_shape_pets_dist_height}} 
For each of the following, state whether you expect the distribution to be 
symmetric, right skewed, or left skewed. Also specify whether the mean or 
median would best represent a typical observation in the data, and whether 
the variability of observations would be best represented using the 
standard deviation or IQR. Explain your reasoning.
\begin{parts}
\item Number of pets per household. 
\item Distance to work, i.e. number of miles between work and home.
\item Heights of adult males.
\end{parts}
}{}

% 20 - dist_shape_housing_alcohol_salary

\eoce{\qt{Distributions and appropriate statistics, Part II\label{dist_shape_housing_alcohol_salary}} 
For each of the following, state whether you expect the distribution to be symmetric, 
right skewed, or left skewed. Also specify whether the mean or median would best 
represent a typical observation in the data, and whether the variability of observations 
would be best represented using the standard deviation or IQR. Explain your reasoning.
\begin{parts}
\item Housing prices in a country where 25\% of the houses cost below \$350,000, 
50\% of the houses cost below \$450,000, 75\% of the houses cost below \$1,000,000 
and there are a meaningful number of houses that cost more than \$6,000,000.
\item Housing prices in a country where 25\% of the houses cost below \$300,000, 
50\% of the houses cost below \$600,000, 75\% of the houses cost below \$900,000 
and very few houses that cost more than \$1,200,000.
\item Number of alcoholic drinks consumed by college students in a given week. 
Assume that most of these students don't drink since they are under 21 years old, 
and only a few drink excessively.
\item Annual salaries of the employees at a Fortune 500 company where only a few 
high level executives earn much higher salaries than all the other employees.
\end{parts}
}{}

% 21 - income_coffee_shop

\eoce{\qt{Income at the coffee shop\label{income_coffee_shop}} The first histogram 
below shows the distribution of the yearly incomes of 40 patrons at a college 
coffee shop. Suppose two new people walk into the coffee shop: one making 
\$225,000 and the other \$250,000. The second histogram shows the new income 
distribution. Summary statistics are also provided. \\
\begin{minipage}[c]{0.57\textwidth}
\includegraphics[width=\textwidth]{ch_summarizing_data/figures/eoce/income_coffee_shop/income_coffee_shop.pdf}
\end{minipage}
\begin{minipage}[c]{0.4\textwidth}
\begin{center}
\begin{tabular}{rrr}
\hline
        & (1)       & (2) \\ 
\hline
n       & 40        & 42 \\ 
Min.    & 60,680    & 60,680 \\ 
1st Qu. & 63,620    & 63,710 \\ 
Median  & 65,240    & 65,350 \\ 
Mean    & 65,090    & 73,300 \\ 
3rd Qu. & 66,160    & 66,540 \\ 
Max.    & 69,890    & 250,000 \\ 
SD      & 2,122     & 37,321 \\ 
\hline
\end{tabular}
\end{center}
\end{minipage}
\begin{parts}
\item Would the mean or the median best represent what we might think of as a 
typical income for the 42 patrons at this coffee shop? What does this say about 
the robustness of the two measures?
\item Would the standard deviation or the IQR best represent the amount of 
variability in the incomes of the 42 patrons at this coffee shop? What does 
this say about the robustness of the two measures?
\end{parts}
}{}

% 22 - midrange

\eoce{\qt{Midrange\label{midrange}} The \textit{midrange} of a distribution is defined as 
the average of the maximum and the minimum of that distribution. Is this statistic 
robust to outliers and extreme skew? Explain your reasoning
}{}

% 23 - county_commute_times_ap

\eoce{\qt{Commute times\label{county_commute_times_ap}} The US census collects data on 
time it takes Americans to commute to work, among many other variables. The 
histogram below shows the distribution of average commute times in 3,142 US 
counties in 2010. Also shown below is a spatial intensity map of the same data.
\begin{center}\tabspecial{tableau-histogramschoose}{
\includegraphics[width=0.48\textwidth]{ch_summarizing_data/figures/eoce/county_commute_times_ap/county_commute_times_hist.pdf}}
\tabspecial{tableau-intensitymapsall}{\includegraphics[width=0.48\textwidth]{ch_summarizing_data/figures/eoce/county_commute_times_ap/county_commute_times_map.pdf}}
\end{center}
\begin{parts}
\item Describe the numerical distribution for commute times.  
\item Describe the spatial distribution of commuting times using the map provided.
\end{parts} 
}{}

% 24 - county_hispanic_latine_pop

\eoce{\qt{Hispanic/Latine population\label{county_hispanic_latine_pop}} The US census collects 
data on race and ethnicity of Americans, among many other variables. The 
histogram below shows the distribution of the percentage of the population 
that is Hispanic/Latine in 3,142 counties in the US in 2017. \begin{center}
\tabspecial{tableau-histogramschoose}{
\includegraphics[width=0.45\textwidth]{ch_summarizing_data/figures/eoce/county_hispanic_latine_pop/county_hispanic_pop_hist.pdf}}

\tabspecial{tableau-intensitymapsall}{
\includegraphics[width=0.45\textwidth]{ch_summarizing_data/figures/eoce/county_hispanic_latine_pop/county_hispanic_pop_map.pdf}
}
\end{center}
\begin{parts}
\item Describe the distribution of percent of population that is Hispanic/Latine for counties in the US.
\item What features of the distribution of the Hispanic/Latine population in US 
counties are apparent in the map but not in the histogram? What features are 
apparent in the histogram but not the map?
\end{parts}
}{}
