\chapter*{{\color{oiB}Preface}\vspace{-6mm}}

\noindent%
\emph{Advanced High School Statistics} covers a first course in statistics, providing an introduction to applied statistics that is clear, concise, and accessible. This book was written to align with the AP$^{\text{\textregistered}}$ Statistics Course
Description\footnote{AP$^{\text{\textregistered}}$ is
  a trademark registered and owned by the College Board,
  which was not involved in the production of, and does
  not endorse, this product.
  \oiRedirect{collegeboard-ap_statistics_course_description}
      {apcentral.collegeboard.org/pdf/ap-statistics-course-description.pdf}}, but it's also popular in non-AP courses and community colleges.
\vspace{3mm}

\noindent%
This book may be downloaded as a free PDF at \oiRedirect{ahss}{\color{black}\textbf{openintro.org/ahss}}.
\vspace{3mm}

\noindent%
We hope readers will take away three ideas from this book in addition to forming a foundation of statistical thinking and methods.\vspace{-1mm}
\begin{enumerate}
\setlength{\itemsep}{0mm}
\item[(1)] Statistics is an applied field with a wide range of practical applications.
\item[(2)] You don't have to be a math guru to learn from real, interesting data.
\item[(3)] Data are messy, and statistical tools are imperfect. But, when you understand the strengths and weaknesses of these tools, you can use them to learn about the real~world.
\end{enumerate}


\subsection*{{\color{oiB}Textbook overview}}

The chapters of this book are as follows:
\begin{description}
\setlength{\itemsep}{0mm}
\item[1. Data collection.] Data structures, variables, and basic data collection techniques.
\item[2. Summarizing data.] Data summaries and graphics.
\item[3. Probability.] The basic principles of probability.
\item[4. Distributions of random variables.] Introduction to key distributions, and how the normal model applies to the sample mean and sample proportion.
\item[5. Foundations for inference.] General ideas for statistical inference in the context of estimating the population proportion.
\item[6. Inference for categorical data.]
    Inference for proportions and contingency tables
    using the normal and chi-square distributions.
\item[7. Inference for numerical data.]
    Inference for one or two sample means using the
    $t$-distribution.
\item[8. Introduction to linear regression.] An introduction to regression with two variables, and inference on the slope of the regression line.
\end{description}

\subsection*{{\color{oiB}Online resources}}
OpenIntro is focused on increasing access to education by developing free, high-quality education materials. In addition to textbooks, we provide the following accompanying resources to help teachers and students be successful.

\begin{itemize}
\setlength{\itemsep}{0mm}
\item Video overviews for each section of the textbook
\item Lecture slides for each section of the textbook
\item \oiRedirect{casio_all}{Casio} and \oiRedirect{ti_all}{TI} calculator tutorials
\item \oiRedirect{ahss-solutions}{Video solutions} for selected section and chapter exercises
\item Statistical software labs
\item A small but growing number of \oiRedirect{ahss-demos}{Desmos activities}\footnote{\oiRedirect{ahss-desmos}{\url{teacher.desmos.com/collection/5cae8f9c0924751e461dd6f7}}}
\item \oiRedirect{ahss-quizlet}{Quizlet} sets for each chapter\footnote{\oiRedirect{ahss-quizlet}{\url{quizlet.com/openintro-ahss}}} 
\item A \oiRedirect{ahss-tableau}{Tableau public} page to further interact with data sets\footnote{\oiRedirect{ahss-tableau}{\url{public.tableau.com/profile/openintro}}}
\item \oiRedirect{ahss2-online-temp}{Online, interactive version of textbook}\footnote{Developed by Emiliano Vega and Ralf Youtz of Portland Community College using PreTeXt and MathJax.}
\item Complete companion course with the learning management software \oiRedirect{ahss_mom}{{MyOpenMath}}\footnote{\oiRedirect{myopenmath-ahss}{myopenmath.com/course/public.php?cid=11774}}
\item Complete Canvas course accessible through \oiRedirect{ahss_canvas}{{Canvas Commons}}\footnote{\oiRedirect{canvas_course-sfuhs}{sfuhs.instructure.com/courses/1068}}
\end{itemize}

\noindent%
All of these resources can be found at: 
\begin{center}
\oiRedirect{ahss}
    {\color{black}\textbf{openintro.org/ahss}}
\end{center}

\noindent%	
We also have improved the ability to access data in this book
through the addition of Appendix~\ref{appendix_data},
which provides additional information for each of the data sets
used in the main text and is new in the Second Edition.
Online guides to each of these data sets are also provided at
\oiRedirect{data}
    {\color{black}\textbf{openintro.org/data}}
and through a
\oiRedirect{textbook-github_openintro}
    {companion R~package}.

% Official:
% http://www.openintro.org/package/openintro
% Currently redirect it to:
% http://openintrostat.github.io/openintro-r-package/
\vspace{3mm}

\subsection*{{\color{oiB}Examples and exercises}}
%, and appendices}

\noindent%
Many examples are provided to establish an understanding of how
to apply methods.

\begin{examplewrap}
\begin{nexample}{This is an example.}
  Full solutions to examples are provided here, within the example.
\end{nexample}
\end{examplewrap}

\noindent%
When we think the reader should be ready to do an example problem on their own, we frame it as Guided Practice.

\begin{exercisewrap}
\begin{nexercise}
The reader may check or learn the answer to any Guided Practice
problem by reviewing the full solution in a footnote.\footnotemark{}
%Readers are strongly encouraged to attempt these practice problems.
\end{nexercise}
\end{exercisewrap}
\footnotetext{Guided Practice solutions are always located down here!}

\noindent%
Exercises are also provided at the end of each section
and each chapter for practice or homework assignments.  
Solutions for odd-numbered exercises are given in
Appendix~\ref{eoceSolutions}.

\subsection*{{\color{oiB}Getting involved}}
We encourage anyone learning or teaching statistics to visit
\oiRedirect{textbook-openintro}
    {\color{black}\textbf{openintro.org}}
and get involved.
We value your feedback.
Please send any questions or comments to \href{mailto:leah@openintro.org}{leah@openintro.org}.  You can also provide feedback, report typos, and review known typos at
\begin{center}
\oiRedirect{ahss_feedback}
    {\color{black}\textbf{openintro.org/ahss/feedback}}
\end{center}


\subsection*{{\color{oiB}Acknowledgements}}

This project would not be possible without the passion
and dedication of all those involved.
The authors would like to thank the
\oiRedirect{textbook-openintro_about}{OpenIntro Staff}
for their involvement and ongoing contributions.
We~are also very grateful to the hundreds of students
and instructors who have provided us with valuable feedback
since we first started working on this project in~2009.
A special thank you to Catherine Ko for proofreading
the second edition of AHSS.




