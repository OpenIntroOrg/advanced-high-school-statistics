\chapter{Data sets within the text}
\label{appendix_data}
\label{data_appendix}

%A foundational principle that supports quality statistical
%analysis is well-organized data.

\index{data|(}

Each data set within the text is described in this appendix,
and there is a corresponding page for each of these data sets at
\oiRedirect{data}
    {\color{black}\textbf{openintro.org/data}}.
This page also includes additional data sets that can be
used for honing your skills.
Each data set has its own page with the following information:
\begin{itemize}
\item
    Description of each data set.
\item
    Detailed overview of each data set's variables.
\item
    CSV download.
\item
    R object file download.
\end{itemize}
Over time we will also expand the information available
on these pages.

%\vspace{10mm}

\Comment{Redirects must be created for each link in this appendix.}



\section*{Chapter 1:  \nameref{introductionToData}}
\label{introductionToData}

\begin{itemize}
\item[\ref{basicExampleOfStentsAndStrokes}]
    [\datalink{stent30}, \datalink{stent365}]
    The stent data is split across two data sets,
    one for the 0-30 day and one for the 0-365 day results. \\
    Chimowitz MI, Lynn MJ, Derdeyn CP, et al. 2011.
    Stenting versus Aggressive Medical Therapy for
    Intracranial Arterial Stenosis.
    New England Journal of Medicine 365:993-1003.
    \oiRedirect{textbook-nejm_stent_study}
        {www.nejm.org/doi/full/10.1056/NEJMoa1105335}. \\
    NY Times article:
    \oiRedirect{textbook-nytimes_stent_study}
        {www.nytimes.com/2011/09/08/health/research/08stent.html}.

\item[\ref{dataBasics}]
    [\datalink{loan50}, \datalink{loans\_full\_schema}]
    This data comes from Lending Club
    (\href{https://www.lendingclub.com/info/download-data.action}
        {lendingclub.com}),
    which provides a large set of data on the people who
    received loans through their platform.
    The data used in the textbook comes from a sample
    of the loans made in Q1 (Jan, Feb, March) 2018.
\item[\ref{dataBasics}]
    [\datalink{county}, \datalink{county\_complete}]
    These data come from several government sources.
    For those variables included in the
    county data set, only the most recent data is reported,
    as of what was available in late 2018.
    Data prior to 2011 is all from
    \href{http://census.gov}{census.gov},
    where the specific Quick Facts page providing the data
    is no longer available.
    The more recent data comes from
    \href{https://www.ers.usda.gov/data-products/county-level-data-sets/download-data/}
        {USDA (ers.usda.gov)},
    \href{https://www.bls.gov/lau/}
        {Bureau of Labor Statistics (bls.gov/lau)},
    \href{https://www.census.gov/did/www/saipe/}
        {SAIPE (census.gov/did/www/saipe)},
    and
    \href{https://www.census.gov/programs-surveys/acs/}
        {American Community Survey
            (census.gov/programs-surveys/acs)}.

\item[\ref{section_obs_data_sampling}]
    The Nurses' Health Study was mentioned.
    For more information on this data set, see \\
    \oiRedirect{textbook-channing_nurse_study}
        {www.channing.harvard.edu/nhs}

\item[\ref{experimentsSection}]
    The study we had in mind when discussing the
    simple randomization (no blocking) study was \\
    Anturane Reinfarction Trial Research Group. 1980.
    \emph{Sulfinpyrazone in the prevention of sudden
    death after myocardial infarction.}
    New England Journal of Medicine 302(5):250-256.
\end{itemize}






\section*{Chapter 2:  \nameref{ch_summarizing_data}}
\label{ch_summarizing_data_data}

\begin{itemize}
\item[\ref{numericalData}]
    [\datalink{loan50}, \datalink{county}]
    These data sets are described in
    Data Appendix~\ref{ch_intro_to_data_data}.

\item[\ref{categoricalData}]
    [\datalink{loan50}, \datalink{county}]
    These data sets are described in
    Data Appendix~\ref{ch_intro_to_data_data}.

\item[\ref{caseStudyMalariaVaccine}]
    [\datalink{malaria}]
    Lyke et al. 2017.
    PfSPZ vaccine induces strain-transcending T cells
    and durable protection against heterologous controlled
    human malaria infection.
    PNAS 114(10):2711-2716.
    \url{http://www.pnas.org/content/114/10/2711}
\end{itemize}









\section*{Chapter 3:  \nameref{ch_probability}}
\label{ch_probability_data}

\begin{itemize}
\item[\ref{basicsOfProbability}]
    [\datalink{loan50}, \datalink{county}]
    These data sets are described in
    Data Appendix~\ref{ch_intro_to_data_data}.
\item[\ref{basicsOfProbability}]
    [\datalink{playing\_cards}]
    A table describing the 52 cards in a standard deck.

\item[\ref{conditionalProbabilitySection}]
    [\datalink{family\_college}]
    A simulated data set based on real population summaries at\\
    \oiRedirect{textbook-student_parent_college_2001}
        {nces.ed.gov/pubs2001/2001126.pdf}.
\item[\ref{conditionalProbabilitySection}]
    [\datalink{smallpox}]
    Fenner F. 1988.
    Smallpox and Its Eradication
    (History of International Public Health, No. 6).
    Geneva: World Health Organization. ISBN 92-4-156110-6.

\item[\ref{conditionalProbabilitySection}]
    [Mammogram screening, probabilities.]
    The probabilities reported were obtained using studies
    reported at
    \oiRedirect{textbook-breastCancerDotOrg_20090831b}
        {www.breastcancer.org}
    and \oiRedirect{textbook-ncbi_nih_breast_cancer}
        {www.ncbi.nlm.nih.gov/pmc/articles/PMC1173421}. 

\item[\ref{randomVariablesSection}]
    [\datalink{stocks\_18}]
    Monthly returns for Caterpillar, Exxon Mobil Corp,
    and Google for November 2015 to October 2018.

\item[\ref{contDist}]
    [\datalink{fcid}]
    This sample can be considered a simple random sample
    from the US population.
    It relies on the USDA Food Commodity Intake Database.

\end{itemize}




\section*{Chapter 4:  \nameref{ch_distributions}}
\label{ch_distributions_data}

\begin{itemize}
\item[\ref{normalDist}]
    [SAT and ACT score distributions]
    The SAT score data comes from the 2018 distribution,
    which is provided at \\
    {\small
    \oiRedirect{textbook-collegeboard_sat_2018_score_distribution}
        {reports.collegeboard.org/pdf/2018-total-group-sat-suite-assessments-annual-report.pdf}} \\
    The ACT score data is available at \\
    {\footnotesize
    \oiRedirect{textbook-act_2018_score_distribution}
        {act.org/content/dam/act/unsecured/documents/cccr2018/P\_99\_999999\_N\_S\_N00\_ACT-GCPR\_National.pdf}} \\
    We also acknowledge that the actual ACT score distribution
    is \emph{not} nearly normal.
    However, since the topic is very accessible,
    we decided to keep the context and examples.
\item[\ref{normalDist}]
    [Male heights]
    The distribution is based on the
    USDA Food Commodity Intake Database.
\item[\ref{normalDist}]
    [\datalink{possum}]
    The distribution parameters are based on a sample
    of possums from Australia and New Guinea.
    The original source of this data is as follows.
    Lindenmayer DB, et al. 1995.
    \emph{Morphological variation among columns of the
        mountain brushtail possum, Trichosurus caninus
        Ogilby (Phalangeridae: Marsupiala)}.
    Australian Journal of Zoology 43: 449-458.

%\item[\ref{assessingNormal}]
%    [\datalink{male\_heights\_fcid}]
%    This sample can be considered a simple random sample
%    from the US population.
%    It relies on the USDA Food Commodity Intake Database.
%\item[\ref{assessingNormal}]
%    [\datalink{simulated\_normal}]
%    These data were simulated from a standard normal distribution.
%    This data set includes three different data sets.
%\item[\ref{assessingNormal}]
    %[\datalink{nba\_players\_19}]
   % Summary information from the NBA players for the
   % 2018-2019 season.
   % Data were retrieved from
   %\oiRedirect{data-nba_players_19}{www.nba.com/players}.
%\item[\ref{assessingNormal}]
%    [\datalink{poker}]
%    Poker winnings (and losses) for 50 days by a professional
%    poker player, which represents their first 50 days trying
%    to play for a living.
%    Anonymity has been requested by the player.
%\item[\ref{assessingNormal}]
%    [\datalink{simulated\_dist}]
%    Simulated data sets,
%    not necessarily drawn from a normal distribution.
%    This data set includes six different data sets.

\item[\ref{geomDist}]
    [Exceeding insurance deductible]
    These statistics were made up but are possible
    values one might observe for low-deductible plans.
 
\item[\ref{binomialModel}]
    [US smoking rate]
    The 15\% smoking rate in the US figure is close to
    the value from the Centers for Disease Control and
    Prevention website, which reports a value of 14\%
    as of the 2017 estimate: 
    \href{https://www.cdc.gov/tobacco/data_statistics/fact_sheets/adult_data/cig_smoking/index.htm}{cdc.gov/tobacco/data\_statistics/fact\_sheets/adult\_data/cig\_smoking/index.htm}

\end{itemize}








\section*{Chapter 5:  \nameref{ch_foundations_for_inf}}
\label{ch_foundations_for_inf_data}

\begin{itemize}
\item[\ref{pointEstimates}]
    [\datalink{pew\_energy\_2018}]
    The actual data has more observations than were referenced
    in this chapter.
    That is, we used a subsample since it helped smooth some
    of the examples to have a bit more variability.
    The \data{pew\us{}energy\us{}2018} data set represents
    the full data set for each of the different energy source
    questions, which covers solar, wind, offshore drilling,
    hydrolic fracturing, 
\newpage and nuclear energy.
    The statistics used to construct the data are from
    the following page:
    \begin{center}
    \oiRedirect{textbook-pew_2018_poll_on_solar_and_wind_expansion}
        {{\small{www.pewinternet.org/2018/05/14/majorities-see-government-efforts-to-protect-the-environment-as-insufficient/}}}
    \end{center}
    
\item[\ref{confidenceIntervals}]
    [\datalink{pew\_energy\_2018}]
    See the details for this data set above
    in the Section~\ref{pointEstimates} data section.
\item[\ref{confidenceIntervals}]
    [\datalink{ebola\_survey}]
    In New York City on October 23rd, 2014, a doctor who had
    recently been treating Ebola patients in Guinea went to
    the hospital with a slight fever and was subsequently
    diagnosed with Ebola.
    Soon thereafter, an NBC~4 New York/The Wall Street
    Journal/Marist Poll found that
    82\% of New Yorkers favored a
    ``mandatory 21-day quarantine for anyone who has come
    in contact with an Ebola patient''.
    This poll included responses of 1,042
    New York adults between Oct 26th and~28th, 2014.
    \oiRedirect{textbook-maristpoll_ebola_201410}
        {Poll ID NY141026 on maristpoll.marist.edu}.

\item[\ref{hypothesisTesting}]
    [\datalink{pew\_energy\_2018}]
    See the details for this data set above
    in the Section~\ref{pointEstimates} data section.
\end{itemize}







\section*{Chapter 6:  \nameref{ch_inference_for_props}}
\label{ch_inference_for_props_data}

\begin{itemize}
\item[\ref{hypothesisTesting}]
    [Nuclear energy]
   A Gallup poll of 1,019 adults in the US, conducted in March of 2016, found that 54\% of respondents oppose nuclear energy.  This was the first time since Gallup first asked the question in 1994 that a majority of respondents said they oppose nuclear energy.\\
\oiRedirect{textbook-nuclear_energy_support_2016}
        {https://news.gallup.com/poll/190064/first-time-majority-oppose-nuclear-energy.aspx}

\item[\ref{hypothesisTesting}]
    [Supreme Court]
    The Gallup organization began measuring the public's view of the Supreme Court's job performance in 2000, and has measured it every year since then with the question: ``Do you approve or disapprove of the way the Supreme Court is handling its job?".  In 2018, the Gallup poll randomly sampled 1,033 adults in the U.S. and found that 53\% of them approved. \\   
\oiRedirect{textbook-gallup_supreme_court_approval_2018}
        {https://news.gallup.com/poll/237269/supreme-court-approval-highest-2009.aspx}

\item[\ref{hypothesisTesting}]
    [Life on other planets]
    A February 2018 Marist Poll reported: ``Many Americans (68\%) think there is intelligent life on other planets".  The results were based on a random sample of 1,033 adults in the U.S. \\
\oiRedirect{textbook-intelligent_life_2018}{http://maristpoll.marist.edu/212-are-americans-poised-for-an-alien-invasion}


\item[\ref{differenceOfTwoProportions}]
    [\datalink{cpr}]
    B$\ddot{\text{o}}$ttiger et al.
    \emph{Efficacy and safety of thrombolytic therapy after
        initially unsuccessful cardiopulmonary resuscitation:
        a prospective clinical trial}.
        The Lancet, 2001.
\item[\ref{differenceOfTwoProportions}]
    [\datalink{fish\_oil\_18}]
    Manson JE, et al. 2018.
    Marine n-3 Fatty Acids and Prevention of
    Cardiovascular Disease and Cancer. NEJMoa1811403.
\item[\ref{differenceOfTwoProportions}]
    [\datalink{mammogram}]
    \oiRedirect{textbook-90k_mammogram_study_2014}
        {Miller AB. 2014.
            \emph{Twenty five year follow-up for breast cancer
            incidence and mortality of the Canadian National
            Breast Screening Study: randomised screening trial}.
            BMJ 2014;348:g366.}
\item[\ref{differenceOfTwoProportions}]
    [\datalink{drone\_blades}]
    The quality control data set for quadcopter drone blades
    is a made-up data set for an example.
    We provide the simulated data in the
    \data{drone\us{}blades} data set.

\item[\ref{oneWayChiSquare}]
    [M\$Ms]
   Starting at the end of 2016, Rick Wicklin, a statistician working at the statistical software company SAS, collected a sample of 712 candies, or about 1.5 pounds, and counted how many there were of each color.   \\
\oiRedirect{textbook-mandms}{https://qz.com/918008/the-color-distribution-of-mms-as-determined-by-a-phd-in-statistics}



\item[\ref{twoWayTablesAndChiSquare}]
    [\datalink{ask}]
    Minson JA, Ruedy NE, Schweitzer ME.
    \emph{There is such a thing as a stupid question:
    Question disclosure in strategic communication}. \\
    {\small\oiRedirect{minson_ruedy_data_source}
        {opim.wharton.upenn.edu/DPlab/papers/workingPapers/}}\\
    {\small\oiRedirect{minson_ruedy_data_source}
        {Minson\_working\_Ask\%20(the\%20Right\%20Way)\%20and\%20You\%20Shall\%20Receive.pdf}}

\item[\ref{twoWayTablesAndChiSquare}]
    [\datalink{diabetes2}]
    Zeitler P, et al. 2012.
    A Clinical Trial to Maintain Glycemic Control in Youth
    with Type~2 Diabetes.
    N Engl J Med.

\end{itemize}






\section*{Chapter 7:  \nameref{ch_inference_for_means}}
\label{ch_inference_for_means_data}

\begin{itemize}
\item[\ref{oneSampleMeansWithTDistribution}]
    [Risso's dolphins]
    Endo T and Haraguchi K. 2009.
    High mercury levels in hair samples from residents of Taiji,
    a Japanese whaling town.
    Marine Pollution Bulletin 60(5):743-747.

    Taiji was featured in the movie
    \emph{The Cove}, and it is a significant source of dolphin
    and whale meat in Japan.
    Thousands of dolphins pass through the Taiji area annually,
    and we will assume these 19 dolphins represent a simple
    random sample from those dolphins.
\item[\ref{oneSampleMeansWithTDistribution}]
    [Croaker white fish]
    \oiRedirect{textbook-fda_mercury_in_fish_2010}
        {www.fda.gov/food/foodborneillnesscontaminants/metals/ucm115644.htm}
\item[\ref{oneSampleMeansWithTDistribution}]
    [\datalink{run17}]
    \oiRedirect{textbook-cherryblossom_org}{www.cherryblossom.org}

\item[\ref{pairedData}]
    [\datalink{textbooks}, \datalink{ucla\_textbooks\_f18}]
    Data were collected by OpenIntro staff in 2010 and again
    in 2018.
    For the 2018 sample, we sampled 201 UCLA courses.
    Of those, 68 required books that could be
    found on Amazon.
    The websites where information was retrieved: \\
    \oiRedirect{ucla_class_schedule}
        {sa.ucla.edu/ro/public/soc},
    \oiRedirect{ucla_verbacompare}{ucla.verbacompare.com},
    and \oiRedirect{amazon}{amazon.com}.

\item[\ref{theTDistributionForTheDifferenceOfTwoMeans}]
    [Jennifer-John]
    Bertrand M, Mullainathan S. 2004.
    \emph{Science faculty's subtle gender biases favor male students}.
    PNAS October 9, 2012 109 (41) 16474-16479.\\
    \oiRedirect{textbook-jennifer-john}
        {https://www.pnas.org/content/109/41/16474}


\item[\ref{theTDistributionForTheDifferenceOfTwoMeans}]
    [\datalink{resume}]
    Bertrand M, Mullainathan S. 2004.
    \emph{Are Emily and Greg More Employable than Lakisha and Jamal?
    A Field Experiment on Labor Market Discrimination}.
    The American Economic Review 94:4 (991-1013).
    \oiRedirect{resume-data-2004}
        {www.nber.org/papers/w9873}

\item[\ref{theTDistributionForTheDifferenceOfTwoMeans}]
    [\datalink{stem\_cells}]
        {Menard C, et al. 2005.
            Transplantation of cardiac-committed mouse embryonic
            stem cells to infarcted sheep myocardium:
            a preclinical study.
            The Lancet: 366:9490, p1005-1012.}
    \oiRedirect{menard-stem-cells-2005}{https://www.thelancet.com/journals/lancet/article/PIIS0140-6736(05)67380-1/fulltext}


\end{itemize}






\section*{Chapter 8:  \nameref{ch_regr_simple_linear}}
\label{ch_regr_simple_linear_data}

\begin{itemize}
\item[\ref{fitting_line_to_data_section}]
    [\datalink{simulated\_scatter}]
    Fake data used for the first three plots.
    The perfect linear plot uses group~4 data,
    where \var{group} variable in the data set
    (Figure~\ref{perfLinearModel}).
    The group of 3 imperfect linear plots use groups~1-3
    (Figure~\ref{imperfLinearModel}).
    The sinusoidal curve uses group~5 data
    (Figure~\ref{notGoodAtAllForALinearModel}).
    The curved plot of data with the curved band
    uses group~30 data
    (right panel of Figure~\ref{scattHeadLTotalLTube}).
    The group of 3 scatterplots with residual plots use groups~6-8
    (Figure~\ref{sampleLinesAndResPlots}).
    The correlation plots uses groups~9-19 data
    (Figures~\ref{posNegCorPlots} and~\ref{corForNonLinearPlots}).
\item[\ref{fitting_line_to_data_section}]
    [\datalink{possum}]
    This data is described in
    Data Appendix~\ref{ch_distributions_data}.

\item[\ref{fittingALineByLSR}]
    [\datalink{elmhurst}]
    These data were sampled from a table of data for all
    freshman from the 2011 class at Elmhurst College that
    accompanied an article titled
    \emph{What Students Really Pay to Go to College}
    published online by \emph{The~Chronicle of Higher Education}:
    \oiRedirect{textbook-chronicle_elmhurst_article}
        {chronicle.com/article/What-Students-Really-Pay-to-Go/131435}.
\item[\ref{fittingALineByLSR}]
    [\datalink{simulated\_scatter}]
    The plots for things that can go wrong uses groups 20-23
    (Figure~\ref{whatCanGoWrongWithLinearModel}).
\item[\ref{fittingALineByLSR}]
    [\datalink{mario\_kart}]
    \Comment{Confirm data set name change.}
    Auction data from Ebay (ebay.com) for the game Mario Kart
    for the Nintendo Wii.
    This data set was collected in early October, 2009.

\item[\ref{typesOfOutliersInLinearRegression}]
    [\datalink{simulated\_scatter}]
    The plots for types of outliers uses groups 24-29
    (Figure~\ref{outlierPlots}).

\item[\ref{inferenceForLinearRegression}]
    [\datalink{midterms\_house}]
    Data was retrieved from Wikipedia.

\end{itemize}


\index{data|)}



