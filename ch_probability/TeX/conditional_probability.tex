\exercisesheader{}

% 13

\eoce{\qt{Joint and conditional probabilities\label{joint_cond}} P(A) = 0.3, 
P(B) = 0.7
\begin{parts}
\item Can you compute P(A and B) if you only know P(A) and P(B)?
\item Assuming that events A and B arise from independent random processes,
\begin{subparts}
\item what is P(A and B)?
\item what is P(A or B)?
\item what is P(A$|$B)?
\end{subparts}
\item If we are given that P(A and B) = 0.1, are the random variables giving rise 
to events A and B independent?
\item If we are given that P(A and B) = 0.1, what is P(A$|$B)?
\end{parts}
}{}

% 14

\eoce{\qt{PB \& J\label{pbj}} Suppose 80\% of people like peanut butter, 89\% 
like jelly, and 78\% like both. Given that a randomly sampled person likes peanut 
butter, what's the probability that he also likes jelly?
}{}

% 15

\eoce{\qt{Global warming\label{global_warming}} \ \videosolution{ahss_eoce_sol-global_warming}\ \ A Pew Research poll asked 
1,306 Americans ``From what you've read and heard, is there solid evidence that 
the average temperature on earth has been getting warmer over the past few 
decades, or not?". The table below shows the distribution of responses by party 
and ideology, where the counts have been replaced with relative frequencies.
\footfullcite{globalWarming}
\begin{center}
\begin{tabular}{ll  ccc c} 
                    &                           & \multicolumn{3}{c}{\textit{Response}} \\
\cline{3-5}
                    &                           & Earth is  & Not       & Don't Know    &   \\
                    &                           & warming   & warming   & Refuse        & Total\\
\cline{2-6}
                    & Conservative Republican   & 0.11      & 0.20      & 0.02      & 0.33  \\
\textit{Party and}  & Mod/Lib Republican        & 0.06      & 0.06      & 0.01      & 0.13 \\
\textit{Ideology}   & Mod/Cons Democrat         & 0.25      & 0.07      & 0.02      & 0.34 \\
                    & Liberal Democrat          & 0.18      & 0.01      & 0.01      & 0.20\\
\cline{2-6}
                    &Total                      & 0.60      & 0.34      & 0.06      & 1.00
\end{tabular}
\end{center}
\begin{parts}
\item Are believing that the earth is warming and being a liberal Democrat mutually 
exclusive?
\item What is the probability that a randomly chosen respondent believes the 
earth is warming or is a liberal Democrat?
\item What is the probability that a randomly chosen respondent believes the 
earth is warming given that he is a liberal Democrat?
\item What is the probability that a randomly chosen respondent believes the 
earth is warming given that he is a conservative Republican?
\item Does it appear that whether or not a respondent believes the earth is 
warming is independent of their party and ideology? Explain your reasoning.
\item What is the probability that a randomly chosen respondent is a 
moderate/liberal Republican given that he does not believe that the earth is 
warming? 
\end{parts}
}{}

\D{\newpage}

% 16

\eoce{\qt{Health coverage, relative frequencies\label{health_coverage_rel_freqs}} 
The Behavioral Risk Factor Surveillance System (BRFSS) is an annual telephone 
survey designed to identify risk factors in the adult population and report 
emerging health trends. The following table displays the distribution of health 
status of respondents to this survey (excellent, very good, good, fair, poor) 
and whether or not they have health insurance.
\begin{center}
\begin{tabular}{rrrrrrrr}
& &  \multicolumn{5}{c}{\textit{Health Status}} &  \\ 
\cline{3-7}
                    &       & Excellent & Very good & Good      & Fair      & Poor      & Total \\ 
\cline{2-8}
\textit{Health}     & No    & 0.0230    & 0.0364    & 0.0427    & 0.0192    & 0.0050    & 0.1262 \\ 
\textit{Coverage}   & Yes   & 0.2099    & 0.3123    & 0.2410    & 0.0817    & 0.0289    & 0.8738 \\ 
\cline{2-8}
                    & Total & 0.2329    & 0.3486    & 0.2838    & 0.1009    & 0.0338    & 1.0000
\end{tabular}
\end{center}
\begin{parts}
\item Are being in excellent health and having health coverage mutually 
exclusive?
\item What is the probability that a randomly chosen individual has excellent 
health?
\item What is the probability that a randomly chosen individual has excellent 
health given that he has health coverage?
\item What is the probability that a randomly chosen individual has excellent 
health given that he doesn't have health coverage?
\item Do having excellent health and having health coverage appear to be 
independent?
\end{parts}
}{}

% 17

\eoce{\qt{Burger preferences\label{burger_preferences}} A 2010 SurveyUSA poll 
asked 500 Los Angeles residents, ``What is the best hamburger place in Southern 
California? Five Guys Burgers? In-N-Out Burger? Fat Burger? Tommy's Hamburgers? 
Umami Burger? Or somewhere else?'' The distribution of responses by gender is 
shown below. \footfullcite{burgers}
\begin{center}
\begin{tabular}{l p{4cm} r r r }
                    &                       & \multicolumn{2}{c}{\textit{Gender}} \\
\cline{3-4}
                    &                       & Male  & Female    & Total \\
\cline{2-5}
                    & Five Guys Burgers     & 5     & 6         & 11 \\
                    & In-N-Out Burger       & 162   & 181       & 343 \\
\textit{Best}       & Fat Burger            & 10    & 12        & 22 \\
\textit{hamburger}  & Tommy's Hamburgers    & 27    & 27        & 54 \\ 
\textit{place}      & Umami Burger          & 5     & 1         & 6 \\
                    & Other                 & 26    & 20        & 46 \\
                    & Not Sure              & 13    & 5         & 18 \\
\cline{2-5}  
                    & Total                 & 248   & 252       & 500
\end{tabular}
\end{center}
\begin{parts}
\item Are being female and liking Five Guys Burgers mutually exclusive?
\item What is the probability that a randomly chosen male likes In-N-Out the best?
\item What is the probability that a randomly chosen female likes In-N-Out the 
best?
\item What is the probability that a man and a woman who are dating both like 
In-N-Out the best? Note any assumption you make and evaluate whether you think 
that assumption is reasonable.
\item What is the probability that a randomly chosen person likes Umami best or 
that person is female?
\end{parts}
}{}

\D{\newpage}

% 18

\eoce{\qt{Assortative mating\label{assortative_mating}} Assortative mating is a 
nonrandom mating pattern where individuals with similar genotypes and/or 
phenotypes mate with one another more frequently than what would be expected 
under a random mating pattern. Researchers studying this topic collected data on 
eye colors of 204 Scandinavian men and their female partners. The table below 
summarizes the results. For simplicity, we only include heterosexual 
relationships in this exercise. \footfullcite{Laeng:2007}
\begin{center}
\begin{tabular}{ll  ccc c} 
                                        &           & \multicolumn{3}{c}{\textit{Partner (female)}} \\
\cline{3-5}
                                        &           & Blue  & Brown     & Green     & Total \\
\cline{2-6}
                                        & Blue      & 78    & 23        & 13        & 114 \\
\multirow{2}{*}{\textit{Self (male)}}   & Brown     & 19    & 23        & 12        & 54 \\
                                        & Green     & 11    & 9         & 16        & 36 \\
\cline{2-6}  
                                        & Total     & 108   & 55        & 41        & 204
\end{tabular}
\end{center}
\begin{parts}
\item What is the probability that a randomly chosen male respondent or his 
partner has blue eyes?
\item What is the probability that a randomly chosen male respondent with blue 
eyes has a partner with blue eyes? 
\item What is the probability that a randomly chosen male respondent with brown 
eyes has a partner with blue eyes? What about the probability of a randomly 
chosen male respondent with green eyes having a partner with blue eyes?
\item Does it appear that the eye colors of male respondents and their partners 
are independent? Explain your reasoning.
\end{parts}
}{}


% 19

\eoce{\qt{Marbles in an urn\label{marbles_in_urn}} Imagine you have an urn 
containing 5 red, 3 blue, and 2 orange marbles in it. 
\begin{parts}
\item What is the probability that the first marble you draw is blue?
\item Suppose you drew a blue marble in the first draw. If drawing with 
replacement, what is the probability of drawing a blue marble in the second draw?
\item Suppose you instead drew an orange marble in the first draw. If drawing 
with replacement, what is the probability of drawing a blue marble in the second 
draw?
\item If drawing with replacement, what is the probability of drawing two blue 
marbles in a row?
\item When drawing with replacement, are the draws independent? Explain.
\end{parts}
}{}

% 20

\eoce{\qt{Socks in a drawer\label{socks_in_drawer}} In your sock drawer you have 
4 blue, 5 gray, and 3 black socks. Half asleep one morning you grab 2 socks at 
random and put them on. Find the probability you end up wearing
\begin{parts}
\item 2 blue socks
\item no gray socks
\item at least 1 black sock
\item a green sock
\item matching socks
\end{parts}
}{}

% 21

\eoce{\qt{Chips in a bag\label{chips_in_bag}} Imagine you have a bag 
containing 5 red, 3 blue, and 2 orange chips.
\begin{parts}
\item Suppose you draw a chip and it is blue. If drawing without replacement, 
what is the probability the next is also blue?
\item Suppose you draw a chip and it is orange, and then you draw a second chip 
without replacement. What is the probability this second chip is blue?
\item If drawing without replacement, what is the probability of drawing two blue 
chips in a row?
\item When drawing without replacement, are the draws independent? Explain.
\end{parts}
}{}

\D{\newpage}

% 22

\eoce{\qt{Books on a bookshelf\label{books_on_shelf}} The table below shows the 
distribution of books on a bookcase based on whether they are nonfiction or 
fiction and hardcover or paperback.
\begin{center}
\begin{tabular}{ll  cc c} 
                                &           & \multicolumn{2}{c}{\textit{Format}} \\
\cline{3-4}
                                &           & Hardcover     & Paperback     & Total \\
\cline{2-5}
\multirow{2}{*}{\textit{Type}}  & Fiction   & 13            & 59            & 72 \\
                                & Nonfiction& 15            & 8             & 23 \\
\cline{2-5} 
                                & Total     & 28            & 67            & 95 \\
\cline{2-5}
\end{tabular}
\end{center}
\begin{parts}
\item Find the probability of drawing a hardcover book first then a paperback 
fiction book second when drawing without replacement.
\item Determine the probability of drawing a fiction book first and then a 
hardcover book second, when drawing without replacement.
\item Calculate the probability of the scenario in part~(b), except this time 
complete the calculations under the scenario where the first book is placed back 
on the bookcase before randomly drawing the second book.
\item The final answers to parts~(b) and~(c) are very similar. Explain why this 
is the case.
\end{parts}
}{}

% 23

\eoce{\qt{Student outfits\label{student_outfits}} \videohref{ahss_eoce_sol-student_outfits}\ \ In a classroom with 24 
students, 7 students are wearing jeans, 4 are wearing shorts, 8 are wearing 
skirts, and the rest are wearing leggings. If we randomly select 3 students 
without replacement, what is the probability that one of the selected students is 
wearing leggings and the other two are wearing jeans? Note that these are 
mutually exclusive clothing options.
}{}

% 24

\eoce{\qt{The birthday problem\label{birthday_problem}} Suppose we pick three 
people at random. For each of the following questions, ignore the special case 
where someone might be born on February 29th, and assume that births are evenly 
distributed throughout the year.
\begin{parts}
\item What is the probability that the first two people share a birthday? 
\item What is the probability that at least two people share a birthday?
\end{parts}
}{}


% 25

\eoce{\qt{Drawing box plots\label{tree_drawing_box_plots}} After an introductory 
statistics course, 80\% of students can successfully construct box plots. Of 
those who can construct box plots, 86\% passed, while only 65\% of those students 
who could not construct box plots passed.
\begin{parts}
\item Construct a tree diagram of this scenario.
\item Calculate the probability that a student is able to construct a box plot 
if it is known that he passed.
\end{parts}
}{}

% 26

\eoce{\qt{Predisposition for thrombosis\label{tree_thrombosis}} A genetic test is 
used to determine if people have a predisposition for \textit{thrombosis}, which 
is the formation of a blood clot inside a blood vessel that obstructs the flow of 
blood through the circulatory system. It is believed that 3\% of people actually 
have this predisposition. The genetic test is 99\% accurate if a person actually 
has the predisposition, meaning that the probability of a positive test result 
when a person actually has the predisposition is 0.99. The test is 98\% accurate 
if a person does not have the predisposition. What is the probability that a 
randomly selected person who tests positive for the predisposition by the test 
actually has the predisposition?
}{}

% 27

\eoce{\qt{It's never lupus\label{tree_lupus}} \videohref{ahss_eoce_sol-tree_lupus}\ \  Lupus is a medical phenomenon where 
antibodies that are supposed to attack foreign cells to prevent infections 
instead see plasma proteins as foreign bodies, leading to a high risk of blood 
clotting. It is believed tha t 2\% of the population suffer from this disease. The 
test is 98\% accurate if a person actually has the disease. The test is 74\% 
accurate if a person does not have the disease. There is a line from the Fox 
television show \emph{House} that is often used after a patient tests positive 
for lupus: ``It's never lupus." Do you think there is truth to this statement? 
Use appropriate probabilities to support your answer.
}{}

% 28

\eoce{\qt{Exit poll\label{tree_exit_poll}} Edison Research gathered exit poll 
results from several sources for the Wisconsin recall election of Scott Walker. 
They found that 53\% of the respondents voted in favor of Scott Walker. 
Additionally, they estimated that of those who did vote in favor for Scott 
Walker, 37\% had a college degree, while 44\% of those who voted against Scott 
Walker had a college degree. Suppose we randomly sampled a person who 
participated in the exit poll and found that he had a college degree. What is the 
probability that he voted in favor of Scott Walker?
\footfullcite{data:scott}
}{}
