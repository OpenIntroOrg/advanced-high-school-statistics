\exercisesheader{}

% 31 - hearts

\eoce{\qt{Hearts win\label{hearts}} In a new card game, you start
with a well-shuffled full deck and draw 3 cards without replacement.
If you draw 3 hearts, 
you win \$50. If you draw 3 black cards, you win \$25. For any other draws, you 
win nothing.
\begin{parts}
\item Create a probability model for the amount you win at this game, and find 
the expected winnings. Also compute the standard deviation of this distribution.
\item If the game costs \$5 to play, what would be the expected value and 
standard deviation of the net profit (or loss)? \textit{(Hint: 
profit = winnings $-$ cost; $X-5$)}
\item If the game costs \$5 to play, should you play this game? Explain.
\end{parts}
}{}

% 32 - ace_of_clubs

\eoce{\qt{Ace of clubs wins\label{ace_of_clubs}} Consider the following card game 
with a well-shuffled deck of cards. If you draw a red card, you win nothing. If 
you get a spade, you win \$5. For any club, you win \$10 plus an extra \$20 for 
the ace of clubs.
\begin{parts}
\item Create a probability model for the amount you win at this game. Also, find 
the expected winnings for a single game and the standard deviation of the 
winnings.
\item What is the maximum amount you would be willing to pay to play this game? 
Explain your reasoning.
\end{parts}
}{}

% 33 - portfolio_return

\eoce{\qt{Portfolio return\label{portfolio_return}} A portfolio's value increases 
by 18\% during a financial boom and by 9\% during normal times. It decreases by 
12\% during a recession. What is the expected return on this portfolio if each 
scenario is equally likely?
}{}

% 34 - baggage_fees

\eoce{\qt{Baggage fees\label{baggage_fees}} An airline charges the following 
baggage fees: \$25 for the first bag and \$35 for the second. Suppose 54\% of 
passengers have no checked luggage, 34\% have one piece of checked luggage and 
12\% have two pieces. We suppose a negligible portion of people check more than 
two bags.
\begin{parts}
\item Build a probability model, compute the average revenue per passenger, and 
compute the corresponding standard deviation.
\item About how much revenue should the airline expect for a flight of 120 
passengers? With what standard deviation? Note any assumptions you make and if 
you think they are justified.
\end{parts}
}{}

% 35 - roulette_american

\eoce{\qt{American roulette\label{roulette_american}} The game of American 
roulette involves spinning a wheel with 38 slots: 18 red, 18 black, and 2 green. 
A ball is spun onto the wheel and will eventually land in a slot, where each slot 
has an equal chance of capturing the ball. Gamblers can place bets on red or 
black. If the ball lands on their color, they double their money. If it lands on 
another color, they lose their money. Suppose you bet \$1 on red. What's the 
expected value and standard deviation of your winnings?
}{}

% 36 - roulette_european

\eoce{\qt{European roulette\label{roulette_european}} The game of European 
roulette involves spinning a wheel with 37 slots: 18 red, 18 black, and 1 green. 
A ball is spun onto the wheel and will eventually land in a slot, where each slot 
has an equal chance of capturing the ball. Gamblers can place bets on red or 
black. If the ball lands on their color, they double their money. If it lands on 
another color, they lose their money.
\begin{parts}
\item Suppose you play roulette and bet \$3 on a single round. What is the 
expected value and standard deviation of your total winnings?
\item Suppose you bet \$1 in three different rounds. What is the expected value 
and standard deviation of your total winnings?
\item How do your answers to parts (a) and (b) compare? What does this say about 
the riskiness of the two games?
\end{parts}
}{}

% 37 - lemonade

\eoce{\qt{Lemonade at The Cafe\label{lemonade}} Drink pitchers at The Cafe are intended to hold about 64 ounces of lemonade and glasses hold about 12 ounces. However, when the pitchers are filled by a server, they do not always fill it with exactly 64 ounces. There is some variability. Similarly, when they pour out some of the lemonade, they do not pour exactly 12 ounces. The amount of lemonade in a pitcher is normally distributed with mean 64 ounces and standard deviation 1.732 ounces. The amount of lemonade in a glass is normally distributed with mean 12 ounces and standard deviation 1 ounce. 
\begin{parts}
\item How much lemonade would you expect to be left in a pitcher after pouring one glass of lemonade? 
\item What is the standard deviation of the amount left in a pitcher after pouring one glass of lemonade? 
\item What is the probability that more than 50 ounces of lemonade is left in a pitcher after pouring one glass of lemonade?
\end{parts}
}{}

% 38 - spray_paint_1

\eoce{\qt{Spray paint, Part I} \label{spray_paint_1} Suppose the area that can be painted using a single can of spray paint is slightly variable and follows a nearly normal distribution with a mean of 25 square feet and a standard deviation of 3 square feet. Suppose also that you buy three cans of spray paint.
\begin{parts}
\item How much area would you expect to cover with these three cans of spray paint?
\item What is the standard deviation of the area you expect to cover with these three cans of spray paint?
\item The area you wanted to cover is 80 square feet. What is the probability that you will be able to cover this entire area with these three cans of spray paint?
\end{parts}
}{}

% 39 - GRE_scores_3_ap

\eoce{\qt{GRE scores, Part III\label{GRE_scores_3}}
\videosolution{ahss_eoce_sol-gre_scores_part_III}
In Exercises~\ref{GRE_intro_ap} and~\ref{GRE_cutoffs_ap} we saw two distributions for GRE scores: $N(\mu=151, \sigma=7)$ for the verbal part of the exam and $N(\mu=153, \sigma=7.67)$ for the quantitative part. Suppose performance on these two sections is independent. Use this information to compute each of the following:
\begin{parts}
\item The probability of a combined (verbal + quantitative) score above 320. 
\item The score of a student who scored better than 90\% of the test takers overall.
\end{parts}
}{}

% 40 - betting_on_dinner_1

\eoce{\qt{Betting on dinner, Part I \label{betting_on_dinner_1}} Suppose a restaurant is running a promotion where prices of menu items are random following some underlying distribution. If you're lucky, you can get a basket of fries for \$3, or if you're not so lucky you might end up having to pay \$10 for the same menu item. The price of basket of fries is drawn from a normal distribution with mean \$6 and standard deviation of \$2. The price of a fountain drink is drawn from a normal distribution with mean \$3 and standard deviation of \$1. What is the probability that you pay more than \$10 for a dinner consisting of a basket of fries and a fountain drink?
}{}
