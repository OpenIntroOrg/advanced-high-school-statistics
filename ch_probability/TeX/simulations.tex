\exercisesheader{}

% 33
\eoce{\qt{Smog check, Part I} \label{smogCheck} Suppose 16\% of cars fail pollution tests (smog checks) in California. We would like to estimate the probability that an entire fleet of seven cars would pass using a simulation. We assume each car is independent. We only want to know if the entire fleet passed, i.e. none of the cars failed. What is wrong with each of the following simulations to represent whether an entire (simulated) fleet passed?
\begin{parts}
\item Flip a coin seven times where each toss represents a car. A head means the car passed and a tail means it failed. If all cars passed, we report PASS for the fleet. If at least one car failed, we report FAIL.
\item Read across a random number table starting at line 5. If a number is a 0 or 1, let it represent a failed car. Otherwise the car passes. We report PASS if all cars passed and FAIL otherwise.
\item Read across a random number table, looking at two digits for each simulated car. If a pair is in the range [00-16], then the corresponding car failed. If it is in [17-99], the car passed. We report PASS if all cars passed and FAIL otherwise.
\end{parts}
}{}

% 34

\eoce{\qt{Left-handed} Studies suggest that approximately 10\% of the world population is left-handed. Use ten simulations to answer each of the following questions. For each question, describe your simulation scheme clearly.
\begin{parts}
\item What is the probability that at least one out of eight people are left-handed?
\item On average, how many people would you have to sample until the first person who is left-handed?
\item On average, how many left-handed people would you expect to find among a random sample of six people?
\end{parts}
}{}

% 35

\eoce{\qt{Smog check, Part II} Consider the fleet of seven cars in Exercise~\ref{smogCheck}. Remember that 16\% of cars fail pollution tests (smog checks) in California, and that we assume each car is independent.
\begin{parts}
\item Write out how to calculate the probability of the fleet failing, i.e. at least one of the cars in the fleet failing, via simulation.
\item Simulate 5 fleets. Based on these simulations, estimate the probability at least one car will fail in a fleet.
\item Compute the probability at least one car fails in a fleet of seven.
\end{parts}
}{}

% 36

\eoce{\qt{To catch a thief} \label{thief} Suppose that at a retail store, $1/5^{th}$ of all employees steal some amount of merchandise. The stores would like to put an end to this practice, and one idea is to use lie detector tests to catch and fire thieves. However, there is a problem: lie detectors are not 100\% accurate. Suppose it is known that a lie detector has a failure rate of 25\%. A thief will slip by the test 25\% of the time and an honest employee will only pass 75\% of the time. 
\begin{parts}
\item Describe how you would simulate whether an employee is honest or is a thief using a random number table. Write your simulation very carefully so someone else can read it and follow the directions exactly.
\item Using a random number table, simulate 20 employees working at this store and determine if they are honest or not. Make sure to record the random digits assigned to each employee as you will refer back to these in part (c).
\item Determine the result of the lie detector test for each simulated employee  from part (b) using a new simulation scheme.
\item How many of these employees are ``honest and passed" and how many are ``honest and failed"?
\item How many of these employees are ``thief and passed" and how many are ``thief and failed"?
\item Suppose the management decided to fire everyone who failed the lie detector test. What percent of fired employees were honest? What percent of not fired employees were thieves?
\end{parts}
}{}
