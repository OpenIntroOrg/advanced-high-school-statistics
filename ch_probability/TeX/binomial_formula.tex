\exercisesheader{}

% 29
\eoce{\qt{Exploring combinations} A coin is tossed 5 times.  How many sequences / combinations of Heads/Tails are there that have: 
\begin{parts}
\item Exactly 1 Tail?
\item Exactly 4 Tails?
\item Eactly 3 Tails?
\item At least 3 Tails?
\end{parts}
}{}

% 30

\eoce{\qt{Political affiliation}  Suppose that in a large population, 51\% identify as Democrat. A researcher takes a random sample of 3 people.
\begin{parts}
\item Use the binomial model to calculate the probability that two of them identify as Democrat.
\item Write out all possible orderings of 3 people, 2 of whom identify as Democrat. Use these scenarios to calculate the same probability from part (a) but using the Addition Rule for disjoint events. Confirm that your answers from parts (a) and (b) match.
\item If we wanted to calculate the probability that a random sample of 8 people will have 3 that identify as Democrat, briefly describe why the approach from part (b) would be more tedious than the approach from part (a).
\end{parts}
}{}

% 31

\eoce{\qt{Underage drinking, Part I\label{underage_drinking_intro}} \ \videohref{ahss_eoce_sol-underage_drinking_intro}\ \ 
Data collected by the Substance Abuse and Mental Health
Services Administration (SAMSHA) suggests that 69.7\% of
18-20 year olds consumed alcoholic beverages in any given
year.\footfullcite{webpage:alcohol}
\begin{parts}
\item Suppose a random sample of ten 18-20 year olds is taken. Is the use 
of the binomial distribution appropriate for calculating the probability that 
exactly six consumed alcoholic beverages? Explain.
\item Calculate the probability that exactly 6 out of 10 randomly sampled 18-
20 year olds consumed an alcoholic drink.
\item What is the probability that exactly four out of ten 18-20 year 
olds have \textit{not} consumed an alcoholic beverage?
\item What is the probability that at most 2 out of 5 randomly sampled 18-20 
year olds have consumed alcoholic beverages?
\item What is the probability that at least 1 out of 5 randomly sampled 18-20 
year olds have consumed alcoholic beverages?
\end{parts}
}{}

% 32

\eoce{\qt{Chicken pox, Part I\label{chicken_pox_intro}} The National Vaccine 
Information Center estimates that 90\% of Americans have had chickenpox by 
the time they reach adulthood.  \footfullcite{webpage:chickenpox}
\begin{parts}
\item Suppose we take a random sample of 100 American adults. Is the use of 
the binomial distribution appropriate for calculating the probability that exactly 97 
out of 100 randomly sampled American adults had chickenpox during childhood? Explain.
\item Calculate the probability that exactly 97 out of 100 randomly sampled 
American adults had chickenpox during childhood.
\item What is the probability that exactly 3 out of a new sample of 100 
American adults have \textit{not} had chickenpox in their childhood?
\item What is the probability that at least 1 out of 10 randomly sampled 
American adults have had chickenpox?
\item What is the probability that at most 3 out of 10 randomly sampled 
American adults have \textit{not} had chickenpox?
\end{parts}
}{}

