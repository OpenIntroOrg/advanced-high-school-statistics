\exercisesheader{}

% 1 - dream_act_mosaic

\eoce{\qt{Views on the DREAM Act\label{dream_act_mosaic}} A random sample of registered 
voters from Tampa, FL were asked if they support the DREAM Act, a proposed law which would provide a path to citizenship for people brought illegally to the US as children.
The survey also collected information on the political ideology of the respondents. 
Based on the mosaic plot shown below, do views on the DREAM Act and  
political ideology appear to be independent? Explain your reasoning.
\footfullcite{survey:immigFL:2012}
\begin{center}
\FigureFullPath[A mosaic plot is shown. The square (or, more accurately, a rectangle in this case), is divided into three main categories as tall rectangles: Conservative (about 40\% of the data), Moderate (about 40\% of the data), and Liberal (about 20\%). The tall rectangles are each divided into "Support", "Not Support", and "Not Sure".  The "Support" category is about 45-50\% for the Conservative and Moderate political groups and about 60-65\% for Liberal. The "Not Support" category is about 40-45\% for the Conservative and Moderate groups, while it is about 30\% for the Liberal group. In all three of the main groupings, "Not sure" is about the same, representing about 5-10\% of each political categories.]{0.8}{ch_probability/figures/eoce/dream_act_mosaic/dream_act_mosaic}
\end{center}
}{}

% 2 - raise_taxes_mosaic

\eoce{\qt{Raise taxes\label{raise_taxes_mosaic}} A random sample of registered 
voters nationally were asked whether they think it's better to raise taxes 
on the rich or raise taxes on the poor. The survey also collected information 
on the political party affiliation of the respondents. Based on the mosaic 
plot shown below, do views on raising taxes and  
political affiliation appear to be independent? Explain your reasoning.
\footfullcite{survey:raiseTaxes:2015}
\begin{center}
\FigureFullPath[A mosaic plot is shown for variables of political affiliation (main variable split) and opinion on whether to raise taxes on the rich, poor, or not sure. The political split, representing the main vertical splits in the mosaic plot, is roughly evenly split between Democrat, Republican, and Independent/Other, with perhaps a little more respondents in the Democrat group. The very large portion of the Democrat group -- about 85\% -- overwhelmingly supports raising taxes on the rich, with only about 5\% of this group supports raising taxes on the poor, and 5\% are unsure. About 45-50\% of the Republican and Independent/Other groups each support raising taxes on the rich, about 10\% of these groups support raising taxes on the poor, and about 40-45\% of each of these groups are not sure.]{0.75}{ch_probability/figures/eoce/raise_taxes_mosaic/raise_taxes_mosaic}
\end{center}
}{}

% 3 - side_effects_avandia_ahss

\eoce{\qt{Side effects of Avandia, Part I\label{side_effects_avandia_ahss}} Rosiglitazone is the 
active ingredient in the controversial type~2 diabetes medicine Avandia and has 
been linked to an increased risk of serious cardiovascular problems such as 
stroke, heart failure, and death. A common alternative treatment is pioglitazone, 
the active ingredient in a diabetes medicine called Actos. In a nationwide 
retrospective observational study of 227,571 Medicare beneficiaries aged  
65 years or older, it was found that 2,593 of the 67,593 patients using 
rosiglitazone and 5,386 of the 159,978 using pioglitazone had serious 
cardiovascular problems. These data are summarized in the contingency 
table below. \footfullcite{Graham:2010}
\begin{center}
\begin{tabular}{ll  cc c} 
                                &   & \multicolumn{2}{c}{\textit{Cardiovascular problems}} \\
\cline{3-4} 
                                    &               & Yes   & No        & Total \\
\cline{2-5}
\multirow{2}{*}{\textit{Treatment}} & Rosiglitazone & 2,593 & 65,000    & 67,593 \\
                                    & Pioglitazone  & 5,386 & 154,592   & 159,978 \\
\cline{2-5}
                                    & Total         & 7,979 & 219,592   & 227,571
\end{tabular}
\end{center}
Determine if each of the following statements is true or false. If false, explain why. \textit{Be careful:} The reasoning may be wrong even if the statement's conclusion is correct. In such cases, the statement should be considered false.
\begin{parts}
\item Since more patients on pioglitazone had cardiovascular problems (5,386 vs. 2,593), we can conclude that the rate of cardiovascular problems for those on a pioglitazone treatment is higher.
\item The data suggest that diabetic patients who are taking rosiglitazone are more likely to have cardiovascular problems since the rate of incidence was (2,593 / 67,593 = 0.038) 3.8\% for patients on this treatment, while it was only (5,386 / 159,978 = 0.034) 3.4\% for patients on pioglitazone.
\item The fact that the rate of incidence is higher for the rosiglitazone group proves that rosiglitazone causes serious cardiovascular problems.
\item Based on the information provided so far, we cannot tell if the difference between the rates of incidences is due to a relationship between the two variables or due to chance.
\end{parts}
}{}

% 4 - immigration_contingency_table

\eoce{\qt{Views on immigration\label{immigration}} 910 randomly sampled registered 
voters from Tampa, FL were asked if they thought undocumented workers in the US should be (i) allowed to keep their jobs and apply for 
US citizenship, (ii) allowed to keep their jobs as temporary guest workers 
but not allowed to apply for US citizenship, or (iii) lose their jobs and 
have to leave the country. The results of the survey by political ideology 
are shown below.\footfullcite{survey:immigFL:2012}
\begin{center}
\begin{tabular}{l l c c c c}
                        &                           & \multicolumn{3}{c}{\textit{Political ideology}} \\
\cline{3-5}
                        &                           & Conservative  & Moderate  & Liberal   & Total \\
\cline{2-6}
                        & (i) Apply for citizenship & 57            & 120       & 101       & 278 \\
                        & (ii) Guest worker         & 121           & 113       & 28        & 262 \\
\raisebox{1.5ex}[0pt]{\emph{Response}} & (iii) Leave the country    & 179       & 126       & 45        & 350 \\ 
                        & (iv) Not sure             & 15            & 4         & 1         & 20\\
\cline{2-6}
                        & Total                     & 372           & 363       & 175       & 910
\end{tabular}
\end{center}
\begin{parts}
\item What percent of these Tampa, FL voters identify themselves as conservatives?
\item What percent of these Tampa, FL voters are in favor of the citizenship option?
\item What percent of these Tampa, FL voters identify themselves as conservatives 
and are in favor of the citizenship option?
\item What percent of these Tampa, FL voters who identify themselves as 
conservatives are also in favor of the citizenship option? What percent of 
moderates share this view? What percent of liberals share this view?
\item Do political ideology and views on immigration appear to be independent? 
Explain your reasoning.
\end{parts}
}{}

% 5 - heart_transplant_data_display

\eoce{\qt{Heart transplant data display\label{heart_transplant_data_display}} The Stanford University Heart Transplant Study was conducted to determine whether an experimental heart transplant program increased lifespan. Each patient entering the program was officially designated a heart transplant candidate, meaning that they were gravely ill and might benefit from a new heart. Patients were randomly assigned into treatment and control groups. Patients in the treatment group received a transplant, and those in the control group did not. The visualizations below display two different versions of the study results.\footfullcite{Turnbull+Brown+Hu:1974}\vspace{-2mm}
\begin{center}
\includegraphics[width=0.85\textwidth]{ch_probability/figures/eoce/heart_transplant_data_display/heart_transplant_bar_graphs}
\end{center}
\begin{parts}
\item Provide one aspect of the two group comparison that is easier to see from the stacked bar plot (left)?

\item Provide one aspect of the two group comparison that is easier to see from the standardized bar plot (right)?

\item For the Heart Transplant Study which of those aspects would be more important to display? That is, which bar plot would be better as a data visualization?

\end{parts}
}{}

% 6 - shipping_holiday_gifts_data_display

\eoce{\qt{Shipping holiday gifts data display\label{shipping_holiday_gifts_data_display}} A local news survey asked 500 randomly sampled Los Angeles residents which shipping carrier they prefer to use for shipping holiday gifts. The bar plots below show the distribution of responses by age group as well as distribution of responses by shipping method.
\begin{center}
\includegraphics[width=0.85\textwidth]{ch_probability/figures/eoce/shipping_holiday_gifts_data_display/shipping_holiday_gifts_data_display}
\end{center}
\begin{parts}
\item Which graph (top or bottom) would you use to understand the shipping choices of people of different ages? Explain.

\item Which graph (top or bottom) would you use to understand the age distribution across different types of shipping choices? Explain.
\item A new shipping company would like to market to people over the age of 55. Who will be their biggest competitor? Explain.

\item FedEx would like to reach out to grow their market share so as to balance the age demographics of FedEx users. To what age group should FedEx market?
\end{parts}








}{}
