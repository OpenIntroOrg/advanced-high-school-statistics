\exercisesheader{}

% 1 - distribution_p_hat_1

\eoce{\qt{Distribution of $\hat{p}$\label{distribution_p_hat_1}} Suppose the true population proportion were $p = 0.95$. The figure below shows what the distribution of a sample proportion looks like when the sample size is $n = 20$, $n = 100$, and $n = 500$. (a) What does each point (observation) in each of the samples represent? (b) Describe the distribution of the sample proportion, $\hat{p}$. How does the distribution of the sample proportion change as $n$ becomes larger?
\begin{center}
\includegraphics[width=0.85\textwidth]{ch_distributions/figures/eoce/distribution_p_hat_1/eoce-p-hat-simulations-p95}
\end{center}
}{}

% 2 - distribution_p_hat_2

\eoce{\qt{Distribution of $\hat{p}$\label{distribution_p_hat_2}} Suppose the true population proportion were $p = 0.5$. The figure below shows what the distribution of a sample proportion looks like when the sample size is $n = 20$, $n = 100$, and $n = 500$. What does each point (observation) in each of the samples represent? Describe how the distribution of the sample proportion, $\hat{p}$, changes as $n$ becomes larger.
\begin{center}
\includegraphics[width=0.85\textwidth]{ch_distributions/figures/eoce/distribution_p_hat_2/eoce-p-hat-simulations-p5}
\end{center}
}{}

\D{\newpage}
% 3 - distribution_p_hat_3

\eoce{\qt{Distribution of $\hat{p}$\label{distribution_p_hat_3}} \videosolution{ahss_eoce_sol-distribution_of_phat} Suppose the true population proportion were $p = 0.5$ and a researcher takes a simple random sample of size $n=50$.  
\begin{parts}
\item Find and interpret the standard deviation of the sample proportion $\hat{p}$.  
\item Calculate the probability that the sample proportion will be larger than 0.55 for a random sample of size 50.
\end{parts}
}{}

% 4 - distribution_p_hat_4

\eoce{\qt{Distribution of $\hat{p}$\label{distribution_p_hat_4}} Suppose the true population proportion were $p = 0.6$ and a researcher takes a simple random sample of size $n=50$.  
\begin{parts}
\item Find and interpret the standard deviation of the sample proportion $\hat{p}$. 
\item Calculate the probability that the sample proportion will be larger than 0.65 for a random sample of size 50.
\end{parts}
}{}

% 5 - nearsighted_children

\eoce{\qt{Nearsighted children\label{nearsighted_children}} \videosolution{ahss_eoce_sol-nearsighted_children} It is believed that nearsightedness affects about 8\% of all children. We are interested in finding the probability that fewer than 12 out of 200 randomly sampled children will be nearsighted.
\begin{parts}
\item Estimate this probability using the normal approximation to the binomial distribution.
\item Estimate this probability using the distribution of the sample proportion.
\item How do your answers from parts (a) and (b) compare?
\end{parts}
}{}

% 6 - social_network_use

\eoce{\qt{Social network use\label{social_network_use}} The Pew Research Center estimates that as of January 2014, 89\% of 18-29 year olds in the United States use social networking sites.\footfullcite{data:pewsocialnetwork:2014} Calculate the probability that at least 95\% of 500 randomly sampled 18-29 year olds use social networking sites.
}{}
