\exercisesheader{}

% 22 - difference_of_props_prob_1

\eoce{\qt{Difference of proportions, Part 1\label{difference_of_means_prob_1}}
The fraction of workers who are considered ``supercommuters'',
because they commute more than 90 minutes to get to work, varies by state.
Suppose the following were the exact values for Nebraska and New York:
\begin{center}
\begin{tabular}{cc}
State & Proportion Supercommuters \\
\hline
Nebraska & 0.01 \\
New York & 0.06 \\
\hline
\end{tabular}
\end{center}
Now suppose that we plan a study to survey 1000 people from each state,
and we will compute the sample proportions $\hat{p}_{NE}$ for Nebraska
and $\hat{p}_{NY}$ for New York.
\begin{parts}
\item What is the associated mean and standard deviation of $\hat{p}_{NE}$?
\item What is the associated mean and standard deviation of $\hat{p}_{NY}$?
\item Calculate and interpret the mean and standard deviation associated with the difference in sample proportions for the two groups, $\hat{p}_{NY} - \hat{p}_{NE}$.
\item How are the standard deviations from parts~(a), (b), and~(c) related?
\end{parts}
}{}

% 23 - difference_of_props_prob_2

\eoce{\qt{Difference of proportions, Part 2\label{difference_of_means_prob_2}}
The fraction of workers who are considered ``supercommuters'',
because they commute more than 90 minutes to get to work, varies by state.
Suppose the following were the exact values for Nebraska and New York:
\begin{center}
\begin{tabular}{cc}
State & Proportion Supercommuters \\
\hline
Nebraska & 0.01 \\
New York & 0.06 \\
\hline
\end{tabular}
\end{center}
Now suppose that we plan a study to survey 1000 people from each state,
and we will compute the sample proportions $\hat{p}_{NE}$ for Nebraska
and $\hat{p}_{NY}$ for New York.
\begin{parts}
\item What distribution is associated with the difference $\hat{p}_{NY} - \hat{p}_{NE}$?
  Justify your answer.
\item Determine the probability that $\hat{p}_{NY} - \hat{p}_{NE}$ will be larger than 0.055.
\item Determine the probability that $\hat{p}_{NY} - \hat{p}_{NE}$ will be smaller than 0.4.
\end{parts}
}{}

% 24 - difference_of_means_prob_1

\eoce{\qt{Difference of means, Part 1\label{difference_of_means_prob_1}}
Suppose we will collect two random samples from the following distributions:
\begin{center}
\begin{tabular}{cccc}
& Mean & Standard Deviation & Sample Size \\
\hline
Sample 1 & 15 & 20 & 50 \\
Sample 2 & 20 & 10 & 30 \\
\hline
\end{tabular}
\end{center}
In each of the parts below, consider the sample means $\bar{x}_1$ and $\bar{x}_2$ that we might observe from these two samples.
\begin{parts}
\item What is the associated mean and standard deviation of $\bar{x}_1$?
\item What is the associated mean and standard deviation of $\bar{x}_2$?
\item Calculate and interpret the mean and standard deviation associated with the difference in sample means for the two groups, $\bar{x}_2 - \bar{x}_1$.
\item How are the standard deviations from parts~(a), (b), and~(c) related?
\end{parts}
}{}

% 25 - difference_of_means_prob_2

\eoce{\qt{Difference of means, Part 2\label{difference_of_means_prob_2}}
Suppose we will collect two random samples from the following distributions:
\begin{center}
\begin{tabular}{cccc}
& Mean & Standard Deviation & Sample Size \\
\hline
Sample 1 & 15 & 20 & 50 \\
Sample 2 & 20 & 10 & 30 \\
\hline
\end{tabular}
\end{center}
In each of the parts below, consider the sample means $\bar{x}_1$ and $\bar{x}_2$ that we might observe from these two samples.
\begin{parts}
\item What distribution is associated with the difference $\bar{x}_2 - \bar{x}_1$?
  Justify your answer.
\item Determine the probability that $\bar{x}_2 - \bar{x}_1$ will be larger than 7.
\item Determine the probability that $\bar{x}_2 - \bar{x}_1$ will be smaller than 3.
\item Determine the probability that $\bar{x}_2 - \bar{x}_1$ will be smaller than 0.
\end{parts}
}{}
