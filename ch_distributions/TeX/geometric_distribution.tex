\exercisesheader{}

% 29

\eoce{\qtq{Is it Bernoulli\label{is_it_bernouilli}} Determine if each trial can be 
considered an independent Bernoulli trial for the following situations.
\begin{parts}
\item Cards dealt in a hand of poker.
\item Outcome of each roll of a die.
\end{parts}
}{}

% 30

\eoce{\qt{With and without replacement\label{with_without_replacement}} In the 
following situations assume that half of the specified population is male and 
the other half is female.
\begin{parts}
\item Suppose you're sampling from a room with 10 people. What is the 
probability of sampling two females in a row when sampling with replacement? 
What is the probability when sampling without replacement?
\item Now suppose you're sampling from a stadium with 10,000 people. What is 
the probability of sampling two females in a row when sampling with 
replacement? What is the probability when sampling without replacement?
\item We often treat individuals who are sampled from a large population as 
independent. Using your findings from parts~(a) and~(b), explain whether or 
not this assumption is reasonable.
\end{parts}
}{}

% 31

\eoce{\qt{Eye color, Part I\label{eye_color_geometric}} A husband and wife both 
have brown eyes but carry genes that make it possible for their children to 
have brown eyes (probability 0.75), blue eyes (0.125), or green eyes (0.125).
\begin{parts}
\item What is the probability the first blue-eyed child they have is their 
third child? Assume that the eye colors of the children are independent of 
each other.
\item On average, how many children would such a pair of parents have before 
having a blue-eyed child? What is the standard deviation of the number of 
children they would expect to have until the first blue-eyed child?
\end{parts}
}{}

% 32

\eoce{\qt{Defective rate\label{defective_rate}} A machine that produces a special 
type of transistor (a component of computers) has a 2\% defective rate. The 
production is considered a random process where each transistor is 
independent of the others.
\begin{parts}
\item What is the probability that the $10^{th}$ transistor produced is the 
first with a defect?
\item What is the probability that the machine produces no defective 
transistors in a batch of 100?
\item On average, how many transistors would you expect to be produced before 
the first with a defect? What is the standard deviation?
\item Another machine that also produces transistors has a 5\% defective rate 
where each transistor is produced independent of the others. On average how 
many transistors would you expect to be produced with this machine before the 
first with a defect? What is the standard deviation?
\item Based on your answers to parts (c) and (d), how does increasing the 
probability of an event affect the mean and standard deviation of the wait 
time until success?
\end{parts}
}{}

% 33

\eoce{\qt{Bernoulli, the mean\label{bernoulli_mean_derivation}}
Use the probability rules from
Section~\ref{randomVariablesSection}
to derive the mean of a Bernoulli random variable,
i.e. a random variable $X$ that takes value 1
with probability $p$ and value 0 with probability $1 - p$.
That is, compute the expected value of a generic
Bernoulli random variable.
}{}

% 34

\eoce{\qt{Bernoulli, the standard deviation\label{bernoulli_sd_derivation}}
Use the probability rules from
Section~\ref{randomVariablesSection}
to derive the standard deviation of a Bernoulli random variable,
i.e. a random variable $X$ that takes value 1
with probability $p$ and value 0 with probability $1 - p$.
That is, compute the square root of the variance of a generic
Bernoulli random variable.
}{}
