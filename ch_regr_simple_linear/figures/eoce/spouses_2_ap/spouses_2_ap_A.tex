\begin{parts}
\item The hypotheses are as follows:
\begin{hyp}
\item[] $H_0$: Height of woman's spouse has no linear relationship with height of woman
($\beta = 0$).
\item[] $H_A$: Height of woman's spouse has a positive, linear relationship with height of woman
 ($\beta > 0$).
\end{hyp}
The test-statistic is 4.17 (with df = 170 - 2 = 168), and p-value for a 
two-sided alternative hypothesis is approximately 0. The p-value for a one-
tailed alternative will also be very low. With such a low p-value, we 
reject $H_0$ and conclude the data provide convincing evidence that womens' 
heights are positively correlated with their spouses' heights and the true slope 
parameter is indeed greater than 0.
\item The equation of the regression line is
\[ \widehat{height}_{S} = 43.5755 + 0.2863 \times height_{W}. \]
\item Slope: For each additional inch in woman's height, the average 
spouse's height is expected to be an additional 0.2863 inches on average. \\
Intercept: Women who are 0 inches tall are expected to have spouses who are on 
average 43.5755 inches tall. The intercept here is meaningless, and it 
serves only to adjust the height of the line.
\item The slope is positive, so $R$ must also be positive. 
$R = \sqrt{0.09} = 0.30$.
\item Using the equation of the regression line: \\
\[ \widehat{height}_{S} = 43.5755 + 0.2863 \times 69 = 63.2612. \]
Since $R^2$ is low, the prediction based on this regression model is not 
very reliable.
\item No, we shouldn't use the same model to predict the height of the 
spouse of a woman who is 79 inches tall. The scatterplot shows that women 
in this data set are approximately 55 to 70 inches tall. The regression 
model may not be reasonable outside of this range.
\end{parts}

