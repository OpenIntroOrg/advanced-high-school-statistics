\reviewexercisesheader{}

% 20 - chronic_illness_tf

\eoce{\qt{Chronic illness, Part II\label{chronic_illness_tf}} In 2013, the Pew Research Foundation reported that 
``45\% of U.S. adults report that they live with one or more chronic 
conditions'', and the standard error for this estimate is 1.2\%. Identify each 
of the following statements as true or false. Provide an explanation to justify 
each of your answers.
\begin{parts}
\item We can say with certainty that the confidence interval from 
Exercise~\ref{chronic_illness_intro} contains the true percentage of U.S. adults who 
suffer from a chronic illness.
\item If we repeated this study 1,000 times and constructed a 95\% confidence 
interval for each study, then approximately 950 of those confidence intervals 
would contain the true fraction of U.S. adults who suffer from chronic illnesses.
\item The poll provides statistically significant evidence (at the 
$\alpha = 0.05$ level) that the percentage of U.S. adults who suffer from 
chronic illnesses is below 50\%.
\item Since the standard error is 1.2\%, only 1.2\% of people in the study 
communicated uncertainty about their answer.
\end{parts}
}{}

% 21 - twitter_users_tf

\eoce{\qt{Twitter users and news, Part II\label{twitter_users_tf}} A poll conducted in 2013 found that 52\% of 
U.S. adult Twitter users get at least some news on Twitter, and the standard 
error for this estimate was 2.4\%. Identify each of the following statements as 
true or false. Provide an explanation to justify each of your answers.
\begin{parts}
\item The data provide statistically significant evidence that more than half of 
U.S. adult Twitter users get some news through Twitter. Use a significance level 
of $\alpha = 0.01$.
\item Since the standard error is 2.4\%, we can conclude that 97.6\% of all U.S. 
adult Twitter users were included in the study.
\item If we want to reduce the standard error of the estimate, we should collect 
less data.
\item If we construct a 90\% confidence interval for the percentage of U.S. 
adults Twitter users who get some news through Twitter, this confidence interval 
will be wider than a corresponding 99\% confidence interval.
\end{parts}
}{}

% 22 - relax_after_work

\eoce{\qt{Relaxing after work\label{relax_after_work}} The General Social Survey asked the question:
``After an average work day, about how many hours do you have to relax or pursue 
activities that you enjoy?" to a random sample of 1,155 Americans.\footfullcite{data:gss} A 95\% confidence interval for the mean number of hours spent 
relaxing or pursuing activities they enjoy was (1.38, 1.92).
\begin{parts}
\item Interpret this interval in context of the data.
\item Suppose another set of researchers reported a confidence interval with a 
larger margin of error based on the same sample of 1,155 Americans. How does 
their confidence level compare to the confidence level of the interval stated 
above?
\item Suppose next year a new survey asking the same question is conducted, and 
this time the sample size is 2,500. Assuming that the population 
characteristics, with respect to how much time people spend relaxing after work, 
have not changed much within a year. How will the margin of error of the 95\% 
confidence interval constructed based on data from the new survey compare to the 
margin of error of the interval stated above?
\end{parts}
}{}

% 23 - errors_food_safety

\eoce{\qt{Testing for food safety\label{errors_food_safety}} A food safety inspector 
is called upon to investigate a restaurant with a few customer reports of poor 
sanitation practices. The food safety inspector uses a hypothesis testing 
framework to evaluate whether regulations are not being met. If he decides 
the restaurant is in gross violation, its license to serve food will be revoked.
\begin{parts}
\item Write the hypotheses in words.
\item What is a Type~1 Error in this context?
\item What is a Type~2 Error in this context?
\item Which error is more problematic for the restaurant owner? Why?
\item Which error is more problematic for the diners? Why?
\item As a diner, would you prefer that the food safety inspector requires 
strong evidence or very strong evidence of health concerns before revoking a 
restaurant's license? Explain your reasoning.
\end{parts}
}{}

% 24 - tf_found_inf_prop_friendly

\eoce{\qt{True or false\label{tf_found_inf_prop_friendly}}
Determine if the following statements are true or false, and 
explain your reasoning. If false, state how it could be corrected.
\begin{parts}
\item If a given value (for example, the null hypothesized value of a parameter) 
is within a 95\% confidence interval, it will also be within a 99\% confidence 
interval.
\item Decreasing the significance level ($\alpha$) will increase the probability 
of making a Type~1 Error.
\item Suppose the null hypothesis is $p = 0.5$ and we fail to reject $H_0$. 
Under this scenario, the true population proportion is 0.5.
\item With large sample sizes, even small differences between the null value and 
the observed point estimate, a difference often called the
effect size\index{effect size}, will be identified as statistically significant.
\end{parts}
}{}

% 25 - prac_stat_sig

\eoce{\qt{Practical vs. statistical significance\label{prac_stat_sig}}
Determine whether the following statement is true
or false, and explain your reasoning:
``With large sample sizes, even small differences
between the null value and the observed point
estimate can be statistically significant.''
}{}

% 26 - same_obs_diff_n

\eoce{\qt{Same observation, different sample size\label{same_obs_diff_n}} Suppose you 
conduct a hypothesis test based on a sample where the sample size is $n = 50$, 
and arrive at a p-value of 0.08. You then refer back to your notes and discover 
that you made a careless mistake, the sample size should have been $n = 500$. 
Will your p-value increase, decrease, or stay the same? Explain.
}{}
