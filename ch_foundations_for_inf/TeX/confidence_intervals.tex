\exercisesheader{}

% 7 - chronic_illness_intro

\eoce{\qt{Chronic illness, Part I\label{chronic_illness_intro}} 
In 2013, the Pew Research Foundation reported that ``45\% of U.S. adults report 
that they live with one or more chronic conditions''.
\footfullcite{data:pewdiagnosis:2013} However, this value was based on a sample, 
so it may not be a perfect estimate for the population parameter of interest on 
its own. The study reported a standard error of about 1.2\%, and a normal model 
may reasonably be used in this setting. Create a 95\% confidence interval for 
the proportion of U.S. adults who live with one or more chronic conditions. Also 
interpret the confidence interval in the context of the study.
}{}

% 8 - twitter_users_intro

\eoce{\qt{Twitter users and news, Part I\label{twitter_users_intro}} 
A poll conducted in 2013 found that 52\% of U.S. adult Twitter users 
get at least some news on Twitter.\footfullcite{data:pewtwitternews:2013}. 
The standard error for this estimate was 2.4\%, and a normal distribution 
may be used to model the sample proportion. Construct a 99\% confidence 
interval for the fraction of U.S. adult Twitter users who get some 
news on Twitter, and interpret the confidence interval in context.
}{}

% 9 - er_wait_intro_prop_ok

\eoce{\qt{Waiting at an ER, Part I\label{er_wait_intro_prop_ok}} A hospital administrator 
hoping to improve wait times decides to estimate the average emergency 
room waiting time at her hospital. She collects a simple random sample 
of 64 patients and determines the time (in minutes) between when they 
checked in to the ER until they were first seen by a doctor. A 95\% 
confidence interval based on this sample is (128 minutes, 147 minutes), 
which is based on the normal model for the mean. Determine whether the 
following statements are true or false, and explain your reasoning.
\begin{parts}
\item We are 95\% confident that the average waiting time of these 64 emergency 
room patients is between 128 and 147 minutes.
\item We are 95\% confident that the average waiting time of all patients at 
this hospital's emergency room is between 128 and 147 minutes.
\item 95\% of random samples have a sample mean between 128 and 147 minutes.
\item A 99\% confidence interval would be narrower than the 95\% confidence 
interval since we need to be more sure of our estimate.
\item The margin of error is 9.5 and the sample mean is 137.5.
\item In order to decrease the margin of error of a 95\% confidence interval to 
half of what it is now, we would need to double the sample size.
\end{parts}
}{}

% 10 - mental_health

\eoce{\qt{Mental health\label{mental_health}}
The General Social Survey asked the question:
``For how many days during the past 30 days was your 
mental health, which includes stress, depression,
and problems with emotions, not good?"
Based on responses from 1,151 US residents,
the survey reported a 95\% confidence interval of
3.40 to 4.24 days in 2010.
\begin{parts}
\item
    Interpret this interval in context of the data.
\item
    What does ``95\% confident" mean? Explain in the
    context of the application.
\item
    Suppose the researchers think a 99\% confidence level
    would be more appropriate for this interval.
    Will this new interval be smaller or wider than the
    95\% confidence interval?
\item
    If a new survey were to be done with 500 Americans,
    do you think the standard error of the estimate be
    larger, smaller, or about the same.
\end{parts}
}{}
