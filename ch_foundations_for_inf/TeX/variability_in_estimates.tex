\exercisesheader{}

% 1

\eoce{\qt{Identify the parameter, Part I\label{identify_parameter_1}} For each of the following situations, state 
whether the parameter of interest is a mean or a proportion. It may be helpful 
to examine whether individual responses are numerical or categorical.
\begin{parts}
\item In a survey, one hundred college students are asked how many hours per 
week they spend on the Internet.
\item In a survey, one hundred college students are asked: ``What percentage of 
the time you spend on the Internet is part of your course work?"
\item In a survey, one hundred college students are asked whether or not they 
cited information from Wikipedia in their papers.
\item In a survey, one hundred college students are asked what percentage of 
their total weekly spending is on alcoholic beverages.
\item In a sample of one hundred recent college graduates, it is found that 85 
percent expect to get a job within one year of their graduation date.
\end{parts}
}{}

% 2

\eoce{\qt{Identify the parameter, Part II\label{identify_parameter_2}} For each of the 
following situations, state whether the parameter of interest is a mean or a 
proportion. 
\begin{parts}
\item A poll shows that 64\% of Americans personally worry a great deal about 
federal spending and the budget deficit.
\item A survey reports that local TV news has shown a 17\% increase in revenue 
within a two year period while newspaper revenues decreased by 6.4\% during this 
time period.
\item In a survey, high school and college students are asked whether or not 
they use geolocation services on their smart phones.
\item In a survey, smart phone users are asked whether or not they use a web-based taxi service.
\item In a survey, smart phone users are asked how many times they used a web-based taxi service over the last year.
\end{parts}
}{}

% 3

\eoce{\qt{Quality control\label{comp_chips_quality_ctrl_prop}}
As part of a quality control process for computer chips,
an engineer at a factory randomly samples 212 chips
during a week of production to test the current rate of
chips with severe defects.
She finds that 27 of the chips are defective.
\begin{parts}
\item
    What population is under consideration in the data set?
\item
    What parameter is being estimated?
\item\label{comp_chips_quality_ctrl_prop_pt_est}%
    What is the point estimate for the parameter?
\item\label{comp_chips_quality_ctrl_prop_se_name}%
    What is the name of the statistic can we use to measure
    the uncertainty of the point estimate?
\item\label{comp_chips_quality_ctrl_prop_se_calc_w_pt_est}%
    Compute the value from
    part~(\ref{comp_chips_quality_ctrl_prop_se_name})
    for this context.
\item
    The historical rate of defects is 10\%.
    Should the engineer be surprised by the observed
    rate of defects during the current week?
\item
    Suppose the true population value was found to be 10\%.
    If we use this proportion to recompute the value in
    part~(\ref{comp_chips_quality_ctrl_prop_se_calc_w_pt_est})
    using $p = 0.1$ instead of $\hat{p}$,
    does the resulting value change much?
\end{parts}
}{}

% 4

\eoce{\qt{Unexpected expense\label{us_emergency_expense_prop}}
In a random sample 765 adults in the United States, 322 say
they could not cover a \$400 unexpected expense without borrowing
money or going into debt.
% Ref: https://www.federalreserve.gov/publications/files/2017-report-economic-well-being-us-households-201805.pdf
\begin{parts}
\item
    What population is under consideration in the data set?
\item
    What parameter is being estimated?
\item\label{us_emergency_expense_prop_pt_est}%
    What is the point estimate for the parameter?
\item\label{us_emergency_expense_prop_se_name}%
    What is the name of the statistic can we use to measure
    the uncertainty of the point estimate?
\item\label{us_emergency_expense_prop_se_calc_w_pt_est}%
    Compute the value from
    part~(\ref{us_emergency_expense_prop_se_name})
    for this context.
\item
    A cable news pundit thinks the value is actually 50\%.
    Should she be surprised by the data?
\item
    Suppose the true population value was found to be 40\%.
    If we use this proportion to recompute the value in
    part~(\ref{us_emergency_expense_prop_se_calc_w_pt_est})
    using $p = 0.4$ instead of $\hat{p}$,
    does the resulting value change much?
\end{parts}
}{}

% 5

\eoce{\qt{Repeated water samples\label{repeated_water_samples_prop}}
A nonprofit wants to understand the fraction of households that
have elevated levels of lead in their drinking water.
They expect at least 5\% of homes will have elevated levels of
lead, but not more than about 30\%.
They randomly sample 800 homes and work with the owners to retrieve
water samples, and they compute the fraction of these homes
with elevated lead levels.
They repeat this 1,000 times and build a distribution
of sample proportions.
\begin{parts}
\item
    What is this distribution called?
\item
    Would you expect the shape of this distribution to be
    symmetric, right skewed, or left skewed?
    Explain your reasoning.
\item
    If the proportions are distributed around 8\%,
    what is the variability of the distribution?
\item
    What is the formal name of the value you computed in~(c)?
\item
    Suppose the researchers' budget is reduced, and they are only
    able to collect 250 observations per sample, but they can still
    collect 1,000 samples.
    They build a new distribution of sample proportions.
    How will the variability of this new distribution compare
    to the variability of the distribution when each sample
    contained 800 observations?
\end{parts}
}{}

% 6

\eoce{\qt{Repeated student samples\label{repeated_student_samples_prop}}
Of all freshman at a large college, 16\% made the dean's list
in the current year.
As part of a class project, students randomly sample 40 students
and check if those students made the list.
They repeat this 1,000 times and build a distribution
of sample proportions.
\begin{parts}
\item
    What is this distribution called?
\item
    Would you expect the shape of this distribution to be
    symmetric, right skewed, or left skewed?
    Explain your reasoning.
\item
    Calculate the variability of this distribution.
\item
    What is the formal name of the value you computed in~(c)?
\item
    Suppose the students decide to sample again,
    this time collecting 90 students per sample,
    and they again collect 1,000 samples.
    They build a new distribution of sample proportions.
    How will the variability of this new distribution compare
    to the variability of the distribution when each sample
    contained 40 observations?
\end{parts}
}{}
