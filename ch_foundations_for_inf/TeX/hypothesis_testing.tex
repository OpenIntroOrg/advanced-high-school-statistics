\exercisesheader{}

% 13 - identify_hypotheses_prop_and_mean_1

\eoce{\qt{Identify hypotheses, Part I\label{
}}
Write the null and alternative hypotheses in words and then symbols
for each of the following  situations.
\begin{parts}
\item
    A tutoring company would like to understand if most
    students tend to improve their grades (or not) after
    they use their services.
    They sample 200 of the students who used their service
    in the past year and ask them if their grades have
    improved or declined from the previous year.
\item
    Employers at a firm are worried about the effect of March Madness,
    a basketball championship held each spring in the US, on employee
    productivity.
    They estimate that on a regular business day employees spend on
    average 15 minutes of company time checking personal email,
    making personal phone calls, etc.
    They also collect data on how much company time employees spend
    on such non-business activities during March Madness.
    They want to determine if these data provide convincing evidence
    that employee productivity changed during March Madness.
\end{parts}
}{}

% 14 - identify_hypotheses_prop_and_mean_2

\eoce{\qt{Identify hypotheses, Part II\label{identify_hypotheses_prop_and_mean_2}} 
Write the null and alternative hypotheses in words and using symbols 
for each of the following situations.
\begin{parts}
\item
    Since 2008, chain restaurants in California have been required
    to display calorie counts of each menu item. Prior to menus
    displaying calorie counts, the average calorie intake of diners
    at a restaurant was 1100 calories.
    After calorie counts started to be displayed on menus,
    a nutritionist collected data on the number of calories consumed
    at this restaurant from a random sample of diners.
    Do these data provide convincing evidence of a difference in the
    average calorie intake of a diners at this restaurant?
\item
    The state of Wisconsin would like to understand
    the fraction of its adult residents that consumed alcohol
    in the last year,
    specifically if the rate is different from the
    national rate of 70\%.
    To help them answer this question, they conduct
    a random sample of 852 residents and ask them
    about their alcohol consumption.
\end{parts}
}{}

% 15 - online_communication_prop_ht_errors

\eoce{\qt{Online communication\label{online_communication_prop_ht_errors}}
A study suggests that 60\% of college student spend
10~or more hours per week communicating with others online.
You believe that this is incorrect and decide to collect your 
own sample for a hypothesis test.
You randomly sample 160 students from your dorm
and find that 70\% spent 10~or more hours a week
communicating with others online.
A~friend of yours, who offers to help you with
the hypothesis test, comes up with the following
set of hypotheses.
Indicate any errors you see.
\begin{align*}
H_0&: \hat{p} < 0.6 \\
H_A&: \hat{p} > 0.7
\end{align*}
}{}

% 16 - married_at_25_prop_ht_errors

\eoce{\qt{Married at 25\label{married_at_25_prop_ht_errors}}
A study suggests that the 25\% of 25 year olds have
gotten married.
You believe that this is incorrect and decide to collect
your own sample for a hypothesis test.
From a random sample of 25 year olds in census data
with size 776,
you find that 24\% of them are married.
A friend of yours offers to help you with setting
up the hypothesis test and comes up with the following
hypotheses.
Indicate any errors you see.
\begin{align*}
H_0&: \hat{p} = 0.24 \\
H_A&: \hat{p} \neq 0.24
\end{align*}
}{}

\D{\newpage}

% 17 - unemployment_relationship

\eoce{\qt{Unemployment and relationship problems\label{unemployment_relationship}} 
A USA Today/Gallup poll asked a group of
unemployed and underemployed Americans if they have
had major problems in their  relationships with their
spouse or another close family member as a result of
not having a job (if unemployed) or not having
a full-time job (if underemployed).
27\%~of the 1,145 unemployed respondents and
25\%~of the 675 underemployed respondents said they had
major problems in relationships as a  result of their
employment status.
\begin{parts}
\item
    What are the hypotheses for evaluating if the proportions
    of unemployed and underemployed people who had relationship
    problems were different?
\item
    The p-value for this hypothesis test is approximately 0.35.
    Explain what this means in context of the hypothesis test
    and the data.
\end{parts}
}{}

% 18 - prop_which_higher_found_inf

\eoce{\qtq{Which is higher\label{prop_which_higher_found_inf}}
In each part below, there is a value of interest and two
scenarios (I and II).
For each part, report if the value of interest is larger
under scenario I, scenario II, or whether the value is
equal under the scenarios.
\begin{parts}
\item
     The standard error of $\hat{p}$ when
     (I)~$n = 125$ or (II)~$n = 500$.
\item
    The margin of error of a confidence interval
    when the confidence level is
    (I)~90\% or (II)~80\%.
\item
    The p-value for a Z-statistic of 2.5 calculated
    based on a (I)~sample with $n = 500$ or based on
    a (II)~sample with $n = 1000$.
\item
    The probability of making a Type~2 Error when the
    alternative hypothesis is true and the significance
    level is (I)~0.05 or (II)~0.10.
\end{parts}
}{}

% 19 - errors_fibromyalgia

\eoce{\qt{Testing for Fibromyalgia\label{errors_fibromyalgia}} A patient named Diana 
was diagnosed with Fibromyalgia, a long-term syndrome of body pain, and was 
prescribed anti-depressants. Being the skeptic that she is, Diana didn't 
initially believe that anti-depressants would help her symptoms. However after 
a couple months of being on the medication she decides that the 
anti-depressants are working, because she feels like her symptoms are in fact 
getting better.
\begin{parts}
\item Write the hypotheses in words for Diana's skeptical position when she 
started taking the anti-depressants.
\item What is a Type~1 Error in this context?
\item What is a Type~2 Error in this context?
\end{parts}
}{}
